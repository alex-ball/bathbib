% \iffalse meta-comment
%<*internal>
\iffalse
%</internal>
%<*bst>
%</bst>
%<*readme>
# bath-bst: Harvard referencing style as recommended by the University of Bath Library
%</readme>
%<*internal>
\fi
\def\nameofplainTeX{plain}
\ifx\fmtname\nameofplainTeX\else
  \expandafter\begingroup
\fi
%</internal>
%<*install>
\input docstrip.tex
\keepsilent
\askforoverwritefalse

\nopreamble\nopostamble

\usedir{doc/bibtex/\jobname}
\generate{
  \file{README.md}{\from{\jobname.dtx}{readme}}
  \file{\jobname.bib}{\from{\jobname.dtx}{bib}}
}

\preamble
----------------------------------------------------------------
bath-bst --- Harvard referencing style as recommended by the University of Bath Library
Author:  Alex Ball
E-mail:  a.j.ball@bath.ac.uk
License: Released under the LaTeX Project Public License v1.3c or later
See:     http://www.latex-project.org/lppl.txt
----------------------------------------------------------------

\endpreamble
\postamble

Copyright (C) 2016 by University of Bath
\endpostamble

\usedir{bibtex/bst/\jobname}
\generate{
  \file{bath.bst}{\from{\jobname.dtx}{bst}}
}
%</install>
%<install>\endbatchfile
%<*internal>
\usedir{source/bibtex/\jobname}
\generate{
  \file{\jobname.ins}{\from{\jobname.dtx}{install}}
}
\ifx\fmtname\nameofplainTeX
  \expandafter\endbatchfile
\else
  \expandafter\endgroup
\fi
%</internal>
%<*driver>
\ProvidesFile{bath-bst.dtx}
    [2016/10/24 v1.0 Harvard referencing style as recommended by the University of Bath Library]

\documentclass[10pt,a4paper]{article}
\usepackage[british]{babel}
\usepackage[hmargin=3cm,vmargin=2.5cm]{geometry}

% For typesetting the CJK example

\usepackage{iftex}
\ifPDFTeX
  \usepackage{CJKutf8}
\else
  \ifLuaTeX
    \usepackage{luatexja-fontspec}
    \setmainjfont{IPAexGothic}
  \else
    \ifXeTeX
    \usepackage{ctex}
    \fi
  \fi
\fi

% For typesetting the examples

\usepackage{xpatch,csquotes,xcolor,xparse,multicol,fancyvrb}
\xdefinecolor{Green}{rgb}{0,.5,0}
\xdefinecolor{Slate}{RGB}{80,86,94}
\xdefinecolor{BathStone}{RGB}{213,211,185}
\colorlet{ok}{Green}
\colorlet{bad}{red}
\colorlet{hacked}{orange}
\colorlet{manual}{purple}
\RecustomVerbatimEnvironment
  {Verbatim}{Verbatim}
  {commentchar=\%}

\usepackage[tightLists=false]{markdown}
\markdownSetup{rendererPrototypes={%
    link = {\href{#3}{#1}}%
}}

\usepackage{fontawesome}[2015/07/07]
\newcommand{\booksym}{\makebox[1em][c]{\faicon{book}}}
\newcommand{\cogsym}{\makebox[1em][c]{\faicon{cog}}}
\makeatletter
\newcommand{\hangfrom}[1]{%
  \setbox\@tempboxa\hbox{{#1}}%
  \hangindent \wd\@tempboxa\noindent\box\@tempboxa}
\makeatother

\usepackage{tcolorbox}
\tcbuselibrary{listings,breakable,skins,xparse}
\lstdefinelanguage{bafll}%
  { alsoletter={.}
  , morekeywords=[2]%
    { add.period$
    , call.type$
    , change.case$
    , chr.to.int$
    , cite$
    , duplicate$
    , empty$
    , format.name$
    , global.max$
    , if$
    , int.to.chr$
    , int.to.str$
    , missing$
    , newline$
    , num.names$
    , pop$
    , preamble$
    , purify$
    , quote$
    , skip$
    , sort.key$
    , stack$
    , substring$
    , swap$
    , text.length$
    , text.prefix$
    , top$
    , type$
    , warning$
    , while$
    , width$
    , write$
    }
  , morekeywords=[3]%
    { ENTRY
    , INTEGERS
    , STRINGS
    , MACRO
    , FUNCTION
    , READ
    , EXECUTE
    , ITERATE
    , SORT
    , REVERSE
    }
  , morekeywords=[1]{}
  , otherkeywords=%
    { +
    , -
    , >
    , <
    , =
    , *
    , :=
    }
  , sensitive=true
  , morestring=[b]"
  , morecomment=[l]\%
  }[keywords,strings,comments]
\lstloadlanguages{[LaTeX]TeX,bafll}
\lstdefinestyle{dtxlatex}%
  { columns=fullflexible
  , basicstyle=\ttfamily
  , language={[LaTeX]TeX}
  , texcsstyle=*\color{red!75!black}
  , commentstyle=\color{gray}\itshape
  , moretexcs=
    { citet
    , citep
    , noop
    , urlprefix
    , urldateprefix
    }
  , moredelim=**[s][\color{violet}]{[}{]}
  , moredelim=**[s][\color{blue!75!black}]{\{}{\}}
  }
\lstdefinestyle{dtxbst}%
  { columns=fullflexible
  , basicstyle=\ttfamily
  , language=bafll
  , keywordstyle=[1]\color{orange!75!black}
  , keywordstyle=[2]\color{red!75!black}
  , keywordstyle=[3]\color{green!50!black}
  , commentstyle=\color{gray}\itshape
  , stringstyle=\color{violet}
  , mathescape=false
  , showstringspaces=false
  }
\lstset{style=dtxlatex}
\tcbset
  { colframe = Slate
  , colback = BathStone!25
  , listing options =
    { style = tcblatex
    , style = dtxlatex
    , basicstyle=\ttfamily\small
    }
  }
\NewTColorBox{bibexbox}{D(){ok}d<>m}%
  {bicolor
  ,colframe = #1
  ,colback = #1!5!white
  ,colbacklower = white
  ,fontlower = \footnotesize
  ,before upper = {\hangfrom{\booksym\space}}
  ,after upper = {\par\hangfrom{\cogsym\space}\bibentry{#3}.}
  ,IfNoValueTF={#2}{}%
    {overlay = {
      \node[anchor=south east,text=teal] at (frame.south east) {#2};
      }
    }
  }

% For typesetting the implementation
\usepackage{doc}
\makeatletter
\newwrite\ydocwrite
\def\ydocfname{\jobname.listing}
\def\ydoc@catcodes{%
  \let\do\@makeother
  \dospecials
  \catcode`\\=\active
  \catcode`\^^M=\active
  \catcode`\ =\active
}
\def\macrocode{%
  \begingroup
  \ydoc@catcodes
  \macro@code
}
\def\endmacrocode{}
\begingroup
\endlinechar\m@ne
\@firstofone{%
\catcode`\|=0\relax
\catcode`\(=1\relax
\catcode`\)=2\relax
\catcode`\*=14\relax
\catcode`\{=12\relax
\catcode`\}=12\relax
\catcode`\ =12\relax
\catcode`\%=12\relax
\catcode`\\=\active
\catcode`\^^M=\active
\catcode`\ =\active
}*
|gdef|macro@code#1^^M%    \end{macrocode}(*
|endgroup|expandafter|macro@@code|expandafter(|ydoc@removeline#1|noexpand|lastlinemacro)*
)*
|gdef|ydoc@removeline#1^^M(|noexpand|firstlinemacro)*
|gdef|ydoc@defspecialmacros(*
|def^^M(|noexpand|newlinemacro)*
|def (|noexpand|spacemacro)*
|def\(|noexpand|bslashmacro)*
)*
|gdef|ydoc@defrevspecialmacros(*
|def|newlinemacro(|noexpand^^M)*
|def|spacemacro(|noexpand )*
|def|bslashmacro(|noexpand\)*
)*
|endgroup
\def\macro@@code#1{%
  {\ydoc@defspecialmacros
  \xdef\themacrocode{#1}}%
  \PrintMacroCode
  \end{macrocode}%
}
\def\PrintMacroCode{%
  \begingroup
  \let\firstlinemacro\empty
  \let\lastlinemacro\empty
  \def\newlinemacro{^^J}%
  \let\bslashmacro\bslash
  \let\spacemacro\space
  \immediate\openout\ydocwrite=\ydocfname\relax
  \immediate\write\ydocwrite{\themacrocode}%
  \immediate\closeout\ydocwrite
  \let\input\@input
  \tcbinputlisting{enhanced,breakable,size=small,listing only,listing file=\ydocfname}%
  \endgroup
}
\makeatother

% Documentation

\usepackage{natbib}
\newcommand*{\urlprefix}{Available from: }
\newcommand*{\urldateprefix}{Accessed }

\usepackage{bibentry}
\bibliographystyle{bath}
\nobibliography*

\usepackage{readprov}
\usepackage[british,cleanlook]{isodate}

\usepackage[colorlinks,citecolor=black]{hyperref}
\makeatletter
\Urlmuskip=0mu plus 3mu\relax
\mathchardef\UrlBigBreakPenalty=100\relax
\mathchardef\UrlBreakPenalty=200\relax
\def\UrlBigBreaks{\do\:\do\-}%
\def\UrlBreaks{%
  \do\.\do\@\do\/\do\\\do\!\do\_\do\|\do\;\do\>\do\]\do\)\do\}%
  \do\,\do\?\do\'\do\+\do\=\do\#\do\$\do\&\do\*\do\^\do\"}%
\let\do=\noexpand
\makeatother

\sloppy

\title{bath-bst: Harvard referencing style as recommended by the University of Bath Library}
\author{%
  Maintainer: Alex Ball\thanks{%
    To contact the maintainer about this package, please visit the repository
    where the code is hosted: \url{https://github.bath.ac.uk/ab318/bathbib}.%
  }%
}
\date{Package \UseVersionOf{bath-bst.dtx} --\printdateTeX{\UseDateOf{bath-bst.dtx}}}

\begin{document}
\maketitle

\section{Introduction}

\begin{markdown*}{hybrid=true}
%</driver>
%<*driver|readme>

This package provides a [BibTeX] style to format reference lists in the
[Harvard style][bath-harvard] recommended by the University of Bath Library.
It should be used in conjunction with [natbib] for citations.

## Installation

You can use this style simply by copying the `bath.bst` file into your working
directory, that is, the directory holding the main `.tex` file for your
document. If you want the style to be available for all your documents without
having to copy it over each time, you can install it using the instructions
below.

### Automated way

A makefile is provided which you can use with the Make utility on
UNIX-like systems:

  * Running `make source` generates the derived files
      - README.md
      - bath.bst
      - bath-bst.bib
      - bath-bst.ins
  * Running `make` generates the above files and also bath-bst.pdf.
  * Running `make inst` installs the files in the user's TeX tree.
    You can undo this with `make uninst`.
  * Running `make install` installs the files in the local TeX tree.
    You can undo this with `make uninstall`.

### Manual way

You only need to follow the first two steps if you have made your own
changes to the .dtx file. The compiled files you need are included in
the zip archive.

 1. Run `luatex bath-bst.dtx` to generate the source files.

 2. Compile bath-bst.dtx with [LuaLaTeX] and BibTeX to generate the
    documentation. You will need, among other things, the [luatexja],
    [adobemapping] and [ipaex] packages installed; this is just for the
    documentation, not for the BibTeX style itself.

 3. If you are using TeX Live, find your home TeX tree using the following
    command at the command prompt/terminal:

    ```
    kpsewhich -var-value=TEXMFHOME
    ```

    If you are using MikTeX, consult the MikTeX manual entry for [integrating
    local additions](http://docs.miktex.org/manual/localadditions.html). You
    can use one of the roots (TeX trees) already defined – preferably one of
    the User roots – or set up a new one.

 4. Move the files to your TeX tree as follows:
      - `source/bibtex/bath-bst`:
        bath-bst.dtx,
        (bath-bst.ins)
      - `bibtex/bst/bath-bst`:
        bath.bst
      - `doc/bibtex/bath-bst`:
        bath-bst.pdf,
        README.md

 5. You may then have to update your installation's file name database
    before TeX and friends can see the files.

[bath-harvard]: http://www.bath.ac.uk/library/infoskills/referencing-plagiarism/harvard-bath-style.html
[BibTeX]: http://ctan.org/pkg/bibtex
[natbib]: http://www.ctan.org/pkg/natbib
[LuaLaTeX]: http://ctan.org/pkg/lualatex-doc
[luatexja]: http://ctan.org/pkg/luatexja
[adobemapping]: http://ctan.org/pkg/adobemapping
[ipaex]: http://ctan.org/pkg/ipaex
%</driver|readme>
%<*driver>
\end{markdown*}

\section{Using the style}

To use the style, add these lines to your preamble:

\begin{tcblisting}{listing only}
\usepackage{natbib}
\newcommand*{\urlprefix}{Available from: }
\newcommand*{\urldateprefix}{Accessed }
\bibliographystyle{bath}
\end{tcblisting}

Remember also to specify your .bib file at the end of the document:

\begin{tcblisting}{listing only}
\bibliography{file}
\end{tcblisting}

To make a citation in the text, use the key that corresponds to the entry in your .bib file:

\begin{tcblisting}{}
While collections can be supplemented by other means \citep{williams1997edd},
the absence of an invisible collection amongst historians is noted by
\citet[p.556]{stieg1981inh}. It may be, as \citet{burchard1965hhl} points out,
that they have no assistants or are reluctant to delegate, or it may be
down to economic factors \citep{adams2009tc1, adams2014tc2, gb.pa2014,
adams2017tc3}\dots
\end{tcblisting}

Please refer to the documentation for \href{http://www.ctan.org/pkg/natbib}{\textsf{natbib}}
for the full range of commands available for in-text citations.


\section{Examples}

The examples below are shown in three parts.
The first, marked with \faBook, shows an extract from the Harvard (Bath) referencing style sheet.
The second, marked with \faCog, shows the reference as formatted by Bib\TeX.
The last shows how the reference was entered in the .bib file.

Some examples are highlighted in
\tcbox[colframe=hacked,colback=hacked!5!white,nobeforeafter,size=fbox,tcbox raise base]{orange}.
This indicates that some fields have been `abused' to achieve the right effect;
in other words, they contain information that does not conform with their intended use.
Some others make use of the \texttt{note} and \texttt{titleaddon} fields to achieve
the right effect, where other styles might need the information placed differently.
Particular care should be taken with such items when switching between different styles,
though of course any item might need adjustment to take account of differing conventions.


\subsection{Books and book chapters}

\subsubsection*{Book with author(s)}

\begin{bibexbox}<HBS 1>{rang.etal2012rdp}
  Rang, H.P., Dale, M.M., Ritter, J.M., Flower, R.J. and Henderson, G., 2012. \emph{Rang and Dale's pharmacology}. 7th ed. Edinburgh:\@ Elsevier Churchill Livingstone.
  \tcblower
\begin{Verbatim}
%</driver>
%<*driver|bib>
@book{rang.etal2012rdp,
  author = {Rang, H. P. and Dale, M. M. and Ritter, J. M. and Flower, R. J. and Henderson, G.},
  year = {2012},
  title = {Rang and {Dale's} Pharmacology},
  edition = {7},
  address = {Edinburgh},
  publisher = {Elsevier Churchill Livingstone}}
%</driver|bib>
%<*driver>
\end{Verbatim}
\end{bibexbox}

\begin{bibexbox}<HBS 1>{ou1972em}
  Open University, 1972. \emph{Electricity and magnetism}. Bletchley:\@ Open University Press.
  \tcblower
\begin{Verbatim}
%</driver>
%<*driver|bib>
@book{ou1972em,
  author = {{Open University}},
  year = {1972},
  title = {Electricity and Magnetism},
  address = {Bletchley},
  publisher = {Open University Press}}
%</driver|bib>
%<*driver>
\end{Verbatim}
\end{bibexbox}


\subsubsection*{Book with editor(s) instead of author(s)}

\begin{bibexbox}<HBS 4>{rothman.etal2008me}
  Rothman, K.J., Greenland, S. and Lash, T.L., eds., 2008. \emph{Modern epidemiology}. 3rd ed. Philadelphia, Pa.:\@ Lippincott Williams \& Wilkins.
  \tcblower
\begin{Verbatim}
%</driver>
%<*driver|bib>
@book{rothman.etal2008me,
  editor = {Kenneth J. Rothman and Sander Greenland and Timothy L. Lash},
  year = {2008},
  title = {Modern Epidemiology},
  edition = {3},
  address = {Philadelphia, Pa.},
  publisher = {Lippincott Williams \& Wilkins}}
%</driver|bib>
%<*driver>
\end{Verbatim}
\end{bibexbox}


\subsubsection*{Book usually known by title rather than author}

\begin{bibexbox}<HBS 5>{oed1989}
  \emph{Oxford English dictionary}, 1989. 2nd ed. Oxford:\@ Clarendon Press.
  \tcblower
\begin{Verbatim}
%</driver>
%<*driver|bib>
@book{oed1989,
  title = {Oxford {English} Dictionary},
  year = {1989},
  edition = {2},
  address = {Oxford},
  publisher = {Clarendon Press}}
%</driver|bib>
%<*driver>
\end{Verbatim}
\end{bibexbox}

\begin{bibexbox}<HBS 5>{merckNDidx}
  \emph{The Merck index online}, n.d.\@ [Online]. London:\@ RSC Publishing. Available from:\@ \url{http://www.rsc.org/Merck-Index} [Accessed 16 June 2016].
  \tcblower
\begin{Verbatim}
%</driver>
%<*driver|bib>
@book{merckNDidx,
  title = {The {Merck} Index Online},
  year = {n.d.},
  titleaddon = {[Online]},
  address = {London},
  publisher = {RSC Publishing},
  url = {http://www.rsc.org/Merck-Index},
  urldate = {16/06/2016}}
%</driver|bib>
%<*driver>
\end{Verbatim}
\end{bibexbox}


\subsubsection*{Electronic book}

\begin{bibexbox}<HBS 3>{haynes2014crc}
  Haynes, W.M., ed.\@, 2014. \emph{CRC handbook of chemistry and physics} [Online]. 94th ed. Boca Raton, Fla.:\@ CRC Press/Taylor and Francis. Available from:\@ \url{http://www.hbcpnetbase.com} [Accessed 16 June 2016].
  \tcblower
\begin{Verbatim}
%</driver>
%<*driver|bib>
@book{haynes2014crc,
  editor = {Haynes, W. M.},
  year = {2014},
  title = {{CRC} Handbook of Chemistry and Physics},
  titleaddon = {[Online]},
  edition = {94},
  address = {Boca Raton, Fla.},
  publisher = {CRC Press/Taylor and Francis},
  url = {http://www.hbcpnetbase.com},
  urldate = {16/06/2016}}
%</driver|bib>
%<*driver>
\end{Verbatim}
\end{bibexbox}

\begin{bibexbox}<HBS 3>{hodds2016re}
  Hodds, J., 2016. \emph{Referencing ebooks} [Kindle version 4.18]. Bath:\@ University of Bath.
  \tcblower
\begin{Verbatim}
%</driver>
%<*driver|bib>
@book{hodds2016re,
  author = {Hodds, J.},
  year = {2016},
  title = {Referencing ebooks},
  titleaddon = {[Kindle version 4.18]},
  address = {Bath},
  publisher = {University of Bath}}
%</driver|bib>
%<*driver>
\end{Verbatim}
\end{bibexbox}


\subsubsection*{One chapter\slash paper from a collection (by different authors) in an edited book}

\begin{bibexbox}<ref.\@ list>{burchard1965hhl}
  Burchard, J.E., 1965. How humanists use a library. In: C.F.J. Overhage and J.R. Harman, eds. \emph{Intrex: report on a planning conference and information transfer experiments}. Cambridge, Mass.: MIT Press, pp.41--87.
  \tcblower
\begin{Verbatim}
%</driver>
%<*driver|bib>
@incollection{burchard1965hhl,
  author = {Burchard, J. E.},
  year = {1965},
  title = {How Humanists use a Library},
  editor = {C. F. J. Overhage and J. R. Harman},
  booktitle = {Intrex: report on a planning conference and information transfer experiments},
  address = {Cambridge, Mass.},
  publisher = {MIT Press},
  pages = {41-87}}
%</driver|bib>
%<*driver>
\end{Verbatim}
\end{bibexbox}

\begin{bibexbox}<HBS 2>{reid1967ptp}
  Reid, D.R., 1967. Physical testing of polymer films. In: S.H. Pinner, ed.\@ \emph{Modern packaging films}. London:\@ Butterworths, pp.143--183.
  \tcblower
\begin{Verbatim}
%</driver>
%<*driver|bib>
@incollection{reid1967ptp,
  author = {D. R. Reid},
  year = {1967},
  title = {Physical Testing of Polymer Films},
  editor = {S. H. Pinner},
  booktitle = {Modern Packaging Films},
  address = {London},
  publisher = {Butterworths},
  pages = {143-183}}
%</driver|bib>
%<*driver>
\end{Verbatim}
\end{bibexbox}


\subsection{Articles and periodicals}

\subsubsection*{Journal article}

\begin{bibexbox}<ref.\@ list>{stieg1981cer}
  Stieg, M.F., 1981a. Continuing education and the reference librarian in the academic and research library. \emph{Library Journal}, 105(22), pp.2547--2551.
  \tcblower
\begin{Verbatim}
%</driver>
%<*driver|bib>
@article{stieg1981cer,
  author = {Stieg, M. F.},
  year = {1981},
  title = {Continuing Education and the Reference Librarian in the Academic and Research Library},
  journal = {Library Journal},
  volume = {105},
  number ={22},
  pages = {2547-2551}}
%</driver|bib>
%<*driver>
\end{Verbatim}
\end{bibexbox}

\begin{bibexbox}<ref.\@ list>{stieg1981inh}
  Stieg, M.F., 1981b. The information needs of historians. \emph{College and research libraries}, 42(6), pp.549--560.
  \tcblower
\begin{Verbatim}
%</driver>
%<*driver|bib>
@article{stieg1981inh,
  author = {Stieg, M. F.},
  year = {1981},
  title = {The Information Needs of Historians},
  journal = {College and Research Libraries},
  volume = {42},
  number ={6},
  pages = {549-560}}
%</driver|bib>
%<*driver>
\end{Verbatim}
\end{bibexbox}

\begin{bibexbox}<HBS 6>{newman2010mcb}
  Newman, R., 2010. Malaria control beyond 2010. \emph{Brit.\@ Med.\@ J.}, 341(7765), pp.157--208.
  \tcblower
\begin{Verbatim}
%</driver>
%<*driver|bib>
@article{newman2010mcb,
  author = {Newman, R.},
  year = {2010},
  title = {Malaria control beyond 2010},
  journal = {Brit.\@ Med.\@ J.},
  volume = {341},
  number = {7765},
  pages = {157-208}}
%</driver|bib>
%<*driver>
\end{Verbatim}
\end{bibexbox}


\subsubsection*{Electronic journal article}

\begin{bibexbox}(hacked)<HBS 7>{williams1997edd}
  Williams, F., 1997. Electronic document delivery:\@ a trial in an academic library. \emph{Ariadne} [Online], 10. Available from:\@ \url{http://www.ariadne.ac.uk/issue10/edd/} [Accessed 5 December 1997].
  \tcblower
\begin{Verbatim}
%</driver>
%<*driver|bib>
@article{williams1997edd,
  author = {Williams, F.},
  year = {1997},
  title = {Electronic Document Delivery: a Trial in an Academic Library},
  journal = {Ariadne \textup{[Online]}},
  volume = {10},
  url = {http://www.ariadne.ac.uk/issue10/edd/},
  urldate = {05/12/1997}}
%</driver|bib>
%<*driver>
\end{Verbatim}
\end{bibexbox}


\subsubsection*{Preprint in a digital repository}

\begin{bibexbox}<HBS 8>{shah.corrick2016hsc}
  Shah, I. and Corrick, I. 2016. \emph{How should central banks respond to non-neutral inflation expectations?} Bath:\@ University of Bath. \emph{OPUS} [Online]. Available from:\@ \url{http://opus.bath.ac.uk} [Accessed 4 May 2016].
  \tcblower
\begin{Verbatim}
%</driver>
%<*driver|bib>
@techreport{shah.corrick2016hsc,
  author = {Shah, I. and Corrick, I.},
  year = {2016},
  title = {How should central banks respond to non-neutral inflation expectations?},
  address = {Bath},
  institution = {University of Bath},
  note = {\emph{OPUS} [Online]},
  url = {http://opus.bath.ac.uk},
  urldate = {04/05/2016}}
%</driver|bib>
%<*driver>
\end{Verbatim}
\end{bibexbox}

The Harvard (Bath) style treats preprints as grey literature, and so the
\texttt{techreport} entry type is the best match semantically.


\subsubsection*{Newspaper article}

\begin{bibexbox}(hacked)<HBS 12>{haurant2004bbh}
  Haurant, S., 2004. Britain's borrowing hits £1 trillion. \emph{The Guardian}, 29 July, p.16c.
  \tcblower
\begin{Verbatim}
%</driver>
%<*driver|bib>
@article{haurant2004bbh,
  author = {Haurant, S.},
  year = {2004},
  title = {Britain's Borrowing Hits \pounds 1 Trillion},
  journal = {The Guardian},
  volume = {29 July},
  pages = {16c}}
%</driver|bib>
%<*driver>
\end{Verbatim}
\end{bibexbox}

\begin{bibexbox}(hacked)<HBS 12>{independent1992pub}
  The Independent, 1992. Picking up the bills. \emph{The Independent}, 4 June, p.28a.
  \tcblower
\begin{Verbatim}
%</driver>
%<*driver|bib>
@article{independent1992pub,
  author = {{The Independent}},
  year = {1992},
  title = {Picking Up the Bills},
  journal = {The Independent},
  volume = {4 June},
  pages = {28a}}
%</driver|bib>
%<*driver>
\end{Verbatim}
\end{bibexbox}


\subsection{Conference papers}

\subsubsection*{Conference paper (when proceedings have a named editor)}

\begin{bibexbox}<HBS 10>{crawford1965oim}
  Crawford, G.I., 1965. Oxygen in metals. In:\@ J.M.A. Lenihan and S.J. Thompson, eds. \emph{Activation analysis:\@ proceedings of a NATO Advanced Study Institute}, 2--4 August 1964 Glasgow. London:\@ Academic Press, pp.113--118.
  \tcblower
\begin{Verbatim}
%</driver>
%<*driver|bib>
@inproceedings{crawford1965oim,
  author = {Crawford, G. I.},
  year = {1965},
  title = {Oxygen in Metals},
  editor = {J. M. A. Lenihan and S. J. Thompson},
  booktitle = {Activation Analysis: Proceedings of a {NATO} {Advanced} {Study} {Institute}},
  venue = {2--4 August 1964 Glasgow},
  address = {London},
  publisher = {Academic Press},
  pages = {113-118}}
%</driver|bib>
%<*driver>
\end{Verbatim}
\end{bibexbox}


\subsubsection*{Conference paper (when proceedings have no named editor or are part of a major series)}

\begin{bibexbox}<HBS 11>{soper1972rbc}
  Soper, D., 1972. Review of bracken control experiments with asulam. \emph{Proceedings of the 11th British Weed Control Conference}, 15--17 November 1972 Brighton. Brighton:\@ University of Sussex, pp.24--31.
  \tcblower
\begin{Verbatim}
%</driver>
%<*driver|bib>
@inproceedings{soper1972rbc,
  author = {Soper, D.},
  year = {1972},
  title = {Review of Bracken Control Experiments with Asulam},
  booktitle = {Proceedings of the 11th {British} {Weed} {Control} {Conference}},
  venue = {15--17 November 1972 Brighton},
  address = {Brighton},
  publisher = {University of Sussex},
  pages = {24-31}}
%</driver|bib>
%<*driver>
\end{Verbatim}
\end{bibexbox}


\subsection{Grey literature}

\subsubsection*{Thesis/dissertation}

\begin{bibexbox}<HBS 13>{burrell1973ist}
  Burrell, J.G., 1973. \emph{The importance of school tours in education}. Thesis (M.A.). Queen's University, Belfast.
  \tcblower
\begin{Verbatim}
%</driver>
%<*driver|bib>
@mastersthesis{burrell1973ist,
  author = {Burrell, J. G.},
  year = {1973},
  title = {The Importance of School Tours in Education},
  type = {Thesis ({M.A.})},
  school = {Queen's University, Belfast}}
%</driver|bib>
%<*driver>
\end{Verbatim}
\end{bibexbox}


\subsubsection*{Report}

\begin{bibexbox}<HBS 14>{unesco1993gip}
  UNESCO, 1993. \emph{General information programme and UNISIST}\@. Paris:\@ UNESCO, (PGI-93/WS/22).
  \tcblower
\begin{Verbatim}
%</driver>
%<*driver|bib>
@techreport{unesco1993gip,
  author = {{UNESCO}},
  year = {1993},
  title = {General Information Programme and {UNISIST}},
  address = {Paris},
  institution = {UNESCO},
  number = {PGI-93/WS/22}}
%</driver|bib>
%<*driver>
\end{Verbatim}
\end{bibexbox}


\subsubsection*{Standard}

\begin{bibexbox}(hacked)<HBS 15>{bs5605:1990}
  BS 5605:1990. \emph{Recommendations for citing and referencing published material}. BSI.
  \tcblower
\begin{Verbatim}
%</driver>
%<*driver|bib>
@manual{bs5605:1990,
  author = {{BS 5605:1990}},
  title = {Recommendations for citing and referencing published material},
  organization = {BSI}}
%</driver|bib>
%<*driver>
\end{Verbatim}
\end{bibexbox}

You can also use \texttt{standard} as an alias for \texttt{manual}.


\subsubsection*{Patent}

\begin{bibexbox}<HBS 16>{pm1981opa}
  Phillipp Morris Inc., 1981. \emph{Optical perforating apparatus and system}. European patent application 0021165A1. 1981-01-07.
  \tcblower
\begin{Verbatim}
%</driver>
%<*driver|bib>
@manual{pm1981opa,
  author = {{Phillipp Morris Inc.}},
  year = {1981},
  title = {Optical perforating apparatus and system},
  series = {European patent application},
  number = {0021165A1. 1981-01-07}}
%</driver|bib>
%<*driver>
\end{Verbatim}
\end{bibexbox}

You can also use \texttt{patent} as an alias for \texttt{manual}.


\subsection{Audiovisual materials}

\subsubsection*{Image}

These examples are taken from the Library guide
\href{http://www.bath.ac.uk/library/pass.bho/images-referencing.pdf}{\emph{Referencing images}}.

\begin{bibexbox}<RI>{nasa2015nat}
   NASA, 2015, \emph{NASA Astronaut Tim Kopra on Dec.\@ 21 Spacewalk}. Washington: NASA. Available from: \url{http://www.nasa.gov/image-feature/nasa-astronaut-tim-kopra-on-dec-21-spacewalk} [Accessed 7 January 2015].
  \tcblower
\begin{Verbatim}
%</driver>
%<*driver|bib>
@online{nasa2015nat,
  author = {{NASA}},
  year = {2015},
  title = {{NASA} Astronaut {Tim} {Kopra} on {Dec.\@} 21 Spacewalk},
  address = {Washington},
  organization = {NASA},
  url = {http://www.nasa.gov/image-feature/nasa-astronaut-tim-kopra-on-dec-21-spacewalk},
  urldate = {2015-01-07}}
%</driver|bib>
%<*driver>
\end{Verbatim}
\end{bibexbox}

\begin{bibexbox}<RI>{iliff2006rcb}
   Iliff, D., 2006. \emph{Royal Crescent in Bath, England - July 2006}, San Francisco: Wikimedia Foundation. Available from: \url{https://commons.wikimedia.org/wiki/File:Royal_Crescent_in_Bath,_England_-_July_2006.jpg} [Accessed 7 January 2016].
  \tcblower
\begin{Verbatim}
%</driver>
%<*driver|bib>
@online{iliff2006rcb,
  author = {D. Iliff},
  year = {2006},
  title = {{Royal} {Crescent} in {Bath,} {England} - {July} 2006},
  address = {San Francisco},
  organization = {Wikimedia Foundation},
  url = {https://commons.wikimedia.org/wiki/File:Royal_Crescent_in_Bath,_England_-_July_2006.jpg},
  urldate = {2015-01-07}}
%</driver|bib>
%<*driver>
\end{Verbatim}
\end{bibexbox}

\begin{bibexbox}<RI>{anon1946peb}
  Anon., 1946. \emph{Prototype electric bicycle displayed at the \enquote{Britain Can Make It} exhibition organised by the Council of Industrial Design and held at the Victoria and Albert Museum, London, 1946. Designed by B.~G. Bowden} [photograph]. At: London. Design Council Slide Collection.
  \tcblower
\begin{Verbatim}
%</driver>
%<*driver|bib>
@unpublished{anon1946peb,
  author = {Anon.},
  year = {1946},
  title = {Prototype electric bicycle displayed at the \enquote{Britain Can Make It} exhibition
    organised by the {Council} of {Industrial} {Design} and held at the {Victoria} and {Albert}
    {Museum}, {London}, 1946. {Designed} by {B.~G.} {Bowden}},
  titleaddon = {[photograph]},
  note = {At: London. Design Council Slide Collection}}
%</driver|bib>
%<*driver>
\end{Verbatim}
\end{bibexbox}


\subsubsection*{Map}

\begin{bibexbox}<HBS 18>{andrews.dury1773wilts}
  Andrews, J. and Dury, A., 1773. \emph{Map of Wiltshire}, 1 inch to 2 miles. Devizes:\@ Wiltshire Record Society.
  \tcblower
\begin{Verbatim}
%</driver>
%<*driver|bib>
@manual{andrews.dury1773wilts,
  author = {Andrews, J. and Dury, A.},
  year = {1773},
  title = {Map of {Wiltshire}},
  series = {1 inch to 2 miles},
  address = {Devizes},
  publisher = {Wiltshire Record Society}}
%</driver|bib>
%<*driver>
\end{Verbatim}
\end{bibexbox}

The \texttt{book} entry type would also work for this reference.


\subsubsection*{Film, video or DVD}

\begin{bibexbox}<HBS 19>{macbeth1948}
  \emph{Macbeth}, 1948. Film.\@ Directed by Orson Welles. USA:\@ Republic Pictures.
  \tcblower
\begin{Verbatim}
%</driver>
%<*driver|bib>
@booklet{macbeth1948,
  year = {1948},
  title = {Macbeth},
  howpublished = {Film. Directed by Orson Welles},
  address = {USA},
  publisher = {Republic Pictures}}
%</driver|bib>
%<*driver>
\end{Verbatim}
\end{bibexbox}

You can also use \texttt{movie} or \texttt{video} as an alias for \texttt{booklet}.


\subsubsection*{Streamed video (YouTube, TED Talks, etc.)}

\begin{bibexbox}<HBS 20>{moran2016sol}
  Moran, C., 2016. \emph{Save Our Libraries} [Online]. Available from:\@ \url{https://www.youtube.com/watch?v=gKTfCz4JtVE&feature=youtu.be} [Accessed 29 April 2016]
  \tcblower
\begin{Verbatim}
%</driver>
%<*driver|bib>
@online{moran2016sol,
  author = {Moran, C.},
  year = {2016},
  title = {Save Our Libraries},
  titleaddon = {[Online]},
  url = {https://www.youtube.com/watch?v=gKTfCz4JtVE&feature=youtu.be},
  urldate = {29/04/2016}}
%</driver|bib>
%<*driver>
\end{Verbatim}
\end{bibexbox}

\begin{bibexbox}<RI>{uob2015wie}
   University of Bath, 2015. \emph{What is engineering?} Available from: \url{https://www.youtube.com/watch?v=NoyZarq-Zbo} [Accessed 12 January 2016].
  \tcblower
\begin{Verbatim}
%</driver>
%<*driver|bib>
@online{uob2015wie,
  author = {{University of Bath}},
  year = {2015},
  title = {What is Engineering?},
  url = {https://www.youtube.com/watch?v=NoyZarq-Zbo},
  urldate = {2016-01-12}}
%</driver|bib>
%<*driver>
\end{Verbatim}
\end{bibexbox}


\subsubsection*{Television or radio broadcast}

\begin{bibexbox}<HBS 21>{rsfo2006ep5}
  \emph{Rick Stein's French Odyssey: Episode 5}, 2006. TV. BBC2, 23 August. 20.30 hrs.
  \tcblower
\begin{Verbatim}
%</driver>
%<*driver|bib>
@booklet{rsfo2006ep5,
  year = {2006},
  title = {Rick {Stein's} {French} {Odyssey}: Episode 5},
  howpublished = {TV. BBC2, 23 August. 20.30 hrs}}
%</driver|bib>
%<*driver>
\end{Verbatim}
\end{bibexbox}

\begin{bibexbox}<HBS 21>{archers20060823}
  \emph{The Archers}, 2006. Radio.\@ BBC Radio 4, 23 August. 19.02 hrs.
  \tcblower
\begin{Verbatim}
%</driver>
%<*driver|bib>
@booklet{archers20060823,
  year = {2006},
  title = {The {Archers}},
  howpublished = {Radio. BBC Radio 4, 23 August. 19.02 hrs}}
%</driver|bib>
%<*driver>
\end{Verbatim}
\end{bibexbox}

You can also use \texttt{audio} or \texttt{music} as an alias for \texttt{booklet}.


\subsubsection*{Music score}

\begin{bibexbox}<HBS 22>{beethoven1950symph1}
  Beethoven, L. van, 1950. \emph{Symphony no.1 in C, Op.21}. Harmondsworth:\@ Penguin.
  \tcblower
\begin{Verbatim}
%</driver>
%<*driver|bib>
@book{beethoven1950symph1,
  author = {Ludwig van Beethoven},
  year = {1950},
  title = {Symphony no.1 in {C,} {Op.21}},
  address = {Harmondsworth},
  publisher = {Penguin}}
%</driver|bib>
%<*driver>
\end{Verbatim}
\end{bibexbox}


\subsection{Digital media}

\subsubsection*{Website\slash webpage}

While it is normal when using Bib\TeX\ to use \texttt{misc} for websites,
with this style you should use \texttt{electronic}, \texttt{online} or \texttt{www} instead.
Internally these are all aliases for \texttt{manual}.

\begin{bibexbox}<HBS 9>{holland2002gci}
  Holland, M., 2002. \emph{Guide to citing internet sources} [Online]. Poole:\@ Bournemouth University. Available from:\@ \url{http://www.bournemouth.ac.uk/library/using/guide_to_citing_internet_sourc.html} [Accessed 4 November 2002].
  \tcblower
\begin{Verbatim}
%</driver>
%<*driver|bib>
@online{holland2002gci,
  author = {Holland, M.},
  year = {2002},
  title = {Guide to Citing Internet Sources},
  titleaddon = {[Online]},
  address = {Poole},
  organization = {Bournemouth University},
  url = {http://www.bournemouth.ac.uk/library/using/guide_to_citing_internet_sourc.html},
  urldate = {04/11/2002}}
%</driver|bib>
%<*driver>
\end{Verbatim}
\end{bibexbox}

\begin{bibexbox}<HBS 9>{wiltshire2015gww}
  Wiltshire Council, 2015. \emph{Get Wiltshire walking} [Online]. Trowbridge:\@ Wiltshire Council. Available from:\@ \url{http://www.wiltshire.gov.uk/leisureandrecreation/sportphysicalactivity/getwiltshirewalking.html} [Accessed 19 August 2015].
  \tcblower
\begin{Verbatim}
%</driver>
%<*driver|bib>
@online{wiltshire2015gww,
  author = {{Wiltshire Council}},
  year = {2015},
  title = {Get {Wiltshire} walking},
  titleaddon = {[Online]},
  address = {Trowbridge},
  organization = {Wiltshire Council},
  url = {http://www.wiltshire.gov.uk/leisureandrecreation/sportphysicalactivity/getwiltshirewalking.html},
  urldate = {19/08/2015}}
%</driver|bib>
%<*driver>
\end{Verbatim}
\end{bibexbox}


\subsubsection*{Email discussion lists (jiscmail\slash listserv etc.)}

\begin{bibexbox}(hacked)<HBS 23>{clark2004euk}
  Clark, T., 5 July 2004. A European UK Libraries Plus? \emph{Lis-link} [Online]. Available from:\@ \url{lis-link@jiscmail.ac.uk} [Accessed 30 July 2004].
  \tcblower
\begin{Verbatim}
%</driver>
%<*driver|bib>
@article{clark2004euk,
  author = {Clark, T.},
  year = {05/07/2004},
  title = {A {European} {UK} {Libraries} {Plus}?},
  journal = {Lis-link \textup{[Online]}},
  url = {lis-link@jiscmail.ac.uk},
  urldate = {30/07/2004}}
%</driver|bib>
%<*driver>
\end{Verbatim}
\end{bibexbox}


\subsubsection*{Personal emails}

\begin{bibexbox}(hacked)<HBS 24>{alston2004sah}
  Alston, S., 19 July 2004. \emph{Society of Architectural Historians of GB}. Email to K.M. Jordan.
  \tcblower
\begin{Verbatim}
%</driver>
%<*driver|bib>
@unpublished{alston2004sah,
  author = {Alston, S.},
  year = {19/07/2004},
  title = {Society of {Architectural} {Historians} of {GB}},
  note = {Email to K.M. Jordan}}
%</driver|bib>
%<*driver>
\end{Verbatim}
\end{bibexbox}


\subsubsection*{Database}

\begin{bibexbox}<HBS 40>{bvd2008bt}
  Bureau van Dijk, 2008. \emph{BT Group plc company report}. \emph{FAME} [Online]. London:\@ Bureau van Dijk. Available from:\@ \url{http://www.portal.euromonitor.com} [Accessed 6 November 2014].
  \tcblower
\begin{Verbatim}
%</driver>
%<*driver|bib>
@online{bvd2008bt,
  author = {{Bureau van Dijk}},
  year = {2008},
  title = {{BT} {Group} PLC Company Report},
  series = {\emph{FAME} [Online]},
  address = {London},
  organization = {Bureau van Dijk},
  url = {http://www.portal.euromonitor.com},
  urldate = {06/11/2014}}
%</driver|bib>
%<*driver>
\end{Verbatim}
\end{bibexbox}


\subsubsection*{Dataset}

\begin{bibexbox}<HBS 41>{wilson2013rgc}
  Wilson, D., 2013. \emph{Real geometry and connectedness via triangular description:\@ CAD example bank} [Online]. Bath:\@ University of Bath. Available from:\@ \url{http://doi.org/10.15125/BATH-00069} [Accessed 20 April 2016].
  \tcblower
\begin{Verbatim}
%</driver>
%<*driver|bib>
@online{wilson2013rgc,
  author = {Wilson, D.},
  year = {2013},
  title = {Real Geometry and Connectedness via Triangular Description: {CAD} Example Bank},
  titleaddon = {[Online]},
  address = {Bath},
  organization = {University of Bath},
  doi = {10.15125/BATH-00069},
  urldate = {20/04/2016}}
%</driver|bib>
%<*driver>
\end{Verbatim}
\end{bibexbox}

You can also use \texttt{dataset} as an alternative alias for \texttt{manual}.


\subsubsection*{Computer program}

\begin{bibexbox}<HBS 42>{screencasto}
  @screencasto, n.d.\@ \emph{Screencast-O-Matic} (v.2) [computer program]. Available from: \url{https://screencast-o-matic.com/} [Accessed 16 May 2016].
  \tcblower
\begin{Verbatim}
%</driver>
%<*driver|bib>
@online{screencasto,
  author = {@screencasto},
  year = {n.d.},
  title = {{Screencast-O-Matic}},
  titleaddon = {(v.2) [computer program]},
  url = {https://screencast-o-matic.com/},
  urldate = {16/05/2016}}
%</driver|bib>
%<*driver>
\end{Verbatim}
\end{bibexbox}

You can also use \texttt{software} as an alternative alias for \texttt{manual}.


\subsection{Works in languages other than English}

\subsubsection*{Work in translation}

\begin{bibexbox}<HBS 26>{aristotle2007ne}
  Aristotle, 2007. \emph{Nicomachean ethics} (W.D. Ross. Trans.). South Dakota:\@ NuVisions.
  \tcblower
\begin{Verbatim}
%</driver>
%<*driver|bib>
@book{aristotle2007ne,
  author = {Aristotle},
  year = {2007},
  title = {Nicomachean Ethics},
  translator = {W. D. Ross},
  address = {South Dakota},
  publisher = {NuVisions}}
%</driver|bib>
%<*driver>
\end{Verbatim}
\end{bibexbox}


\subsubsection*{Work in the Roman alphabet}

\begin{bibexbox}<HBS 27>{esquivel2003cap}
  Esquivel, L., 2003. \emph{Como agua para chocolate} [Like water for chocolate]. Barcelona:\@ Debolsillo.
  \tcblower
\begin{Verbatim}
%</driver>
%<*driver|bib>
@book{esquivel2003cap,
  author = {Esquivel, L.},
  year = {2003},
  title = {Como Agua para Chocolate},
  titleaddon = {[Like water for chocolate]},
  address = {Barcelona},
  publisher = {Debolsillo}}
%</driver|bib>
%<*driver>
\end{Verbatim}
\end{bibexbox}

\begin{bibexbox}<HBS 27>{thurfjell1975vhv}
  Thurfjell, W., 1975. Vart har våran doktor tagit vägen? [Where has our doctor gone?] \emph{Läkartidningen} 72, p.789.
  \tcblower
\begin{Verbatim}
%</driver>
%<*driver|bib>
@article{thurfjell1975vhv,
  author = {Thurfjell, W.},
  year = {1975},
  title = {Vart har våran doktor tagit vägen?},
  titleaddon = {[Where has our doctor gone?]},
  journal = {Läkartidningen},
  volume = {72},
  pages = {789}}
%</driver|bib>
%<*driver>
\end{Verbatim}
\end{bibexbox}


\subsubsection*{Work in a non-Roman alphabet}

The following example tricks Bib\TeX\ into treating the original rendering of the author's name as the `von' part of a Roman-alphabet name. This requires the use of a command that simply gobbles its argument, which you have to define yourself:

\begin{tcblisting}{listing only}
\newcommand*{\noop}[1]{}
\end{tcblisting}

For the trick to work, the argument you give to \lstinline[style=dtxlatex]|\noop| must be lowercase, but otherwise it can be anything you like.

\newcommand*{\noop}[1]{}
\begin{bibexbox}(hacked){hua1999qys1}
  Hua, L. 華林甫, 1999.  Qingdai yilai Sanxia diqu shuihan zaihai de chubu yanjiu 清代以來三峽地區水旱災害的初步硏 [A preliminary study of floods and droughts in the Three Gorges region since the Qing dynasty], \emph{Zhongguo shehui kexue} 中國社會科學 , 1, pp.168--79.
  \tcblower
\begin{Verbatim}
%</driver>
%<*driver|bib>
@article{hua1999qys1,
  author = {Linfu \noop{h}華林甫 Hua},
  year = {1999},
  title = {Qingdai yilai {Sanxia} diqu shuihan zaihai de chubu yanjiu
    {清代以來三峽地區水旱災害的初步硏}},
  titleaddon = {[A preliminary study of floods and droughts in the {Three} {Gorges} region since
    the {Qing} dynasty]},
  journal = {Zhongguo shehui kexue \textup{中國社會科學}},
  volume = {1},
  pages = {168-79}}
%</driver|bib>
%<*driver>
\end{Verbatim}
\end{bibexbox}

If the name is due to appear initial first (e.g.\ after `In:'), you can append the non-Roman characters to the author's surname; to do this, use inverted name order as you would for English double-barrelled names without hyphens, e.g.\ \texttt{Hua 華林甫, Linfu}.

\begin{bibexbox}{hua1999qys2}
  Hua, L., 1999. Qingdai yilai Sanxia diqu shuihan zaihai de chubu yanjiu [A preliminary study of floods and droughts in the Three Gorges region since the Qing dynasty], \emph{Zhongguo shehui kexue}, 1, pp.168--79.
  \tcblower
\begin{Verbatim}
%</driver>
%<*driver|bib>
@article{hua1999qys2,
  author = {Hua, Linfu},
  year = {1999},
  title = {Qingdai yilai {Sanxia} diqu shuihan zaihai de chubu yanjiu},
  titleaddon = {[A preliminary study of floods and droughts in the {Three} {Gorges} region since
    the {Qing} dynasty]},
  journal = {Zhongguo shehui kexue},
  volume = {1},
  pages = {168-79}}
%</driver|bib>
%<*driver>
\end{Verbatim}
\end{bibexbox}


\subsection{Legal references: UK legislation and parliamentary reports}

% UK Primary Legislation

\subsubsection*{Act of Parliament (UK Statutes) before 1963}

\begin{bibexbox}<HBS 32>{gb.wa1735}
  \emph{Witchcraft Act 1735} (9 Geo.2, c.5).
  \tcblower
\begin{Verbatim}
%</driver>
%<*driver|bib>
@techreport{gb.wa1735,
  key = {\emph{Witchcraft Act 1735}},
  title = {Witchcraft {Act} 1735},
  series = {9 Geo.2},
  number = {c.5}}
%</driver|bib>
%<*driver>
\end{Verbatim}
\end{bibexbox}


\subsubsection*{Act of Parliament (UK Statutes) 1963 onwards}

\begin{bibexbox}<HBS 33>{gb.pa2014}
  \emph{Pensions Act 2014}, c.19. London:\@ TSO.
  \tcblower
\begin{Verbatim}
%</driver>
%<*driver|bib>
@book{gb.pa2014,
  key = {\emph{Pensions Act 2014}},
  title = {Pensions {Act} 2014},
  number = {c.19},
  address = {London},
  publisher = {TSO}}
%</driver|bib>
%<*driver>
\end{Verbatim}
\end{bibexbox}


\subsubsection*{House of Commons/House of Lords bill}

\begin{bibexbox}<HBS 31>{gb.bill1987/88-66}
  Great Britain. Parliament. House of Commons, 1988. \emph{Local government finance bill}. London:\@ HMSO (Bills | 1987/88, 66).
  \tcblower
\begin{Verbatim}
%</driver>
%<*driver|bib>
@techreport{gb.bill1987/88-66,
  author = {{Great Britain. Parliament. House of Commons}},
  year = {1988},
  title = {Local Government Finance Bill},
  address = {London},
  publisher = {HMSO},
  series = {{Bills | 1987/88}},
  number = {66}}
%</driver|bib>
%<*driver>
\end{Verbatim}
\end{bibexbox}


% UK secondary legislation

\subsubsection*{Statutory instrument}

\begin{bibexbox}<HBS 35>{gb.hmr2012}
  \emph{The Human Medicines Regulations 2012} [Online], No.1916, United Kingdom:\@ HMSO. Available from:\@ \url{http://www.legislation.gov.uk/uksi/2012/1916/pdfs/uksi_20121916_en.pdf} [Accessed 17 April 2016].
  \tcblower
\begin{Verbatim}
%</driver>
%<*driver|bib>
@book{gb.hmr2012,
  key = {\emph{The {Human} {Medicines} {Regulations} 2012}},
  title = {The {Human} {Medicines} {Regulations} 2012},
  titleaddon = {[Online]},
  number = {No.1916},
  address = {United Kingdom},
  publisher = {HMSO},
  url = {http://www.legislation.gov.uk/uksi/2012/1916/pdfs/uksi_20121916_en.pdf},
  urldate = {17/04/2016}}
%</driver|bib>
%<*driver>
\end{Verbatim}
\end{bibexbox}


% Parliamentary reports

\subsubsection*{House of Commons paper}

Use this form for reports of House of Commons select committees.

\begin{bibexbox}<HBS 29>{gb.hc2003/04-30}
  Great Britain. Parliament. House of Commons, 2004. \emph{National Savings investment deposits:\@ account 2002--2003}. London:\@ National Audit Office (HC 2003/04, 30).
  \tcblower
\begin{Verbatim}
%</driver>
%<*driver|bib>
@techreport{gb.hc2003/04-30,
  author = {{Great Britain. Parliament. House of Commons}},
  year = {2004},
  title = {National {Savings} Investment Deposits: account 2002--2003},
  address = {London},
  publisher = {National Audit Office},
  type = {},
  series = {HC 2003/04},
  number = {30}}
%</driver|bib>
%<*driver>
\end{Verbatim}
\end{bibexbox}


\subsubsection*{House of Lords paper}

Use this form for reports of House of Lords select committees.

\begin{bibexbox}<HBS 30>{gb.hl1986/87-66}
  Great Britain. Parliament. House of Lords, 1987. \emph{Social fund (Maternity and Funeral Expenses) Bill}. London:\@ HMSO (HL 1986/87, (66)).
  \tcblower
\begin{Verbatim}
%</driver>
%<*driver|bib>
@techreport{gb.hl1986/87-66,
  author = {{Great Britain. Parliament. House of Lords}},
  year = {1987},
  title = {Social Fund ({Maternity} and {Funeral} {Expenses}) Bill},
  address = {London},
  publisher = {HMSO},
  series = {HL 1986/87},
  number = {66}}
%</driver|bib>
%<*driver>
\end{Verbatim}
\end{bibexbox}

For joint committees, you will have to hack this slightly, putting the session
years in \texttt{series} and the HL and HC numbers in \texttt{number}.


\subsubsection*{Command paper}

\begin{bibexbox}<HBS 34>{gb.cm6041}
  Great Britain. Ministry of Defence, 2004. \emph{Delivering security in a changing world:\@ defence white paper}. London:\@ TSO (Cm.\@ 6041).
  \tcblower
\begin{Verbatim}
%</driver>
%<*driver|bib>
@techreport{gb.cm6041,
  author = {{Great Britain. Ministry of Defence}},
  year = {2004},
  title = {Delivering Security in a Changing World: defence White Paper},
  address = {London},
  publisher = {TSO},
  type = {{Cm.}},
  number = {6041}}
%</driver|bib>
%<*driver>
\end{Verbatim}
\end{bibexbox}


\subsection{Legal references: EU legislation and reports}

% European legislation

\subsubsection*{EU regulation or directive, decision, recommendation or opinion}

\begin{bibexbox}<HBS 38>{eu.dir2015/413}
  Directive (EU) 2015/413 of the European Parliament and of the Council of 11th March 2015 facilitating cross-border exchange of information on road-safety-related traffic offences [2015] \emph{OJ} L68/9.
  \tcblower
\begin{Verbatim}
%</driver>
%<*driver|bib>
@misc{eu.dir2015/413,
  title = {Directive ({EU}) 2015/413 of the {European} {Parliament} and of the {Council} of
    11th {March} 2015 Facilitating Cross-Border Exchange of Information on Road-Safety-Related
    Traffic Offences},
  titleaddon = {[2015] \emph{OJ} L68/9}}
%</driver|bib>
%<*driver>
\end{Verbatim}
\end{bibexbox}


% European reports

\subsubsection*{EU publication}

\begin{bibexbox}<HBS 37>{ec2015gra}
  European Commission, 2015. \emph{General report on the activities of the European Union 2014}. Luxembourg:\@ Publications Office of the European Union.
  \tcblower
\begin{Verbatim}
%</driver>
%<*driver|bib>
@techreport{ec2015gra,
  author = {{European Commission}},
  year = {2015},
  title = {General Report on the Activities of the {European} {Union} 2014},
  address = {Luxembourg},
  publisher = {Publications Office of the European Union}}
%</driver|bib>
%<*driver>
\end{Verbatim}
\end{bibexbox}


\subsection{Legal references: case reports}

\subsubsection*{Legal case study}

\begin{bibexbox}<HBS 36>{seldon-v-c.w.j2012}
  \emph{Seldon v Clarkson Wright \& Jakes}. [2012]. UKSC 16.
  \tcblower
\begin{Verbatim}
%</driver>
%<*driver|bib>
@unpublished{seldon-v-c.w.j2012,
  key = {\emph{Seldon v Clarkson Wright \& Jakes}, {[2012]}},
  title = {Seldon v {Clarkson} {Wright} \& {Jakes}},
  note = {[2012]. UKSC 16}}
%</driver|bib>
%<*driver>
\end{Verbatim}
\end{bibexbox}

Generally speaking, the year should be in square brackets if it is essential to the citation
(unless it is a Scottish case, in which case it is printed bare), and in parentheses if it is
not. Since it is hard to automate this in Bib\TeX\, you should format the reference
more-or-less by hand.


\subsubsection*{Judgment of the European Court of Justice}

\begin{bibexbox}<HBS 39>{srl.etal-v-comm2005}
  \emph{Alessandrini Srl and others v.~Commission} (C-295/03 P) [2005] ECR I-5700.
  \tcblower
\begin{Verbatim}
%</driver>
%<*driver|bib>
@unpublished{srl.etal-v-comm2005,
  title = {Alessandrini {Srl} and others v.~{Commission}},
  titleaddon = {(C-295/03 P) [2005] ECR I-5700}}
%</driver|bib>
%<*driver>
\end{Verbatim}
\end{bibexbox}


\bibliography{bath-bst}

\section*{Licence}

\begin{markdown*}{hybrid=true}
%</driver>
%<readme>
%<readme>## Licence
%<readme>
%<*driver|readme>
Copyright 2016 University of Bath.

This work consists of the documented LaTeX file bath-bst.dtx and a Makefile.

The text files contained in this work may be distributed and/or modified
under the conditions of the [LaTeX Project Public License (LPPL)][lppl],
either version 1.3c of this license or (at your option) any later
version.

This work is ‘maintained’ (as per LPPL maintenance status) by [Alex Ball][me].

[lppl]: http://www.latex-project.org/lppl.txt "LaTeX Project Public License (LPPL)"
[me]: https://github.bath.ac.uk/ab318/bathbib "Alex Ball"
%</driver|readme>
%<*driver>
\end{markdown*}

\newpage
\lstset
  { style=dtxbst
  , aboveskip=0pt
  , belowskip=0pt
  , numbers=left
  , numberstyle=\color{gray}\footnotesize\itshape
  , firstnumber=last
  , basicstyle=\ttfamily\footnotesize
  , breaklines=true
  }%
\tcbset
  { listing options =
    { style = tcblatex
    , style = dtxbst
    , basicstyle=\ttfamily\footnotesize
    }
  }%
\MakeShortVerb{\|}%
\DocInput{\jobname.dtx}

\end{document}
%</driver>
%<*bib>
@book{adams2009tc1,
  author = {Adams, Gomez},
  year = {2009},
  title = {Test Citation One},
  address = {London},
  publisher = {Imperial College Bookstall}}
@book{adams2014tc2,
  author = {Adams, Gomez},
  year = {2014},
  title = {Test Citation Two},
  address = {Oxford},
  publisher = {Oxford University Press}}
@book{adams2017tc3,
  author = {Adams, Gomez},
  year = {2017},
  title = {Test Citation Three},
  address = {Cambridge},
  publisher = {Cambridge University Press}}
%</bib>
%<*bst>
% \fi
%
% \section{Implementation}
%
% \setcounter{lstnumber}{16}
%
% \subsection{Data model}
%
% We begin by listing the entry fields that we will recognize. These are the
% standard Bib\TeX\ ones plus |archive|, |doi|, |eid|, |eprint|, |language|,
% |titleaddon|, |translator|, |url|, |urldate|, and |venue|, all of which have
% been backported from \textsf{natbib} and \textsf{biblatex}.
%
%    \begin{macrocode}
ENTRY
  { address
    archive
    author
    booktitle
    chapter
    doi
    edition
    editor
    eid
    eprint
    howpublished
    institution
    journal
    key
    language
    month
    note
    number
    organization
    pages
    publisher
    school
    series
    title
    titleaddon
    translator
    type
    url
    urldate
    venue
    volume
    year
  }
  {}
  { label extra.label sort.label short.list }
%    \end{macrocode}
%
% \subsection{Writing references out to a file}
%
% We define a set of integers for saving an output state. We will use them to
% to implement the mechanism for stringing the elements of the reference
% together. Whenever an element is printed, it is separated from any previous
% previous text by a delimiter, and the delimiter chosen depends on the state.
% We define five states which trigger increasingly strong delimiters.
%
%    \begin{macrocode}
INTEGERS { output.state before.all mid.clause mid.sentence after.sentence after.block }
FUNCTION {init.state.consts}
{ #0 'before.all :=
  #1 'mid.clause :=
  #2 'mid.sentence :=
  #3 'after.sentence :=
  #4 'after.block :=
}
%    \end{macrocode}
%
% We also define two variables to store strings in.
%
%    \begin{macrocode}
STRINGS { s t }
%    \end{macrocode}
%
% The |output.nonnull| function removes an element from the stack, writes the
% appropriate delimiter to the .bbl file depending on the state, then restores
% the original element. Afterwards, the output state is set to |mid.sentence|.
%
% \begin{itemize}
% \item |before.all| = nothing is written.
% \item |mid.clause| = a space is written.
% \item |mid.sentence| = a comma is written.
% \item |after.sentence| = a period is written.
% \item |after.block| = a period, line break and |\newblock| are written.
% \end{itemize}
%
%    \begin{macrocode}
FUNCTION {output.nonnull}
{ 's :=
  output.state mid.sentence =
    { ", " * write$ }
    { output.state after.sentence =
        { add.period$ " " * write$ }
        { output.state before.all =
            'write$
            { output.state mid.clause =
                { " " * write$ }
                { add.period$ write$
                  newline$
                  "\newblock " write$
                }
              if$
            }
          if$
        }
      if$
      mid.sentence 'output.state :=
    }
  if$
  s
}
%    \end{macrocode}
%
% The `after' output states persist if nothing is printed. The `mid.clause'
% state shouldn't, so we provide a macro for reverting to `mid.sentence'.
%
%    \begin{macrocode}
FUNCTION {end.clause}
{ output.state mid.clause =
    { mid.sentence 'output.state := }
    'skip$
  if$
}
%    \end{macrocode}
%
% Before writing anything out, we need to make sure there is something to be
% printed. We provide two functions for this: |output| is silent, while
% |output.check| will write a warning to the log that the field is empty. The
% latter consumes an additional string (at the top of the stack) for printing
% in the warning message.
%
%    \begin{macrocode}
FUNCTION {output}
{ duplicate$ empty$
    { pop$ end.clause }
    'output.nonnull
  if$
}
FUNCTION {output.check}
{ 't :=
  duplicate$ empty$
    { pop$ "empty " t * " in " * cite$ * warning$
      end.clause
    }
    'output.nonnull
  if$
}
%    \end{macrocode}
%
% We provide some convenience functions for changing the state from
% |mid.sentence|. The order of precedence is before.all, after.block,
% after.sentence, mid.clause/mid.sentence.
%
%    \begin{macrocode}
FUNCTION {new.block}
{ output.state before.all =
    'skip$
    { after.block 'output.state := }
  if$
}
FUNCTION {new.sentence}
{ output.state after.block =
    'skip$
    { output.state before.all =
        'skip$
        { after.sentence 'output.state := }
      if$
    }
  if$
}
FUNCTION {continue.clause}
{ output.state after.sentence =
    'skip$
    { output.state after.block =
        'skip$
        { output.state before.all =
            'skip$
            { mid.clause 'output.state := }
          if$
        }
      if$
    }
  if$
}
%    \end{macrocode}
%
% The |date.block| function is a handy alias.
%
%    \begin{macrocode}
FUNCTION {date.block}
{
  new.block
}
%    \end{macrocode}
%
% The |bibinfo| functions consume two tokens: the value and name of a field
% respectively. If the value is non-null, it is written out to the .bbl file in
% the form |\bibinfo{field name}| |{field value}|. If the value is missing, the
% tokens are replaced with an empty string; if the value is empty, it is put
% back on the stack. The |check| version does this silently, while the |warn|
% version issues warnings in the log file if the value is missing or empty.
%
%    \begin{macrocode}
FUNCTION {bibinfo.check}
{ swap$
  duplicate$ missing$
    {
      pop$ pop$
      ""
    }
    { duplicate$ empty$
        {
          swap$ pop$
        }
        { swap$
          "\bibinfo{" swap$ * "}{" * swap$ * "}" *
        }
      if$
    }
  if$
}
FUNCTION {bibinfo.warn}
{ swap$
  duplicate$ missing$
    {
      swap$ "missing " swap$ * " in " * cite$ * warning$ pop$
      ""
    }
    { duplicate$ empty$
        {
          swap$ "empty " swap$ * " in " * cite$ * warning$
        }
        { swap$
          "\bibinfo{" swap$ * "}{" * swap$ * "}" *
        }
      if$
    }
  if$
}
%    \end{macrocode}
%
% At the end of the reference, |fin.entry| writes out a final period and line
% break. We will handle the start of the reference a bit later.
%
%    \begin{macrocode}
FUNCTION {fin.entry}
{ add.period$
  write$
  newline$
}

%    \end{macrocode}
%
% \subsection{Convenience functions}
%
% We define some useful logical operations for use with |if$|.
%
%    \begin{macrocode}
FUNCTION {not}
{   { #0 }
    { #1 }
  if$
}
FUNCTION {and}
{   'skip$
    { pop$ #0 }
  if$
}
FUNCTION {or}
{   { pop$ #1 }
    'skip$
  if$
}
%    \end{macrocode}
%
% |new.block.checkb| consumes two tokens and starts a new block if at least one
% is non-empty.
%
%    \begin{macrocode}
FUNCTION {new.block.checkb}
{ empty$
  swap$ empty$
  and
    'skip$
    'new.block
  if$
}
%    \end{macrocode}
%
% |field.or.null| replaces an empty field value with an empty string.
%
%    \begin{macrocode}
FUNCTION {field.or.null}
{ duplicate$ empty$
    { pop$ "" }
    'skip$
  if$
}
%    \end{macrocode}
%
% The \texttt{emphasize} function applies emphasis.
%
%    \begin{macrocode}
FUNCTION {emphasize}
{ duplicate$ empty$
    { pop$ "" }
    { "\emph{" swap$ * "}" * }
  if$
}
%    \end{macrocode}
%
% The \texttt{tie.or.space.prefix} function prefixes a string with a non-breaking
% space if it is up to three characters long, or a normal space otherwise.
%
%    \begin{macrocode}
FUNCTION {tie.or.space.prefix}
{ duplicate$ text.length$ #3 <
    { "~" }
    { " " }
  if$
  swap$
}
%    \end{macrocode}
%
% The \texttt{capitalize} function makes putting strings into sentence case a
% little more robust.
%
%    \begin{macrocode}
FUNCTION {capitalize}
{ "u" change.case$ "t" change.case$ }
%    \end{macrocode}
%
% The \texttt{space.word} function adds a space to either side of a string.
%
%    \begin{macrocode}
FUNCTION {space.word}
{ " " swap$ * " " * }
%    \end{macrocode}
%
% |either.or.check| takes a string and a field, and if the field is not empty,
% writes a warning to the log file.
%
%    \begin{macrocode}
FUNCTION {either.or.check}
{ empty$
    'pop$
    { "can't use both " swap$ * " fields in " * cite$ * warning$ }
  if$
}

%    \end{macrocode}
%
% \subsection{Bibliography strings}
%
% Instead of dropping literal English words into the text of references, we
% will use these functions instead. This makes it easier to translate the style
% into other languages.
%
%    \begin{macrocode}
FUNCTION {bbl.and}
{ "and"}

FUNCTION {bbl.etal}
{ "et~al." }

FUNCTION {bbl.editors}
{ "eds." }

FUNCTION {bbl.editor}
{ "ed." }

FUNCTION {bbl.edby}
{ "edited by" }

FUNCTION {bbl.translator}
{ "Trans." }

FUNCTION {bbl.edition}
{ "ed." }

FUNCTION {bbl.volume}
{ "vol." }

FUNCTION {bbl.of}
{ "of" }

FUNCTION {bbl.number}
{ "no." }

FUNCTION {bbl.nr}
{ "no." }

FUNCTION {bbl.in}
{ "in" }

FUNCTION {bbl.pages}
{ "pp." }

FUNCTION {bbl.page}
{ "p." }

FUNCTION {bbl.chapter}
{ "chap." }

FUNCTION {bbl.techrep}
{ "Tech. Rep." }

FUNCTION {bbl.mthesis}
{ "Master's thesis" }

FUNCTION {bbl.phdthesis}
{ "Ph.D. thesis" }

FUNCTION {bbl.first}
{ "1st" }

FUNCTION {bbl.second}
{ "2nd" }

FUNCTION {bbl.third}
{ "3rd" }

FUNCTION {bbl.fourth}
{ "4th" }

FUNCTION {bbl.fifth}
{ "5th" }

FUNCTION {bbl.st}
{ "st" }

FUNCTION {bbl.nd}
{ "nd" }

FUNCTION {bbl.rd}
{ "rd" }

FUNCTION {bbl.th}
{ "th" }

%    \end{macrocode}
%
% These macros are codes that users can put in their .bib file instead of
% literal strings.
%
%    \begin{macrocode}
MACRO {jan} {"January"}

MACRO {feb} {"February"}

MACRO {mar} {"March"}

MACRO {apr} {"April"}

MACRO {may} {"May"}

MACRO {jun} {"June"}

MACRO {jul} {"July"}

MACRO {aug} {"August"}

MACRO {sep} {"September"}

MACRO {oct} {"October"}

MACRO {nov} {"November"}

MACRO {dec} {"December"}

%    \end{macrocode}
%
% This function allows us to convert cardinal numbers to English ordinals.
%
%    \begin{macrocode}
FUNCTION {eng.ord}
{ duplicate$ "1" swap$ *
  #-2 #1 substring$ "1" =
     { bbl.th * }
     { duplicate$ #-1 #1 substring$
       duplicate$ "1" =
         { pop$ bbl.st * }
         { duplicate$ "2" =
             { pop$ bbl.nd * }
             { "3" =
                 { bbl.rd * }
                 { bbl.th * }
               if$
             }
           if$
          }
       if$
     }
   if$
}

%    \end{macrocode}
%
% \subsection{Name handling}
%
% |format.names| consumes two tokens (field, string field name) and writes out a
% list of names in the form `Surname, Suffix, Initials Prefix'. To help with this,
% we define some additional integers and strings.
%
%    \begin{macrocode}
INTEGERS { nameptr namesleft numnames }
STRINGS  { bibinfo }
FUNCTION {format.names}
{ 'bibinfo :=
  duplicate$ empty$ 'skip$ {
  's :=
  "" 't :=
  #1 'nameptr :=
  s num.names$ 'numnames :=
  numnames 'namesleft :=
    { namesleft #0 > }
    { s nameptr
      "{ll}{, jj}{, f{.}.}{~vv}"
      format.name$
      bibinfo bibinfo.check
      't :=
      nameptr #1 >
        {
          namesleft #1 >
            { ", " * t * }
            {
              s nameptr "{ll}" format.name$ duplicate$ "others" =
                { 't := }
                { pop$ }
              if$
              t "others" =
                {
                  " " * bbl.etal *
                }
                {
                  bbl.and
                  space.word * t *
                }
              if$
            }
          if$
        }
        't
      if$
      nameptr #1 + 'nameptr :=
      namesleft #1 - 'namesleft :=
    }
  while$
  } if$
}
%    \end{macrocode}
%
% |format.names.ed| consumes two tokens (field, string field name) and writes
% out a list of names in the form `Initials Prefix Surname Suffix'.
%
%    \begin{macrocode}
FUNCTION {format.names.ed}
{ 'bibinfo :=
  duplicate$ empty$ 'skip$ {
  's :=
  "" 't :=
  #1 'nameptr :=
  s num.names$ 'numnames :=
  numnames 'namesleft :=
    { namesleft #0 > }
    { s nameptr
      "{f{.}.~}{vv~}{ll}{ jj}"
      format.name$
      bibinfo bibinfo.check
      't :=
      nameptr #1 >
        {
          namesleft #1 >
            { ", " * t * }
            {
              s nameptr "{ll}" format.name$ duplicate$ "others" =
                { 't := }
                { pop$ }
              if$
              t "others" =
                {

                  " " * bbl.etal *
                }
                {
                  bbl.and
                  space.word * t *
                }
              if$
            }
          if$
        }
        't
      if$
      nameptr #1 + 'nameptr :=
      namesleft #1 - 'namesleft :=
    }
  while$
  } if$
}
%    \end{macrocode}
%
% |format.authors| writes out author names.
%
%    \begin{macrocode}
FUNCTION {format.authors}
{ author "author" format.names
}
%    \end{macrocode}
%
% |format.editors| writes out editor names, and appends `ed.' or `eds.' as
% appropriate.
%
%    \begin{macrocode}
FUNCTION {get.bbl.editor}
{ editor num.names$ #1 > 'bbl.editors 'bbl.editor if$ }
FUNCTION {format.editors}
{ editor "editor" format.names duplicate$ empty$ 'skip$
    {
      "," *
      " " *
      get.bbl.editor
      *
    }
  if$
}
%    \end{macrocode}
%
% |format.translators| does likewise for translators, in the form
% `(name. Trans.)'.
%
%    \begin{macrocode}
FUNCTION {format.translators}
{ translator "translator" format.names.ed duplicate$ empty$ 'skip$
    {
      "." *
      " " *
      bbl.translator *
      "(" swap$ *
      ")" *
    }
  if$
}
%    \end{macrocode}
%
% |format.full.names| consumes one token (a name field) and, contrary to what
% the name might suggest, writes out a list of names in the form `Prefix Surname'.
%
%    \begin{macrocode}
FUNCTION {format.full.names}
{'s :=
 "" 't :=
  #1 'nameptr :=
  s num.names$ 'numnames :=
  numnames 'namesleft :=
    { namesleft #0 > }
    { s nameptr
      "{vv~}{ll}" format.name$
      't :=
      nameptr #1 >
        {
          namesleft #1 >
            { ", " * t * }
            {
              s nameptr "{ll}" format.name$ duplicate$ "others" =
                { 't := }
                { pop$ }
              if$
              t "others" =
                {
                  " " * bbl.etal *
                }
                {
                  bbl.and
                  space.word * t *
                }
              if$
            }
          if$
        }
        't
      if$
      nameptr #1 + 'nameptr :=
      namesleft #1 - 'namesleft :=
    }
  while$
}
%    \end{macrocode}
%
% |author.editor.key.full| returns the author part of an author--year citation.
% In order of preference, it outputs the author surnames, editor surnames, the
% |key| field, or the first three characters of the entry key. It is used in
% |book| and |inbook| entry types.
%
%    \begin{macrocode}
FUNCTION {author.editor.key.full}
{ author empty$
    { editor empty$
        { key empty$
            { cite$ #1 #3 substring$ }
            'key
          if$
        }
        { editor format.full.names }
      if$
    }
    { author format.full.names }
  if$
}
%    \end{macrocode}
%
% |editor.key.full| does similar but skips the author names. It is used for the
% |proceedings| entry type.
%
%    \begin{macrocode}
FUNCTION {editor.key.full}
{ editor empty$
    { key empty$
         { cite$ #1 #3 substring$ }
          'key
      if$
    }
    { editor format.full.names }
  if$
}
%    \end{macrocode}
%
% |author.key.full| does similar but skips the editor names. It is used for all
% other entry types.
%
%    \begin{macrocode}
FUNCTION {author.key.full}
{ author empty$
    { key empty$
         { cite$ #1 #3 substring$ }
          'key
      if$
    }
    { author format.full.names }
  if$
}
%    \end{macrocode}
%
% |make.full.names| chooses between the above.
%
%    \begin{macrocode}
FUNCTION {make.full.names}
{ type$ "book" =
  type$ "inbook" =
  or
    'author.editor.key.full
    { type$ "proceedings" =
        'editor.key.full
        'author.key.full
      if$
    }
  if$
}
%    \end{macrocode}
%
% |output.bibitem| writes out the very start of the entry in the bibliography,
% which looks like |\bibitem[{author(year)}]{entry key}|. It then starts a new
% line and puts the entry driver into the |before.all| output state. (The year
% part comes from the |label| variable which we will populate later.) 
%
%    \begin{macrocode}
FUNCTION {output.bibitem}
{ newline$
  "\bibitem[{" write$
  label write$
  ")" make.full.names duplicate$ short.list =
     { pop$ }
     { * }
   if$
  "}]{" * write$
  cite$ write$
  "}" write$
  newline$
  ""
  before.all 'output.state :=
}
%    \end{macrocode}
%
% |format.key| takes a field (author or editor). If the field is empty, it
% replaces it with the key field (if not non-null). Otherwise, replaces it with
% an empty string. It is used as a last resort to stop an entry beginning with
% the date.
%
%    \begin{macrocode}
FUNCTION {format.key}
{ empty$
    { key field.or.null }
    { "" }
  if$
}
%    \end{macrocode}
%
% \subsection{Titles}
%
% |select.language| checks if the field before was written out, and if so,
% wraps |{\selectlanguage{language}\dots}| around it.
%
%    \begin{macrocode}
FUNCTION {select.language}
{ duplicate$ empty$
    'pop$
    { language empty$
        'skip$
        { "{\selectlanguage{" language * "}" * swap$ * "}" * }
      if$
    }
  if$
}
%    \end{macrocode}
%
% |format.title| writes out the title and language fields, where the title
% should be in the upright shape.
%
%    \begin{macrocode}
FUNCTION {format.title}
{ title
  duplicate$ empty$ 'skip$
    { "t" change.case$ }
  if$
  "title" bibinfo.check
  duplicate$ empty$
    'skip$
    { select.language }
  if$
}
%    \end{macrocode}
%
% |format.btitle| writes out the title and language fields, where the title
% should be in the italic shape.
%
%    \begin{macrocode}
FUNCTION {format.btitle}
{ title
  duplicate$ empty$ 'skip$
    { "t" change.case$ }
  if$
  "title" bibinfo.check
  duplicate$ empty$ 'skip$
    {
      emphasize
      select.language
    }
  if$
}
%    \end{macrocode}
%
% |format.titleaddon| writes out the titleaddon field.
%
%    \begin{macrocode}
FUNCTION {format.titleaddon}
{ titleaddon
  duplicate$ empty$ 'skip$
    { "titleaddon" bibinfo.check }
  if$
}
%    \end{macrocode}
%
% |get.title.addenda| is a convenience for writing out the elements that
% normally follow the title in |mid.clause| state.
%
%    \begin{macrocode}
FUNCTION {get.title.addenda}
{ continue.clause
  format.titleaddon output
  continue.clause
  format.translators output
}
%    \end{macrocode}
%
% |word.in| provides the `In:' before book titles.
%
%    \begin{macrocode}
FUNCTION {word.in}
{ bbl.in capitalize
  ":" *
  " " * }
%    \end{macrocode}
%
% |format.booktitle| writes out the booktitle field in italics.
%
%    \begin{macrocode}
FUNCTION {format.booktitle}
{ booktitle
  duplicate$ empty$ 'skip$
    { "t" change.case$ }
  if$
  "booktitle" bibinfo.check
  emphasize
}
%    \end{macrocode}
%
% |format.in.ed.booktitle| writes out the booktitle, prepended by `In: editors,
% eds.'
%
%    \begin{macrocode}
FUNCTION {format.in.ed.booktitle}
{ format.booktitle duplicate$ empty$ 'skip$
    {
      editor "editor" format.names.ed duplicate$ empty$ 'pop$
        {
          "," *
          " " *
          get.bbl.editor
          " " *
          * swap$
          *
          word.in swap$ *
        }
      if$
    }
  if$
}
%    \end{macrocode}
%
% \subsection{Dates}
%
% |format.date| writes out the year and any disambiguation label.
%
%    \begin{macrocode}
FUNCTION {format.date}
{ year "year" bibinfo.check duplicate$ empty$
    {
    }
    'skip$
  if$
  extra.label *
}
%    \end{macrocode}
%
% \subsection{Series, volumes, editions}
%
% |format.bvolume| writes out the volume and series information (in italics),
% with a comma between. A warning is given if |number| is also given.
%
%    \begin{macrocode}
FUNCTION {format.bvolume}
{ volume empty$
    { "" }
    { bbl.volume volume tie.or.space.prefix
      "volume" bibinfo.check * *
      series "series" bibinfo.check
      duplicate$ empty$ 'pop$
        { emphasize ", " * swap$ * }
      if$
      "volume and number" number either.or.check
    }
  if$
}
%    \end{macrocode}
%
% |format.number.series| returns an empty string if a volume has been specified.
% Otherwise, it writes out the series and the number separated by a space.
%
%    \begin{macrocode}
FUNCTION {format.number.series}
{ volume empty$
    { number empty$
        { series field.or.null }
        { series empty$
            { number "number" bibinfo.check }
            { series "series" bibinfo.check
              number tie.or.space.prefix "number" bibinfo.check * *
            }
          if$
        }
      if$
    }
    { "" }
  if$
}
%    \end{macrocode}
%
% |is.num| converts a string to an integer and ensures it is in the range 0--9.
%
%    \begin{macrocode}
FUNCTION {is.num}
{ chr.to.int$
  duplicate$ "0" chr.to.int$ < not
  swap$ "9" chr.to.int$ > not and
}
%    \end{macrocode}
%
% |extract.num| takes a string. If it begins with an integer, it returns just
% that integer. Otherwise it returns the whole string.
%
%    \begin{macrocode}
FUNCTION {extract.num}
{ duplicate$ 't :=
  "" 's :=
  { t empty$ not }
  { t #1 #1 substring$
    t #2 global.max$ substring$ 't :=
    duplicate$ is.num
      { s swap$ * 's := }
      { pop$ "" 't := }
    if$
  }
  while$
  s empty$
    'skip$
    { pop$ s }
  if$
}
%    \end{macrocode}
%
% |convert.edition| should operate on the edition field. If the value begins
% with a number (or is an English ordinal word from `first' to `fifth'), then it
% returns the bibliography string version of the ordinal number (1st, 2nd,
% 3rd\dots). Otherwise, returns the edition field verbatim.
%
%    \begin{macrocode}
FUNCTION {convert.edition}
{ extract.num "l" change.case$ 's :=
  s "first" = s "1" = or
    { bbl.first 't := }
    { s "second" = s "2" = or
        { bbl.second 't := }
        { s "third" = s "3" = or
            { bbl.third 't := }
            { s "fourth" = s "4" = or
                { bbl.fourth 't := }
                { s "fifth" = s "5" = or
                    { bbl.fifth 't := }
                    { s #1 #1 substring$ is.num
                        { s eng.ord 't := }
                        { edition 't := }
                      if$
                    }
                  if$
                }
              if$
            }
          if$
        }
      if$
    }
  if$
  t
}
%    \end{macrocode}
%
% |format.edition| writes out the converted edition field and `ed.'
%
%    \begin{macrocode}
FUNCTION {format.edition}
{ edition duplicate$ empty$ 'skip$
    {
      convert.edition
      output.state mid.sentence =
        { "l" }
        { "t" }
      if$ change.case$
      "edition" bibinfo.check
      " " * bbl.edition *
    }
  if$
}
%    \end{macrocode}
%
% |format.thesis.type| simply writes out the type field with an initial capital.
%
%    \begin{macrocode}
FUNCTION {format.thesis.type}
{ type duplicate$ empty$
    'pop$
    { swap$ pop$
      "t" change.case$ "type" bibinfo.check
    }
  if$
}
%    \end{macrocode}
%
% |format.tr.number| writes out the series, type and number fields as one of
% `(number)', `(type number)', `(series, number)', or `(series, type number)'.
%
%    \begin{macrocode}
FUNCTION {format.tr.number}
{ series duplicate$ empty$
    { pop$
      type duplicate$ empty$
        { pop$
          number duplicate$ empty$
            { pop$ ""
            }
            { "(" swap$
              "number" bibinfo.check
              * ")" *
            }
          if$
        }
        { "(" swap$
          "t" change.case$ "type" bibinfo.check
          *
          number duplicate$ empty$
            'pop$
            { tie.or.space.prefix
              "number" bibinfo.check
              * *
            }
          if$
          ")" *
        }
      if$
    }
    { "(" swap$ *
      type duplicate$ empty$
        { pop$
          number duplicate$ empty$
            'pop$
            { ", " swap$
              "number" bibinfo.check
              * *
            }
          if$
        }
        { "t" change.case$ "type" bibinfo.check
          *
          number duplicate$ empty$
            'pop$
            { tie.or.space.prefix
              "number" bibinfo.check
              * *
            }
          if$
        }
      if$
      ")" *
    }
  if$
}
%    \end{macrocode}
%
% \subsection{Notes}
%
% |format.note| writes out the note field, but changes the case of the first
% letter according to the output state (unless protected by braces).
%
%    \begin{macrocode}
FUNCTION {format.note}
{ note empty$
    { "" }
    { note #1 #1 substring$
      duplicate$ "{" =
        'skip$
        { output.state mid.sentence =
          output.state mid.clause =
          or
            { "l" }
            { "u" }
          if$
          change.case$
        }
      if$
      note #2 global.max$ substring$ * "note" bibinfo.check
    }
  if$
}
%    \end{macrocode}
%
% \subsection{Addresses and publishers}
%
% |format.org.or.pub| intercepts a publisher (or whatever) being written out as
% bibinfo. If the write will succeed and an address has been specified, the
% address is written out first, then a colon and space; then the original write
% is allowed to proceed.
%
%    \begin{macrocode}
FUNCTION {format.org.or.pub}
{ 't :=
  ""
  address empty$ t empty$ and
    'skip$
    {
      address "address" bibinfo.check *
      t empty$
        'skip$
        { address empty$
            'skip$
            { ": " * }
          if$
          t *
        }
      if$
    }
  if$
}
%    \end{macrocode}
%
% The next three functions write out the address plus the publisher,
% organization and institution respectively.
%
%    \begin{macrocode}
FUNCTION {format.publisher.address}
{ publisher "publisher" bibinfo.warn format.org.or.pub
}
FUNCTION {format.organization.address}
{ organization "organization" bibinfo.check format.org.or.pub
}
FUNCTION {format.institution.address}
{ institution "institution" bibinfo.check format.org.or.pub
}

%    \end{macrocode}
%
% \subsection{Pinpointing}
%
% |multi.page.check| should operate on the pages field. If the value is a range,
% that is, contains `-', `,' or `+', returns 1, otherwise returns 0.
%
%    \begin{macrocode}
INTEGERS { multiresult }
FUNCTION {multi.page.check}
{ 't :=
  #0 'multiresult :=
    { multiresult not
      t empty$ not
      and
    }
    { t #1 #1 substring$
      duplicate$ "-" =
      swap$ duplicate$ "," =
      swap$ "+" =
      or or
        { #1 'multiresult := }
        { t #2 global.max$ substring$ 't := }
      if$
    }
  while$
  multiresult
}
%    \end{macrocode}
%
% |n.dashify| converts hyphens in page ranges into en dashes.
%
%    \begin{macrocode}
FUNCTION {n.dashify}
{
  't :=
  ""
    { t empty$ not }
    { t #1 #1 substring$ "-" =
        { t #1 #2 substring$ "--" = not
            { "--" *
              t #2 global.max$ substring$ 't :=
            }
            {   { t #1 #1 substring$ "-" = }
                { "-" *
                  t #2 global.max$ substring$ 't :=
                }
              while$
            }
          if$
        }
        { t #1 #1 substring$ *
          t #2 global.max$ substring$ 't :=
        }
      if$
    }
  while$
}
%    \end{macrocode}
%
% |format.pages| writes out the pages fields, prefixed by `p.' or `pp.' as
% appropriate.
%
%    \begin{macrocode}
FUNCTION {format.pages}
{ pages duplicate$ empty$ 'skip$
    { duplicate$ multi.page.check
        {
          bbl.pages swap$
          n.dashify
        }
        {
          bbl.page swap$
        }
      if$
      "pages" bibinfo.check
      *
    }
  if$
}
%    \end{macrocode}
%
% |format.journal.pages| does similar, but it looks at the preceding string and,
% if both are non-empty, inserts a comma between it and the pages string.
%
%    \begin{macrocode}
FUNCTION {format.journal.pages}
{ pages duplicate$ empty$ 'pop$
    { swap$ duplicate$ empty$
        { pop$ pop$ format.pages }
        {
          ", " *
          swap$
          n.dashify
          pages multi.page.check
            'bbl.pages
            'bbl.page
          if$
          swap$
          "pages" bibinfo.check
          * *
        }
      if$
    }
  if$
}
%    \end{macrocode}
%
% |format.journal.eid| writes out the EID, but it looks at the preceding string
% and, if both are non-empty, inserts a comma between it and the EID string.
% The EID is a code used by some online-only journals to identify an article
% within a volume or issue instead of a page range (since the page numbering
% resets for each article, not each volume\slash issue).
%
%    \begin{macrocode}
FUNCTION {format.journal.eid}
{ eid "eid" bibinfo.check
  duplicate$ empty$ 'pop$
    { swap$ duplicate$ empty$ 'skip$
      {
          ", " *
      }
      if$
      swap$ *
    }
  if$
}
%    \end{macrocode}
%
% |format.vol.num.pages| writes out the journal pinpointing in the form
% `volume(number), eid\slash pages'. It writes a warning to the log if |number|
% is given instead of |volume|: if a journal does not organize itself into
% volumes, treat every issue as its own volume!
%
%    \begin{macrocode}
FUNCTION {format.vol.num.pages}
{ volume field.or.null
  duplicate$ empty$ 'skip$
    {
      "volume" bibinfo.check
    }
  if$
  number "number" bibinfo.check duplicate$ empty$ 'skip$
    {
      swap$ duplicate$ empty$
        { "there's a number but no volume in " cite$ * warning$ }
        'skip$
      if$
      swap$
      "(" swap$ * ")" *
    }
  if$ *
  eid empty$
    { format.journal.pages }
    { format.journal.eid }
  if$
}
%    \end{macrocode}
%
% |format.chapter.pages| writes out pinpointing information for books, in the
% form `chap.\@ chapter, p.pages'.
%
%    \begin{macrocode}
FUNCTION {format.chapter.pages}
{ chapter empty$
    'format.pages
    { type empty$
        { bbl.chapter }
        { type "l" change.case$
          "type" bibinfo.check
        }
      if$
      chapter tie.or.space.prefix
      "chapter" bibinfo.check
      * *
      pages empty$
        'skip$
        { ", " * format.pages * }
      if$
    }
  if$
}
%    \end{macrocode}
%
% \subsection{URLs}
%
% |format.eprint| constructs a string |\eprint[archive]{eprint}|.
%
%    \begin{macrocode}
FUNCTION {format.eprint}
{ eprint duplicate$ empty$
    'skip$
    { "\eprint"
      archive empty$
        'skip$
        { "[" * archive * "]" * }
      if$
      "{" * swap$ * "}" *
    }
  if$
}
%    \end{macrocode}
%
% |format.url| constructs a string |Available from: \url{url} [Accessed urldate]|.
%
%    \begin{macrocode}
FUNCTION {format.url}
{ doi empty$
    { url }
    { "http://doi.org/" doi * }
  if$
  duplicate$ empty$
    { pop$ "" }
    { "\urlprefix\url{" swap$ * "}" * }
  if$
  urldate empty$
    'skip$
    { " [\urldateprefix{}" * urldate * "]" * }
  if$
}
%    \end{macrocode}
%
% \subsection{Cross referencing between entries}
%
% The following functions are used for entries that look like `Author, (Year).
% Title. In: Author (Year).' In practice, they should never be used with Harvard
% (Bath).
%
%    \begin{macrocode}
FUNCTION {format.article.crossref}
{
  word.in
  " \cite{" * crossref * "}" *
}
FUNCTION {format.incoll.inproc.crossref}
{
  word.in
  " \cite{" * crossref * "}" *
}
%    \end{macrocode}
%
% |format.book.crossref| is a bit different as it is intended for cases where
% the given book is volume X of a larger work: `Author, (Year). Title. Vol.\@
% volume of Author (Year).' In the absence of a volume, writes a warning to the
% log and behaves like the above.
%
%    \begin{macrocode}
FUNCTION {format.book.crossref}
{ volume duplicate$ empty$
    { "empty volume in " cite$ * "'s crossref of " * crossref * warning$
      pop$ word.in
    }
    { bbl.volume
      capitalize
      swap$ tie.or.space.prefix "volume" bibinfo.check * * bbl.of space.word *
    }
  if$
  " \cite{" * crossref * "}" *
}

%    \end{macrocode}
%
% \subsection{Formatting drivers for specific entry types}
%
% These functions are called when an entry of the given type is being
% formatted.
%
%    \begin{macrocode}
FUNCTION {article}
{ output.bibitem
  format.authors "author" output.check
  author format.key output
  format.date "year" output.check
  date.block
  format.title "title" output.check
  get.title.addenda
  new.block
  crossref missing$
    {
      journal
      "journal" bibinfo.check
      emphasize
      "journal" output.check
      format.vol.num.pages output
    }
    { format.article.crossref output.nonnull
      format.pages output
    }
  if$
  new.block
  format.note output
  new.block
  format.eprint output
  format.url output
  fin.entry
}
FUNCTION {book}
{ output.bibitem
  author empty$
    { editor empty$
        { format.btitle "title" output.check
          format.date "year" output.check
          year empty$ 'skip$ { date.block } if$
          get.title.addenda
        }
        { format.editors "author and editor" output.check
          editor format.key output
          format.date "year" output.check
          date.block
          format.btitle "title" output.check
          get.title.addenda
        }
      if$
    }
    { format.authors output.nonnull
      crossref missing$
        { "author and editor" editor either.or.check }
        'skip$
      if$
      format.date "year" output.check
      date.block
      format.btitle "title" output.check
      get.title.addenda
    }
  if$
  format.bvolume output
  format.number.series output
  new.block
  format.edition output
  crossref missing$
    { new.sentence
      format.publisher.address output
    }
    {
      new.block
      format.book.crossref output.nonnull
    }
  if$
  new.block
  format.note output
  new.block
  format.eprint output
  format.url output
  fin.entry
}
FUNCTION {booklet}
{ output.bibitem
  author empty$
    { format.btitle "title" output.check
      format.date "year" output.check
      year empty$ 'skip$ { date.block } if$
      get.title.addenda
    }
    { format.authors output
      author format.key output
      format.date "year" output.check
      date.block
      format.btitle "title" output.check
      get.title.addenda
    }
  if$
  new.block
  howpublished "howpublished" bibinfo.check output
  publisher empty$ 'skip$
    { new.block}
  if$
  format.publisher.address output
  new.block
  format.note output
  new.block
  format.eprint output
  format.url output
  fin.entry
}

FUNCTION {inbook}
{ output.bibitem
  author empty$
    { format.editors "author and editor" output.check
      editor format.key output
    }
    { format.authors output.nonnull
      crossref missing$
        { "author and editor" editor either.or.check }
        'skip$
      if$
    }
  if$
  format.date "year" output.check
  date.block
  format.btitle "title" output.check
  get.title.addenda
  crossref missing$
    {
      format.publisher.address output
      format.bvolume output
      format.chapter.pages "chapter and pages" output.check
      new.block
      format.number.series output
      new.sentence
    }
    {
      format.chapter.pages "chapter and pages" output.check
      new.block
      format.book.crossref output.nonnull
    }
  if$
  format.edition output
  new.block
  format.note output
  new.block
  format.eprint output
  format.url output
  fin.entry
}

FUNCTION {incollection}
{ output.bibitem
  format.authors "author" output.check
  author format.key output
  format.date "year" output.check
  date.block
  format.title "title" output.check
  get.title.addenda
  new.block
  crossref missing$
    { format.in.ed.booktitle "booktitle" output.check
      format.publisher.address output
      format.bvolume output
      format.number.series output
      format.chapter.pages output
      new.sentence
      format.edition output
    }
    { format.incoll.inproc.crossref output.nonnull
      format.chapter.pages output
    }
  if$
  new.block
  format.note output
  new.block
  format.eprint output
  format.url output
  fin.entry
}
FUNCTION {inproceedings}
{ output.bibitem
  format.authors "author" output.check
  author format.key output
  format.date "year" output.check
  date.block
  format.title "title" output.check
  get.title.addenda
  new.block
  crossref missing$
    { format.in.ed.booktitle "booktitle" output.check
      venue "venue" output.check
      new.sentence
      publisher empty$
        { format.organization.address output }
        { organization "organization" bibinfo.check output
          format.publisher.address output
        }
      if$
      format.bvolume output
      format.number.series output
      format.pages output
    }
    { format.incoll.inproc.crossref output.nonnull
      format.pages output
    }
  if$
  new.block
  format.note output
  new.block
  format.eprint output
  format.url output
  fin.entry
}
FUNCTION {conference} { inproceedings }
FUNCTION {manual}
{ output.bibitem
  author empty$
    { format.btitle "title" output.check
      format.date "year" output.check
      year empty$ 'skip$ { date.block } if$
      get.title.addenda
    }
    { format.authors output
      author format.key output
      format.date "year" output.check
      date.block
      format.btitle "title" output.check
      get.title.addenda
    }
  if$
  new.block
  format.number.series output
  organization address new.block.checkb
  format.organization.address output
  format.edition output
  new.block
  format.note output
  new.block
  format.eprint output
  format.url output
  fin.entry
}

FUNCTION {mastersthesis}
{ output.bibitem
  format.authors "author" output.check
  author format.key output
  format.date "year" output.check
  date.block
  format.btitle "title" output.check
  get.title.addenda
  new.block
  bbl.mthesis format.thesis.type output.nonnull
  new.block
  school "school" bibinfo.warn output
  address "address" bibinfo.check output
  new.block
  format.note output
  new.block
  format.eprint output
  format.url output
  fin.entry
}

FUNCTION {misc}
{ output.bibitem
  author empty$
    { format.title "title" output.check
      format.date "year" output.check
      year empty$ 'skip$ { date.block } if$
      get.title.addenda
    }
    { format.authors output
      author format.key output
      format.date "year" output.check
      date.block
      format.title "title" output.check
      get.title.addenda
    }
  if$
  new.block
  howpublished "howpublished" bibinfo.check output
  new.block
  format.note output
  new.block
  format.eprint output
  format.url output
  fin.entry
}
FUNCTION {phdthesis}
{ output.bibitem
  format.authors "author" output.check
  author format.key output
  format.date "year" output.check
  date.block
  format.btitle "title" output.check
  get.title.addenda
  new.block
  bbl.phdthesis format.thesis.type output.nonnull
  new.block
  school "school" bibinfo.warn output
  address "address" bibinfo.check output
  new.block
  format.note output
  new.block
  format.eprint output
  format.url output
  fin.entry
}

FUNCTION {proceedings}
{ output.bibitem
  editor empty$
    { format.btitle "title" output.check
      format.date "year" output.check
      year empty$ 'skip$ { date.block } if$
      get.title.addenda
    }
    { format.editors output
      editor format.key output
      format.date "year" output.check
      date.block
      format.btitle "title" output.check
      get.title.addenda
    }
  if$
  format.bvolume output
  format.number.series output
  new.sentence
  publisher empty$
    { format.organization.address output }
    { organization "organization" bibinfo.check output
      format.publisher.address output
    }
  if$
  new.block
  format.note output
  new.block
  format.eprint output
  format.url output
  fin.entry
}

FUNCTION {techreport}
{ output.bibitem
  author empty$
    { format.btitle "title" output.check
      format.date "year" output.check
      year empty$ 'skip$ { date.block } if$
      get.title.addenda
    }
    { format.authors output
      author format.key output
      format.date "year" output.check
      date.block
      format.btitle "title" output.check
      get.title.addenda
    }
  if$
  new.block
  publisher empty$
    { format.institution.address output }
    { institution "institution" bibinfo.check output
      format.publisher.address output
    }
  if$
  format.tr.number output
  new.block
  format.note output
  new.block
  format.eprint output
  format.url output
  fin.entry
}

FUNCTION {unpublished}
{ output.bibitem
  author empty$
    { format.btitle "title" output.check
      format.date "year" output.check
      year empty$ 'skip$ { date.block } if$
      get.title.addenda
    }
    { format.authors output
      author format.key output
      format.date "year" output.check
      date.block
      format.btitle "title" output.check
      get.title.addenda
    }
  if$
  new.block
  format.note output
  new.block
  format.eprint output
  format.url output
  fin.entry
}

%    \end{macrocode}
%
% Here are a list of aliases supported by this style.
%
%    \begin{macrocode}
FUNCTION {audio} { booklet }
FUNCTION {movie} { booklet }
FUNCTION {music} { booklet }
FUNCTION {video} { booklet }
FUNCTION {dataset} { manual }
FUNCTION {electronic} { manual }
FUNCTION {online} { manual }
FUNCTION {patent} { manual }
FUNCTION {software} { manual }
FUNCTION {standard} { manual }
FUNCTION {www} { manual }
%    \end{macrocode}
%
% The fallback driver is \texttt{misc}, as with most styles.
%
%    \begin{macrocode}
FUNCTION {default.type} { misc }

%    \end{macrocode}
%
% \subsection{Bibliography processing}
%
% We are now ready for Bib\TeX\ to read in the .bib file.
%
%    \begin{macrocode}
READ
%    \end{macrocode}
%
% These functions are used to generate a sort key for each entry in the .bib
% file.
%
%    \begin{macrocode}
FUNCTION {sortify}
{ purify$
  "l" change.case$
}
INTEGERS { len }
FUNCTION {chop.word}
{ 's :=
  'len :=
  s #1 len substring$ =
    { s len #1 + global.max$ substring$ }
    's
  if$
}
FUNCTION {format.lab.names}
{'s :=
 "" 't :=
  #1 'nameptr :=
  s num.names$ 'numnames :=
  numnames 'namesleft :=
    { namesleft #0 > }
    { s nameptr
      "{vv~}{ll}" format.name$
      't :=
      nameptr #1 >
        {
          nameptr #2 =
          numnames #3 > and
            { "others" 't :=
              #1 'namesleft := }
            'skip$
          if$
          namesleft #1 >
            { ", " * t * }
            {
              s nameptr "{ll}" format.name$ duplicate$ "others" =
                { 't := }
                { pop$ }
              if$
              t "others" =
                {
                  " " * bbl.etal *
                }
                {
                  bbl.and
                  space.word * t *
                }
              if$
            }
          if$
        }
        't
      if$
      nameptr #1 + 'nameptr :=
      namesleft #1 - 'namesleft :=
    }
  while$
}

FUNCTION {author.key.label}
{ author empty$
    { key empty$
        { cite$ #1 #3 substring$ }
        'key
      if$
    }
    { author format.lab.names }
  if$
}

FUNCTION {author.editor.key.label}
{ author empty$
    { editor empty$
        { key empty$
            { cite$ #1 #3 substring$ }
            'key
          if$
        }
        { editor format.lab.names }
      if$
    }
    { author format.lab.names }
  if$
}

FUNCTION {editor.key.label}
{ editor empty$
    { key empty$
        { cite$ #1 #3 substring$ }
        'key
      if$
    }
    { editor format.lab.names }
  if$
}

FUNCTION {calc.short.authors}
{ type$ "book" =
  type$ "inbook" =
  or
    'author.editor.key.label
    { type$ "proceedings" =
        'editor.key.label
        'author.key.label
      if$
    }
  if$
  'short.list :=
}

FUNCTION {calc.label}
{ calc.short.authors
  short.list
  "("
  *
  year duplicate$ empty$
     { pop$ "" }
     'skip$
  if$
  *
  'label :=
}

FUNCTION {sort.format.names}
{ 's :=
  #1 'nameptr :=
  ""
  s num.names$ 'numnames :=
  numnames 'namesleft :=
    { namesleft #0 > }
    { s nameptr
      "{ll{ }}{  f{ }}{  jj{ }}{ vv{}}"
      format.name$ 't :=
      nameptr #1 >
        {
          "   "  *
          namesleft #1 = t "others" = and
            { "zzzzz" 't := }
            'skip$
          if$
          t sortify *
        }
        { t sortify * }
      if$
      nameptr #1 + 'nameptr :=
      namesleft #1 - 'namesleft :=
    }
  while$
}

FUNCTION {sort.format.title}
{ 't :=
  "A " #2
    "An " #3
      "The " #4 t chop.word
    chop.word
  chop.word
  sortify
  #1 global.max$ substring$
}
FUNCTION {author.sort}
{ author empty$
    { title empty$
        { key empty$
            { "to sort, need author, title, or key in " cite$ * warning$
              ""
            }
            { key sortify }
          if$
        }
        { title sort.format.title }
      if$
    }
    { author sort.format.names }
  if$
}
FUNCTION {author.editor.sort}
{ author empty$
    { editor empty$
        { title empty$
            { key empty$
                { "to sort, need author, editor, title, or key in " cite$ * warning$
                  ""
                }
                { key sortify }
              if$
            }
            { title sort.format.title }
          if$
        }
        { editor sort.format.names }
      if$
    }
    { author sort.format.names }
  if$
}
FUNCTION {editor.sort}
{ editor empty$
    { title empty$
        { key empty$
            { "to sort, need editor, title, or key in " cite$ * warning$
              ""
            }
            { key sortify }
          if$
        }
        { title sort.format.title }
      if$
    }
    { editor sort.format.names }
  if$
}

%    \end{macrocode}
%
% The \texttt{presort} function applies the above functions to a given entry to
% generate its sort key and save it to the \lstinline[style=dtxbst]!sort.key$!
% variable.
%
%    \begin{macrocode}
FUNCTION {presort}
{ calc.label
  label sortify
  "    "
  *
  type$ "book" =
  type$ "inbook" =
  or
    'author.editor.sort
    { type$ "proceedings" =
        'editor.sort
        'author.sort
      if$
    }
  if$
  #1 entry.max$ substring$
  'sort.label :=
  sort.label
  *
  "    "
  *
  title field.or.null
  sort.format.title
  *
  #1 entry.max$ substring$
  'sort.key$ :=
}

%    \end{macrocode}
%
% We now tell Bib\TeX\ to go through each entry in turn and apply the
% \texttt{presort} function, then use the resultant sort keys to put the entries
% in order.
%
%    \begin{macrocode}
ITERATE {presort}
SORT
%    \end{macrocode}
%
% If there are any duplicate author--year combinations, these functions detect
% them and disambiguate them by adding a letter to the year.
%
%    \begin{macrocode}
STRINGS { last.label next.extra }
INTEGERS { last.extra.num last.extra.num.extended last.extra.num.blank number.label }
FUNCTION {initialize.extra.label.stuff}
{ #0 int.to.chr$ 'last.label :=
  "" 'next.extra :=
  #0 'last.extra.num :=
  "a" chr.to.int$ #1 - 'last.extra.num.blank :=
  last.extra.num.blank 'last.extra.num.extended :=
  #0 'number.label :=
}
FUNCTION {forward.pass}
{ last.label label =
    { last.extra.num #1 + 'last.extra.num :=
      last.extra.num "z" chr.to.int$ >
       { "a" chr.to.int$ 'last.extra.num :=
         last.extra.num.extended #1 + 'last.extra.num.extended :=
       }
       'skip$
      if$
      last.extra.num.extended last.extra.num.blank >
        { last.extra.num.extended int.to.chr$
          last.extra.num int.to.chr$
          * 'extra.label := }
        { last.extra.num int.to.chr$ 'extra.label := }
      if$
    }
    { "a" chr.to.int$ 'last.extra.num :=
      "" 'extra.label :=
      label 'last.label :=
    }
  if$
  number.label #1 + 'number.label :=
}
FUNCTION {reverse.pass}
{ next.extra "b" =
    { "a" 'extra.label := }
    'skip$
  if$
  extra.label 'next.extra :=
  extra.label
  duplicate$ empty$
    'skip$
    { "{\natexlab{" swap$ * "}}" * }
  if$
  'extra.label :=
  label extra.label * 'label :=
}
%    \end{macrocode}
%
% Applying these functions is a bit more tricky, but once we have done that,
% we re-sort the entries to make absolutely sure they in the right order.
%
%    \begin{macrocode}
EXECUTE {initialize.extra.label.stuff}
ITERATE {forward.pass}
REVERSE {reverse.pass}
FUNCTION {bib.sort.order}
{ sort.label
  "    "
  *
  year field.or.null sortify
  *
  "    "
  *
  title field.or.null
  sort.format.title
  *
  #1 entry.max$ substring$
  'sort.key$ :=
}
ITERATE {bib.sort.order}
SORT
%    \end{macrocode}
%
% We now write out the beginning of the bibliography to a .bbl file. This
% includes opening a \LaTeX\ environment, \texttt{thebibliography}, and some
% user-redefinable strings.
%
%    \begin{macrocode}
FUNCTION {begin.bib}
{ preamble$ empty$
    'skip$
    { preamble$ write$ newline$ }
  if$
  "\begin{thebibliography}{" number.label int.to.str$ * "}" *
  write$ newline$
  "\providecommand{\natexlab}[1]{#1}"
  write$ newline$
  "\providecommand{\url}[1]{\texttt{#1}}"
  write$ newline$
  "\providecommand{\urlprefix}{Available from: }"
  write$ newline$
  "\providecommand{\urldateprefix}{Accessed }"
  write$ newline$
  "\providecommand{\selectlanguage}[1]{\relax}"
  write$ newline$
  "\providecommand{\bibinfo}[2]{#2}"
  write$ newline$
  "\providecommand{\eprint}[2][]{\url{#2}}"
  write$ newline$
}
EXECUTE {begin.bib}
EXECUTE {init.state.consts}
%    \end{macrocode}
%
% We go through each entry in turn, in the right order, and write the
% formatted text out to the file.
%
%    \begin{macrocode}
ITERATE {call.type$}
%    \end{macrocode}
%
% After the last entry, we close the \LaTeX\ environment and stop writing to the
% .bbl file.
%
%    \begin{macrocode}
FUNCTION {end.bib}
{ newline$
  "\end{thebibliography}" write$ newline$
}
EXECUTE {end.bib}
%    \end{macrocode}
% \iffalse
%</bst>
% \fi
