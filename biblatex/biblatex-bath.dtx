% \iffalse meta-comment
%<*internal>
\iffalse
%</internal>
%<*readme>
# biblatex-bath: Harvard referencing style as recommended by the University of Bath Library
%</readme>
%<*internal>
\fi
\def\nameofplainTeX{plain}
\ifx\fmtname\nameofplainTeX\else
  \expandafter\begingroup
\fi
%</internal>
%<*install>
\input docstrip.tex
\keepsilent
\askforoverwritefalse

\nopreamble\nopostamble

\usedir{doc/latex/\jobname}
\generate{
  \file{README.md}{\from{\jobname.dtx}{readme}}
  \file{\jobname.bib}{\from{\jobname.dtx}{bib}}
}

\preamble
----------------------------------------------------------------
biblatex-bath --- Harvard referencing style as recommended by the University of Bath Library
Author:  Alex Ball
E-mail:  a.j.ball@bath.ac.uk
License: Released under the LaTeX Project Public License v1.3c or later
See:     http://www.latex-project.org/lppl.txt
----------------------------------------------------------------

\endpreamble
\postamble

Copyright (C) 2016 by University of Bath
\endpostamble

\usedir{tex/latex/\jobname}
\generate{
  \file{bath.bbx}{\from{\jobname.dtx}{bbx}}
  \file{bath.cbx}{\from{\jobname.dtx}{cbx}}
  \file{bath.dbx}{\from{\jobname.dtx}{dbx}}
  \file{british-bath.lbx}{\from{\jobname.dtx}{lbx}}
}
%</install>
%<install>\endbatchfile
%<*internal>
\usedir{source/latex/\jobname}
\generate{
  \file{\jobname.ins}{\from{\jobname.dtx}{install}}
}
\ifx\fmtname\nameofplainTeX
  \expandafter\endbatchfile
\else
  \expandafter\endgroup
\fi
%</internal>
%<*driver>
\ProvidesFile{biblatex-bath.dtx}
    [2016/09/30 v0.1 Harvard referencing style as recommended by the University of Bath Library]

\documentclass[10pt,a4paper]{article}
\usepackage[british]{babel}
\usepackage[hmargin=3cm,vmargin=2.5cm]{geometry}

\usepackage{iftex}
\ifPDFTeX
  \usepackage{CJKutf8}
\else
  \ifLuaTeX
    \usepackage{luatexja-fontspec}
    \setmainjfont{TakaoPGothic}
  \else
    \ifXeTeX
    \usepackage{ctex}
    \fi
  \fi
\fi

\usepackage{xpatch,csquotes,xcolor,xparse,multicol,fancyvrb}
\xdefinecolor{Green}{rgb}{0,.5,0}
\xdefinecolor{Slate}{RGB}{80,86,94}
\xdefinecolor{BathStone}{RGB}{213,211,185}
\colorlet{ok}{Green}
\colorlet{bad}{red}
\colorlet{hacked}{orange}
\colorlet{manual}{purple}

\usepackage[tightLists=false]{markdown}
\markdownSetup{rendererPrototypes={%
    link = {\href{#3}{#1}}%
}}

\usepackage{fontawesome}[2015/07/07]
\newcommand{\booksym}{\makebox[1em][c]{\faicon{book}}}
\newcommand{\cogsym}{\makebox[1em][c]{\faicon{cog}}}
\makeatletter
\newcommand{\hangfrom}[1]{%
  \setbox\@tempboxa\hbox{{#1}}%
  \hangindent \wd\@tempboxa\noindent\box\@tempboxa}
\makeatother

\usepackage{tcolorbox}
\tcbuselibrary{listings,breakable,skins,xparse}
\lstloadlanguages{[LaTeX]TeX}
\lstdefinestyle{dtxlatex}%
  { columns=fullflexible
  , basicstyle=\ttfamily
  , language={[LaTeX]TeX}
  , texcsstyle=*\color{red!75!black}
  , moretexcs=
    { printbibliography
    , textcite
    , autocite
    , noop
    , addbibresource
    }
  , moredelim=**[s][\color{violet}]{[}{]}
  , moredelim=**[s][\color{blue!75!black}]{\{}{\}}
  , mathescape
  , escapechar=~
  }
\lstset{style=dtxlatex}
\tcbset
  { colframe = Slate
  , colback = BathStone!25
  , listing options =
    { style = tcblatex
    , style = dtxlatex
    , basicstyle=\ttfamily\small
    }
  }
\NewTColorBox{bibexbox}{D(){ok}d<>m}%
  {bicolor
  ,colframe = #1
  ,colback = #1!5!white
  ,colbacklower = white
  ,fontlower = \footnotesize
  ,before upper = {\hangfrom{\booksym\space}}
  ,after upper = {\par\hangfrom{\cogsym\space}\fullcite{#3}}
  ,IfNoValueTF={#2}{}%
    {overlay = {
      \node[anchor=south east,text=teal] at (frame.south east) {#2};
      }
    }
  }

\usepackage[backend=biber,bibencoding=utf8,hyperref=false,isbn=false,style=bath]{biblatex}
\addbibresource{biblatex-bath.bib}
\makeatletter
\DeclareCiteCommand{\fullcite}
  {\usebibmacro{prenote}}
  {\usedriver
     {\defcounter{maxnames}{\blx@maxbibnames}}
     {\thefield{entrytype}}.}
  {\multicitedelim}
  {\usebibmacro{postnote}}
\makeatother

\usepackage{readprov}
\usepackage[british,cleanlook]{isodate}

\usepackage[colorlinks,citecolor=black]{hyperref}

\sloppy

\title{biblatex-bath: Harvard referencing style as recommended by the University of Bath Library}
\author{%
  Maintainer: Alex Ball\thanks{%
    To contact the maintainer about this package, please visit the repository
    where the code is hosted: \url{https://github.bath.ac.uk/ab318/bathbib}.%
  }%
}
\date{Package \UseVersionOf{biblatex-bath.dtx} -- \printdateTeX{\UseDateOf{biblatex-bath.dtx}}}

\begin{document}
\maketitle

\section{Introduction}

\begin{markdown*}{hybrid=true}
%</driver>
%<*driver|readme>

This package provides a biblatex style to format reference lists in the
[Harvard style][bath-harvard] recommended by the University of Bath Library.

## Installation

### Automated way

A makefile is provided which you can use with the Make utility on
UNIX-like systems:

  * Running `make source` generates the derived files
      - README.md
      - bath.bbx
      - british-bath.lbx
      - biblatex-bath.bib
      - biblatex-bath.ins
  * Running `make` generates the above files and also biblatex-bath.pdf.
  * Running `make inst` installs the files in the user's TeX tree.
    You can undo this with `make uninst`.
  * Running `make install` installs the files in the local TeX tree.
    You can undo this with `make uninstall`.

### Manual way

You only need to follow the first two steps if you have made your own
changes to the .dtx file. The compiled files you need are included in
the zip archive.

 1. Run `tex biblatex-bath.dtx` to generate the source files.
 2. Compile biblatex-bath.dtx with LuaLaTeX and Biber to generate the
    documentation. You will need, among other things, the luatexja,
    adobemapping and ipaex packages installed; this is just for the
    documentation, not for the biblatex style itself.
 3. Move the files to your TeX tree as follows:
      - `source/latex/biblatex-bath`:
        biblatex-bath.dtx,
        (biblatex-bath.ins)
      - `tex/latex/biblatex-bath`:
        bath.bbx,
        british-bath.lbx
      - `doc/latex/biblatex-bath`:
        biblatex-bath.pdf,
        README.md

 4. You may then have to update your installation's file name database
    before TeX and friends can see the files.

[bath-harvard]: http://www.bath.ac.uk/library/infoskills/referencing-plagiarism/harvard-bath-style.html
%</driver|readme>
%<*driver>
\end{markdown*}

\section{Using the style}

To use the style, specify it when you load \textsf{biblatex}:

\begin{tcblisting}{listing only}
\usepackage[bibstyle=bath,citestyle=authoryear]{biblatex}
\end{tcblisting}

Remember also to specify your .bib file in the preamble:

\begin{tcblisting}{listing only}
\addbibresource{file.bib}
\end{tcblisting}

Use \lstinline[style=dtxlatex]|\printbibliography| to print your bibliography
towards the end of your document.

To make a citation in the text, use the key that corresponds to the entry in your .bib file:

\begin{tcblisting}{}
While collections can be supplemented by other means \autocite{williams1997edd},
the absence of an invisible collection amongst historians is noted by
\textcite[p.556]{stieg1981inh}. It may be, as \textcite{burchard1965hhl} points
out, that they have no assistants or are reluctant to delegate\dots
\end{tcblisting}

Please refer to the documentation for \href{http://www.ctan.org/pkg/biblatex}{\textsf{biblatex}}
for the full range of commands available for in-text citations.


\section{Examples}

The examples below are shown in three parts.
The first, marked with \faBook, shows an extract from the Harvard (Bath) referencing style sheet.
The second, marked with \faCog, shows the reference as formatted by \textsf{biblatex}.
The last shows how the reference was entered in the .bib file.

Some examples are highlighted in
\tcbox[colframe=hacked,colback=hacked!5!white,nobeforeafter,size=fbox,tcbox raise base]{orange}.
This indicates that some fields have been `abused' to achieve the right effect;
in other words, they contain information that does not conform with their intended use.
Particular care should be taken with such items when switching between different styles,
though of course any item might need adjustment to take account of differing conventions.

\subsection{Book with author(s)}

\begin{bibexbox}{rang.etal2012rdp}
  Rang, H.P., Dale, M.M. Ritter, J.M., Flower, R.J. and Henderson, G., 2012. \emph{Rang and Dale's pharmacology}. 7th ed. Edinburgh:\@ Elsevier Churchill Livingstone.
  \tcblower
\begin{Verbatim}[commentchar=\%]
%</driver>
%<*driver|bib>
@book{rang.etal2012rdp,
  author = {Rang, H. P. and Dale, M. M. and Ritter, J. M. and Flower, R. J. and Henderson, G.},
  year = {2012},
  title = {Rang and {Dale's} Pharmacology},
  edition = {7},
  address = {Edinburgh},
  publisher = {Elsevier Churchill Livingstone}}
%</driver|bib>
%<*driver>
\end{Verbatim}
\end{bibexbox}

\begin{bibexbox}{ou1972em}
  Open University, 1972. \emph{Electricity and magnetism}. Bletchley:\@ Open University Press.
  \tcblower
\begin{Verbatim}[commentchar=\%]
%</driver>
%<*driver|bib>
@book{ou1972em,
  author = {{Open University}},
  year = {1972},
  title = {Electricity and Magnetism},
  address = {Bletchley},
  publisher = {Open University Press}}
%</driver|bib>
%<*driver>
\end{Verbatim}
\end{bibexbox}


\subsection{Book with editor(s) instead of author(s)}

\begin{bibexbox}{rothman.etal2008me}
  Rothman, K.J., Greenland, S. and Lash, T.L., eds., 2008. \emph{Modern epidemiology}. 3rd ed. Philadelphia, Pa.:\@ Lippincott Williams \& Wilkins.
  \tcblower
\begin{Verbatim}[commentchar=\%]
%</driver>
%<*driver|bib>
@book{rothman.etal2008me,
  editor = {Kenneth J. Rothman and Sander Greenland and Timothy L. Lash},
  year = {2008},
  title = {Modern Epidemiology},
  edition = {3},
  address = {Philadelphia, Pa.},
  publisher = {Lippincott Williams \& Wilkins}}
%</driver|bib>
%<*driver>
\end{Verbatim}
\end{bibexbox}


\subsection{Book usually known by its title}

\begin{bibexbox}{oed1989}
  \emph{Oxford English dictionary}, 1989. 2nd ed. Oxford:\@ Clarendon Press.
  \tcblower
\begin{Verbatim}[commentchar=\%]
%</driver>
%<*driver|bib>
@book{oed1989,
  year = {1989},
  title = {Oxford {English} Dictionary},
  edition = {2},
  address = {Oxford},
  publisher = {Clarendon Press}}
%</driver|bib>
%<*driver>
\end{Verbatim}
\end{bibexbox}

\begin{bibexbox}{merckNDidx}
  \emph{The Merck index online} [Online]. London:\@ RSC Publishing. Available from:\@ \url{http://www.rsc.org/Merck-Index} [Accessed 16/06/2016].
  \tcblower
\begin{Verbatim}[commentchar=\%]
%</driver>
%<*driver|bib>
@book{merckNDidx,
  title = {The {Merck} Index Online},
  address = {London},
  publisher = {RSC Publishing},
  url = {http://www.rsc.org/Merck-Index},
  urldate = {2016-06-16}}
%</driver|bib>
%<*driver>
\end{Verbatim}
\end{bibexbox}


\subsection{One chapter\slash paper from a collection in a book}

\begin{bibexbox}{burchard1965hhl}
  Burchard, J.E., 1965. How humanists use a library. In: C.F.J. Overhage and J.R. Harman, eds. \emph{Intrex: report on a planning conference and information transfer experiments}. Cambridge, Mass.: MIT Press, pp.41--87.
  \tcblower
\begin{Verbatim}[commentchar=\%]
%</driver>
%<*driver|bib>
@incollection{burchard1965hhl,
  author = {Burchard, J. E.},
  year = {1965},
  title = {How Humanists use a Library},
  editor = {C. F. J. Overhage and J. R. Harman},
  booktitle = {Intrex: report on a planning conference and information transfer experiments},
  address = {Cambridge, Mass.},
  publisher = {MIT Press},
  pages = {41-87}}
%</driver|bib>
%<*driver>
\end{Verbatim}
\end{bibexbox}

\begin{bibexbox}{reid1967ptp}
  Reid, D.R., 1967.\@ Physical testing of polymer films.\@ In:\@ S.H. Pinner, ed.\@ \emph{Modern packaging films}. London:\@ Butterworths, pp.143--183.
  \tcblower
\begin{Verbatim}[commentchar=\%]
%</driver>
%<*driver|bib>
@incollection{reid1967ptp,
  author = {D. R. Reid},
  year = {1967},
  title = {Physical Testing of Polymer Films},
  editor = {S. H. Pinner},
  booktitle = {Modern Packaging Films},
  address = {London},
  publisher = {Butterworths},
  pages = {143-183}}
%</driver|bib>
%<*driver>
\end{Verbatim}
\end{bibexbox}


\subsection{Electronic book}

\begin{bibexbox}{haynes2014crc}
  Haynes, W.M., ed.\@, 2014.\@ \emph{CRC handbook of chemistry and physics} [Online].\@ 94th ed. Boca Raton, Fla.:\@ CRC Press/Taylor and Francis. Available from:\@ \url{http://www.hbcpnetbase.com} [Accessed 16/06/2016].
  \tcblower
\begin{Verbatim}[commentchar=\%]
%</driver>
%<*driver|bib>
@book{haynes2014crc,
  editor = {Haynes, W. M.},
  year = {2014},
  title = {{CRC} Handbook of Chemistry and Physics},
  edition = {94},
  address = {Boca Raton, Fla.},
  publisher = {CRC Press/Taylor and Francis},
  url = {http://www.hbcpnetbase.com},
  urldate = {2016-06-16}}
%</driver|bib>
%<*driver>
\end{Verbatim}
\end{bibexbox}


\subsection{Journal article}

\begin{bibexbox}{stieg1981cer}
  Stieg, M.F., 1981a. Continuing education and the reference librarian in the academic and research library. \emph{Library Journal}, 105(22), pp.2547--2551.
  \tcblower
\begin{Verbatim}[commentchar=\%]
%</driver>
%<*driver|bib>
@article{stieg1981cer,
  author = {Stieg, M. F.},
  year = {1981},
  title = {Continuing Education and the Reference Librarian in the Academic and Research Library},
  journal = {Library Journal},
  volume = {105},
  number ={22},
  pages = {2547-2551}}
%</driver|bib>
%<*driver>
\end{Verbatim}
\end{bibexbox}

\begin{bibexbox}{stieg1981inh}
  Stieg, M.F., 1981b. The information needs of historians. \emph{College and research libraries}, 42(6), pp.549--560.
  \tcblower
\begin{Verbatim}[commentchar=\%]
%</driver>
%<*driver|bib>
@article{stieg1981inh,
  author = {Stieg, M. F.},
  year = {1981},
  title = {The Information Needs of Historians},
  journal = {College and Research Libraries},
  volume = {42},
  number ={6},
  pages = {549-560}}
%</driver|bib>
%<*driver>
\end{Verbatim}
\end{bibexbox}

\begin{bibexbox}{newman2010mcb}
  Newman, R., 2010. Malaria control beyond 2010. \emph{Brit.\@ Med.\@ J.}, 341(7765), pp.157--208.
  \tcblower
\begin{Verbatim}[commentchar=\%]
%</driver>
%<*driver|bib>
@article{newman2010mcb,
  author = {Newman, R.},
  year = {2010},
  title = {Malaria control beyond 2010},
  journal = {Brit.\@ Med.\@ J.},
  volume = {341},
  number = {7765},
  pages = {157-208}}
%</driver|bib>
%<*driver>
\end{Verbatim}
\end{bibexbox}


\subsection{E-journal article}

\begin{bibexbox}{williams1997edd}
  Williams, F., 1997. Electronic document delivery:\@ a trial in an academic library. \emph{Ariadne} [Online], 10. Available from:\@ \url{http://www.ariadne.ac.uk/issue10/edd/} [Accessed 05/12/1997].
  \tcblower
\begin{Verbatim}[commentchar=\%]
%</driver>
%<*driver|bib>
@article{williams1997edd,
  author = {Williams, F.},
  year = {1997},
  title = {Electronic Document Delivery: a Trial in an Academic Library},
  journal = {Ariadne},
  volume = {10},
  url = {http://www.ariadne.ac.uk/issue10/edd/},
  urldate = {1997-12-05}}
%</driver|bib>
%<*driver>
\end{Verbatim}
\end{bibexbox}


\subsection{Websites}

\begin{bibexbox}{holland2002gci}
  Holland, M., 2002. \emph{Guide to citing internet sources} [Online]. Poole:\@ Bournemouth University. Available from:\@ \url{http://www.bournemouth.ac.uk/library/using/guide_to_citing_internet_sourc.html} [Accessed 04/11/2002].
  \tcblower
\begin{Verbatim}[commentchar=\%]
%</driver>
%<*driver|bib>
@online{holland2002gci,
  author = {Holland, M.},
  year = {2002},
  title = {Guide to Citing Internet Sources},
  address = {Poole},
  organization = {Bournemouth University},
  url = {http://www.bournemouth.ac.uk/library/using/guide_to_citing_internet_sourc.html},
  urldate = {2002-11-04}}
%</driver|bib>
%<*driver>
\end{Verbatim}
\end{bibexbox}

\begin{bibexbox}{wiltshire2015gww}
  Wiltshire Council, 2015. \emph{Get Wiltshire walking} [Online]. Trowbridge:\@ Wiltshire Council. Available from:\@ \url{http://www.wiltshire.gov.uk/leisureandrecreation/sportphysicalactivity/getwiltshirewalking.html} [Accessed 19/08/2015].
  \tcblower
\begin{Verbatim}[commentchar=\%]
%</driver>
%<*driver|bib>
@online{wiltshire2015gww,
  author = {{Wiltshire Council}},
  year = {2015},
  title = {Get {Wiltshire} walking},
  address = {Trowbridge},
  organization = {Wiltshire Council},
  url = {http://www.wiltshire.gov.uk/leisureandrecreation/sportphysicalactivity/getwiltshirewalking.html},
  urldate = {2015-08-19}}
%</driver|bib>
%<*driver>
\end{Verbatim}
\end{bibexbox}


\subsection{Conference paper (when proceedings have a named editor)}

\begin{bibexbox}{crawford1965oim}
  Crawford, G.I., 1965. Oxygen in metals. In:\@ J.M.A. Lenihan AND S.J. Thompson, eds. \emph{Activation analysis:\@ proceedings of a NATO Advanced Study Institute}, 2--4 August 1964 Glasgow. London:\@ Academic Press, pp.113--118.
  \tcblower
\begin{Verbatim}[commentchar=\%]
%</driver>
%<*driver|bib>
@inproceedings{crawford1965oim,
  author = {Crawford, G. I.},
  year = {1965},
  title = {Oxygen in Metals},
  editor = {J. M. A. Lenihan and S. J. Thompson},
  booktitle = {Activation Analysis: Proceedings of a {NATO} {Advanced} {Study} {Institute}},
  eventdate = {1964-08-02/1964-08-04},
  venue = {Glasgow},
  address = {London},
  publisher = {Academic Press},
  pages = {113-118}}
%</driver|bib>
%<*driver>
\end{Verbatim}
\end{bibexbox}


\subsection{Conference paper (when proceedings have no named editor or are part of a major series)}

\begin{bibexbox}{soper1972rbc}
  Soper, D., 1972. Review of bracken control experiments with asulam. \emph{Proceedings of the 11th British Weed Control Conference}, 15--17 November 1972 Brighton. Brighton:\@ University of Sussex, pp.24--31.
  \tcblower
\begin{Verbatim}[commentchar=\%]
%</driver>
%<*driver|bib>
@inproceedings{soper1972rbc,
  author = {Soper, D.},
  year = {1972},
  title = {Review of Bracken Control Experiments with Asulam},
  booktitle = {Proceedings of the 11th {British} {Weed} {Control} {Conference}},
  eventdate = {1972-11-15/1972-11-17},
  venue = {Brighton},
  address = {Brighton},
  publisher = {University of Sussex},
  pages = {24-31}}
%</driver|bib>
%<*driver>
\end{Verbatim}
\end{bibexbox}


\subsection{Newspaper article}

\begin{bibexbox}{haurant2004bbh}
  Haurant, S., 2004. Britain's borrowing hits £1 trillion. \emph{The Guardian}, 29 July, p.16c.
  \tcblower
\begin{Verbatim}[commentchar=\%]
%</driver>
%<*driver|bib>
@article{haurant2004bbh,
  author = {Haurant, S.},
  date = {2004-07-29},
  title = {Britain's Borrowing Hits \pounds 1 Trillion},
  journal = {The Guardian},
  pages = {16c}}
%</driver|bib>
%<*driver>
\end{Verbatim}
\end{bibexbox}

\begin{bibexbox}(hacked){independent1992pub}
  The Independent, 1992. Picking up the bills. \emph{The Independent}, 4 June, p.28a.
  \tcblower
\begin{Verbatim}[commentchar=\%]
%</driver>
%<*driver|bib>
@article{independent1992pub,
  author = {{The Independent}},
  date = {1992-06-04},
  title = {Picking Up the Bills},
  journal = {The Independent},
  pages = {28a}}
%</driver|bib>
%<*driver>
\end{Verbatim}
\end{bibexbox}


\subsection{Thesis/dissertation}

\begin{bibexbox}{burrell1973ist}
  Burrell, J.G., 1973. \emph{The importance of school tours in education}. Thesis (M.A.). Queen's University, Belfast.
  \tcblower
\begin{Verbatim}[commentchar=\%]
%</driver>
%<*driver|bib>
@thesis{burrell1973ist,
  author = {Burrell, J. G.},
  year = {1973},
  title = {The Importance of School Tours in Education},
  type = {Thesis ({M.A.})},
  institution = {Queen's University, Belfast}}
%</driver|bib>
%<*driver>
\end{Verbatim}
\end{bibexbox}


\subsection{Report}

\begin{bibexbox}{unesco1993gip}
  UNESCO, 1993. \emph{General information programme and UNISIST}\@. Paris:\@ UNESCO, (PGI-93/WS/22).
  \tcblower
\begin{Verbatim}[commentchar=\%]
%</driver>
%<*driver|bib>
@report{unesco1993gip,
  author = {{UNESCO}},
  year = {1993},
  title = {General Information Programme and {UNISIST}},
  address = {Paris},
  institution = {UNESCO},
  number = {PGI-93/WS/22}}
%</driver|bib>
%<*driver>
\end{Verbatim}
\end{bibexbox}


\subsection{Standard}

\begin{bibexbox}(hacked){bs5605:1990}
  BS 5605:1990. \emph{Recommendations for citing and referencing published material}. BSI.
  \tcblower
\begin{Verbatim}[commentchar=\%]
%</driver>
%<*driver|bib>
@standard{bs5605:1990,
  author = {{BS 5605:1990}},
  title = {Recommendations for citing and referencing published material},
  organization = {BSI}}
%</driver|bib>
%<*driver>
\end{Verbatim}
\end{bibexbox}


\subsection{Patent}

\begin{bibexbox}{pm1981opa}
  Phillipp Morris Inc., 1981. \emph{Optical perforating apparatus and system}. European patent application 0021165A1. 1981-01-07.
  \tcblower
\begin{Verbatim}[commentchar=\%]
%</driver>
%<*driver|bib>
@patent{pm1981opa,
  author = {{Phillipp Morris Inc.}},
  year = {1981},
  title = {Optical perforating apparatus and system},
  type = {European patent application},
  number = {0021165A1. 1981-01-07}}
%</driver|bib>
%<*driver>
\end{Verbatim}
\end{bibexbox}


\subsection{Map}

\begin{bibexbox}{andrews.dury1773wilts}
  Andrews, J. and Dury, A., 1773. \emph{Map of Wiltshire}, 1 inch to 2 miles. Devizes:\@ Wiltshire Record Society.
  \tcblower
\begin{Verbatim}[commentchar=\%]
%</driver>
%<*driver|bib>
@book{andrews.dury1773wilts,
  author = {Andrews, J. and Dury, A.},
  year = {1773},
  title = {Map of {Wiltshire}},
  series = {1 inch to 2 miles},
  address = {Devizes},
  publisher = {Wiltshire Record Society}}
%</driver|bib>
%<*driver>
\end{Verbatim}
\end{bibexbox}


\subsection{Film, video or DVD}

\begin{bibexbox}{macbeth1948}
  \emph{Macbeth}, 1948. Film.\@ Directed by Orson Welles. USA:\@ Republic Pictures.
  \tcblower
\begin{Verbatim}[commentchar=\%]
%</driver>
%<*driver|bib>
@video{macbeth1948,
  year = {1948},
  title = {Macbeth},
  howpublished = {Film. Directed by Orson Welles},
  address = {USA},
  publisher = {Republic Pictures}}
%</driver|bib>
%<*driver>
\end{Verbatim}
\end{bibexbox}

You can also use \texttt{movie} as an alias for \texttt{video}.


\subsection{Streamed video}

\begin{bibexbox}{moran2016sol}
  Moran, C., 2016. \emph{Save Our Libraries} [Online]. Available from:\@ \url{https://www.youtube.com/watch?v=gKTfCz4JtVE&feature=youtu.be} [Accessed 29/04/2016]
  \tcblower
\begin{Verbatim}[commentchar=\%]
%</driver>
%<*driver|bib>
@online{moran2016sol,
  author = {Moran, C.},
  year = {2016},
  title = {Save Our Libraries},
  url = {https://www.youtube.com/watch?v=gKTfCz4JtVE&feature=youtu.be},
  urldate = {2016-04-29}}
%</driver|bib>
%<*driver>
\end{Verbatim}
\end{bibexbox}


\subsection{Television or radio broadcast}

\begin{bibexbox}{rsfo2006ep5}
  \emph{Rick Stein's French Odyssey:\@ Episode 5}, 2006. TV. BBC2, 23 August.\@ 20.30 hrs.
  \tcblower
\begin{Verbatim}[commentchar=\%]
%</driver>
%<*driver|bib>
@video{rsfo2006ep5,
  year = {2006},
  title = {Rick {Stein's} {French} {Odyssey}: Episode 5},
  howpublished = {TV. BBC2, 23 August. 20.30 hrs}}
%</driver|bib>
%<*driver>
\end{Verbatim}
\end{bibexbox}

\begin{bibexbox}{archers20060823}
  \emph{The Archers}, 2006. Radio.\@ BBC Radio 4, 23 August.\@ 19.02 hrs.
  \tcblower
\begin{Verbatim}[commentchar=\%]
%</driver>
%<*driver|bib>
@audio{archers20060823,
  year = {2006},
  title = {The {Archers}},
  howpublished = {Radio. BBC Radio 4, 23 August. 19.02 hrs}}
%</driver|bib>
%<*driver>
\end{Verbatim}
\end{bibexbox}

You can also use \texttt{music} as an alias for \texttt{audio}.


\subsection{Music score}

\begin{bibexbox}{beethoven1950symph1}
  Beethoven, L. Van, 1950. \emph{Symphony no.1 in C, Op.21}. Harmondsworth:\@ Penguin.
  \tcblower
\begin{Verbatim}[commentchar=\%]
%</driver>
%<*driver|bib>
@book{beethoven1950symph1,
  author = {Ludwig van Beethoven},
  year = {1950},
  title = {Symphony no.1 in {C,} {Op.21}},
  address = {Harmondsworth},
  publisher = {Penguin}}
%</driver|bib>
%<*driver>
\end{Verbatim}
\end{bibexbox}


\subsection{Email discussion lists (jiscmail/listserv etc)}

\begin{bibexbox}(hacked){clark2004euk}
  Clark, T., 5 July 2004. A European UK Libraries Plus? \emph{Lis-link} [Online]. Available from:\@ \url{lis-link@jiscmail.ac.uk} [Accessed 30 July 2004].
  \tcblower
\begin{Verbatim}[commentchar=\%]
%</driver>
%<*driver|bib>
@letter{clark2004euk,
  entrysubtype = {Electronic mailing list message},
  author = {Clark, T.},
  date = {2004-07-05},
  title = {A {European} {UK} {Libraries} {Plus}?},
  journal = {Lis-link},
  url = {lis-link@jiscmail.ac.uk},
  urldate = {2004-07-30}}
%</driver|bib>
%<*driver>
\end{Verbatim}
\end{bibexbox}


\subsection{Personal emails}

\begin{bibexbox}(hacked){alston2004sah}
  Alston, S., 19/07/2004. \emph{Society of Architectural Historians of GB}. Email to K.M. Jordan.
  \tcblower
\begin{Verbatim}[commentchar=\%]
%</driver>
%<*driver|bib>
@letter{alston2004sah,
  author = {Alston, S.},
  date = {2004-04-19},
  title = {Society of {Architectural} {Historians} of {GB}},
  note = {Email to K.M. Jordan}}
%</driver|bib>
%<*driver>
\end{Verbatim}
\end{bibexbox}


\subsection{Preprint}

\begin{bibexbox}{shah.corrick2016hsc}
  Shah, I. and Corrick, I. 2016. \emph{How should central banks respond to non-neutral inflation expectations?} Bath:\@ University of Bath. \emph{OPUS} [Online]. Available from:\@ \url{http://opus.bath.ac.uk} [Accessed 04/05/2016].
  \tcblower
\begin{Verbatim}[commentchar=\%]
%</driver>
%<*driver|bib>
@report{shah.corrick2016hsc,
  author = {Shah, I. and Corrick, I.},
  year = {2016},
  title = {How should central banks respond to non-neutral inflation expectations?},
  address = {Bath},
  institution = {University of Bath},
  note = {\emph{OPUS} [Online]},
  url = {http://opus.bath.ac.uk},
  urldate = {2016-05-04}}
%</driver|bib>
%<*driver>
\end{Verbatim}
\end{bibexbox}


\subsection{Work in translation}

\begin{bibexbox}{aristotle2007ne}
  Aristotle, 2007. \emph{Nicomachean ethics} (W.D. Ross. Trans.) South Dakota:\@ NuVisions.
  \tcblower
\begin{Verbatim}[commentchar=\%]
%</driver>
%<*driver|bib>
@book{aristotle2007ne,
  author = {Aristotle},
  year = {2007},
  title = {Nicomachean Ethics},
  translator = {W. D. Ross},
  address = {South Dakota},
  publisher = {NuVisions}}
%</driver|bib>
%<*driver>
\end{Verbatim}
\end{bibexbox}


\subsection{Non-English work in the Roman alphabet}

\begin{bibexbox}{esquivel2003cap}
  Esquivel, L., 2003. \emph{Como agua para chocolate} [Like water for chocolate]. Barcelona:\@ Debolsillo.
  \tcblower
\begin{Verbatim}[commentchar=\%]
%</driver>
%<*driver|bib>
@book{esquivel2003cap,
  author = {Esquivel, L.},
  year = {2003},
  title = {Como Agua para Chocolate},
  titleaddon = {[Like water for chocolate]},
  address = {Barcelona},
  publisher = {Debolsillo}}
%</driver|bib>
%<*driver>
\end{Verbatim}
\end{bibexbox}

\begin{bibexbox}{thurfjell1975vhv}
  Thurfjell, W., 1975. Vart har våran doktor tagit vägen? [Where has our doctor gone?] \emph{Läkartidningen} 72, p.789.
  \tcblower
\begin{Verbatim}[commentchar=\%]
%</driver>
%<*driver|bib>
@article{thurfjell1975vhv,
  author = {Thurfjell, W.},
  year = {1975},
  title = {Vart har våran doktor tagit vägen?},
  titleaddon = {[Where has our doctor gone?]},
  journal = {Läkartidningen},
  volume = {72},
  pages = {789}}
%</driver|bib>
%<*driver>
\end{Verbatim}
\end{bibexbox}


\subsection{Non-English works in non-Roman alphabets}

\begin{bibexbox}{hua1999qys1}
  Hua, L. 華林甫, 1999.  Qingdai yilai Sanxia diqu shuihan zaihai de chubu yanjiu 清代以來三峽地區水旱災害的初步硏 [A preliminary study of floods and droughts in the Three Gorges region since the Qing dynasty], \emph{Zhongguo shehui kexue} 中國社會科學 , 1, pp.168--79.
  \tcblower
\begin{Verbatim}[commentchar=\%]
%</driver>
%<*driver|bib>
@article{hua1999qys1,
  author = {given=Linfu, family=Hua, cjk=華林甫},
  year = {1999},
  title = {Qingdai yilai {Sanxia} diqu shuihan zaihai de chubu yanjiu
    {清代以來三峽地區水旱災害的初步硏}},
  titleaddon = {[A preliminary study of floods and droughts in the {Three} {Gorges} region since
    the {Qing} dynasty]},
  journal = {Zhongguo shehui kexue \textup{中國社會科學}},
  volume = {1},
  pages = {168-79}}
%</driver|bib>
%<*driver>
\end{Verbatim}
\end{bibexbox}

\begin{bibexbox}{hua1999qys2}
  Hua, L., 1999. Qingdai yilai Sanxia diqu shuihan zaihai de chubu yanjiu [A preliminary study of floods and droughts in the Three Gorges region since the Qing dynasty], \emph{Zhongguo shehui kexue}, 1, pp.168--79.
  \tcblower
\begin{Verbatim}[commentchar=\%]
%</driver>
%<*driver|bib>
@article{hua1999qys2,
  author = {Hua, Linfu},
  year = {1999},
  title = {Qingdai yilai {Sanxia} diqu shuihan zaihai de chubu yanjiu},
  titleaddon = {[A preliminary study of floods and droughts in the {Three} {Gorges} region since
    the {Qing} dynasty]},
  journal = {Zhongguo shehui kexue},
  volume = {1},
  pages = {168-79}}
%</driver|bib>
%<*driver>
\end{Verbatim}
\end{bibexbox}

Although not a feature of the Bath Harvard Style, if you want to suppress the 
punctuation between the family name and the initial (and thereby be more
faithful to the original orthography), you can specify this using
\textsf{biblatex}'s data annotations feature:

\begin{bibexbox}{hua2001foo}
  Hua L. 華林甫, 2001. \emph{Lorem ipsum}.
  \tcblower
\begin{Verbatim}[commentchar=\%]
%</driver>
%<*driver|bib>
@book{hua2001foo,
  author = {given=Linfu, family=Hua, cjk=華林甫},
  author+an = {1=cjk},
  year = {2001},
  title = {Lorem ipsum}}
%</driver|bib>
%<*driver>
\end{Verbatim}
\end{bibexbox}


\subsection{House of Commons paper}

\begin{bibexbox}{gb.hc2003/04-30}
  Great Britain. Parliament. House of Commons, 2004. \emph{National Savings investment deposits:\@ account 2002--2003}. London:\@ National Audit Office (HC 200 3/04, 30).
  \tcblower
\begin{Verbatim}[commentchar=\%]
%</driver>
%<*driver|bib>
@legislation{gb.hc2003/04-30,
  author = {{Great Britain. Parliament. House of Commons}},
  year = {2004},
  title = {National {Savings} Investment Deposits: account 2002--2003},
  address = {London},
  publisher = {National Audit Office},
  type = {{HC}},
  number = {2003/04, 30}}
%</driver|bib>
%<*driver>
\end{Verbatim}
\end{bibexbox}


\subsection{House of Lords paper}

\begin{bibexbox}{gb.hl1986/87-66}
  Great Britain. Parliament. House of Lords. 1987. \emph{Social fund (Maternity and Funeral Expenses) Bill}. London:\@ HMSO (HL 1986/87, (66)).
  \tcblower
\begin{Verbatim}[commentchar=\%]
%</driver>
%<*driver|bib>
@legislation{gb.hl1986/87-66,
  author = {{Great Britain. Parliament. House of Lords}},
  year = {1987},
  title = {Social Fund ({Maternity} and {Funeral} {Expenses}) Bill},
  address = {London},
  publisher = {HMSO},
  type = {{HL}},
  number = {1986/87, (66)}}
%</driver|bib>
%<*driver>
\end{Verbatim}
\end{bibexbox}


\subsection{House of Commons/House of Lords bill}

\begin{bibexbox}{gb.bill1987/88-66}
  Great Britain. Parliament. House of Commons, 1988. \emph{Local government finance bill}. London:\@ HMSO (Bills | 1987/88, 66).
  \tcblower
\begin{Verbatim}[commentchar=\%]
%</driver>
%<*driver|bib>
@legislation{gb.bill1987/88-66,
  author = {{Great Britain. Parliament. House of Commons}},
  year = {1988},
  title = {Local Government Finance Bill},
  address = {London},
  publisher = {HMSO},
  type = {{Bills |}},
  number = {1987/88, 66}}
%</driver|bib>
%<*driver>
\end{Verbatim}
\end{bibexbox}


\subsection{Act of Parliament (UK Statutes) before 1963}

\begin{bibexbox}{gb.wa1735}
  \emph{Witchcraft Act 1735} (9 Geo.2, c.5).
  \tcblower
\begin{Verbatim}[commentchar=\%]
%</driver>
%<*driver|bib>
@legislation{gb.wa1735,
  sortyear = {1735},
  title = {Witchcraft {Act} 1735},
  number = {9 Geo.2, c.5}}
%</driver|bib>
%<*driver>
\end{Verbatim}
\end{bibexbox}


\subsection{Act of Parliament (UK Statues) 1963 onwards}

\begin{bibexbox}(hacked){gb.pa2014}
  \emph{Pensions Act 2014}, c.19. London:\@ TSO.
  \tcblower
\begin{Verbatim}[commentchar=\%]
%</driver>
%<*driver|bib>
@legislation{gb.pa2014,
  sortyear = {2014},
  title = {Pensions {Act} 2014},
  note = {c.19},
  address = {London},
  publisher = {TSO}}
%</driver|bib>
%<*driver>
\end{Verbatim}
\end{bibexbox}


\subsection{Command paper}

\begin{bibexbox}{gb.cm6041}
  Great Britain. Ministry of Defence, 2004. \emph{Delivering security in a changing world:\@ defence white paper}. London:\@ TSO (Cm. 6041).
  \tcblower
\begin{Verbatim}[commentchar=\%]
%</driver>
%<*driver|bib>
@legislation{gb.cm6041,
  author = {{Great Britain. Ministry of Defence}},
  year = {2004},
  title = {Delivering Security in a Changing World: defence White Paper},
  address = {London},
  publisher = {TSO},
  type = {{Cm.}},
  number = {6041}}
%</driver|bib>
%<*driver>
\end{Verbatim}
\end{bibexbox}


\subsection{Statutory instrument}

\begin{bibexbox}{gb.hmr2012}
  \emph{The Human Medicines Regulations 2012} [Online], No.1916, United Kingdom:\@ HMSO. Available from:\@ \url{http://www.legislation.gov.uk/uksi/2012/1916/pdfs/uksi_20121916_en.pdf} [Accessed 17/04/2016]
  \tcblower
\begin{Verbatim}[commentchar=\%]
%</driver>
%<*driver|bib>
@legislation{gb.hmr2012,
  sortyear = {2012},
  title = {The {Human} {Medicines} {Regulations} 2012},
  note = {No.1916},
  address = {United Kingdom},
  publisher = {HMSO},
  url = {http://www.legislation.gov.uk/uksi/2012/1916/pdfs/uksi_20121916_en.pdf},
  urldate = {2016-04-17}}
%</driver|bib>
%<*driver>
\end{Verbatim}
\end{bibexbox}


\subsection{Legal case study}

\begin{bibexbox}{seldon-v-c.w.j2012}
  \emph{Seldon v Clarkson Wright \& Jakes}. [2012]. UKSC 16.
  \tcblower
\begin{Verbatim}[commentchar=\%]
%</driver>
%<*driver|bib>
@jurisdiction{seldon-v-c.w.j2012,
  title = {Seldon v {Clarkson} {Wright} \& {Jakes}},
  note = {[2012]. UKSC 16}}
%</driver|bib>
%<*driver>
\end{Verbatim}
\end{bibexbox}


\subsection{EU publication}

\begin{bibexbox}{ec2015gra}
  European Commission, 2015. \emph{General report on the activities of the European Union 2014}. Luxembourg:\@ Publications Office of the European Union.
  \tcblower
\begin{Verbatim}[commentchar=\%]
%</driver>
%<*driver|bib>
@report{ec2015gra,
  author = {{European Commission}},
  year = {2015},
  title = {General Report on the Activities of the {European} {Union} 2014},
  address = {Luxembourg},
  publisher = {Publications Office of the European Union}}
%</driver|bib>
%<*driver>
\end{Verbatim}
\end{bibexbox}

\subsection{EU regulation or directive, decision, recommendation or opinion}

\begin{bibexbox}{eu.dir2015/413}
  Directive (EU) 2015/413 of the European Parliament and of the Council of 11th March 2015 facilitating cross-border exchange of information on road-safety-related traffic offences [2015] \emph{OJ} L68/9.
  \tcblower
\begin{Verbatim}[commentchar=\%]
%</driver>
%<*driver|bib>
@jurisdiction{eu.dir2015/413,
  title = {Directive ({EU}) 2015/413 of the {European} {Parliament} and of the {Council} of
    11th {March} 2015 Facilitating Cross-Border Exchange of Information on Road-Safety-Related
    Traffic Offences},
  titleaddon = {[2015] \emph{OJ} L68/9}}
%</driver|bib>
%<*driver>
\end{Verbatim}
\end{bibexbox}


\subsection{Judgment of the European Court of Justice}

\begin{bibexbox}{srl.etal-v-comm2005}
  \emph{Alessandrini Srl and others v.~Commission} (C-295/03 P) [2005] ECR I-5700.
  \tcblower
\begin{Verbatim}[commentchar=\%]
%</driver>
%<*driver|bib>
@jurisdiction{srl.etal-v-comm2005,
  title = {Alessandrini {Srl} and others v.\@ {Commission}},
  titleaddon = {(C-295/03 P) [2005] ECR I-5700}}
%</driver|bib>
%<*driver>
\end{Verbatim}
\end{bibexbox}


\subsection{Database}

\begin{bibexbox}{bvd2008bt}
  Bureau van Dijk, 2008. \emph{BT Group plc company report}. \emph{FAME} [Online]. London:\@ Bureau van Dijk. Available from:\@ \url{http://www.portal.euromonitor.com} [Accessed 6/11/2014].
  \tcblower
\begin{Verbatim}[commentchar=\%]
%</driver>
%<*driver|bib>
@online{bvd2008bt,
  author = {{Bureau van Dijk}},
  year = {2008},
  title = {{BT} {Group} PLC Company Report},
  series = {\emph{FAME} [Online]},
  address = {London},
  organization = {Bureau van Dijk},
  url = {http://www.portal.euromonitor.com},
  urldate = {2014-11-06}}
%</driver|bib>
%<*driver>
\end{Verbatim}
\end{bibexbox}


\subsection{Dataset}

\begin{bibexbox}{wilson2013rgc}
  Wilson, D., 2013. \emph{Real geometry and connectedness via triangular description:\@ CAD example bank} [Online]. Bath:\@ University of Bath. Available from:\@ \url{http://doi.org/10.15125/BATH-00069} [Accessed 20/04/2016].
  \tcblower
\begin{Verbatim}[commentchar=\%]
%</driver>
%<*driver|bib>
@online{wilson2013rgc,
  author = {Wilson, D.},
  year = {2013},
  title = {Real Geometry and Connectedness via Triangular Description: {CAD} Example Bank},
  address = {Bath},
  organization = {University of Bath},
  doi = {10.15125/BATH-00069},
  urldate = {2016-04-20}}
%</driver|bib>
%<*driver>
\end{Verbatim}
\end{bibexbox}

You can also use \texttt{dataset} as an alternative alias for \texttt{online}.


\subsection{Computer program}

\begin{bibexbox}{screencasto}
  @screencasto. \emph{Screencast-O-Matic} (v.2) [computer program]. \url{https://screencast-o-matic.com/} [Accessed 16/05/2016].
  \tcblower
\begin{Verbatim}[commentchar=\%]
%</driver>
%<*driver|bib>
@software{screencasto,
  author = {@screencasto},
  title = {{Screencast-O-Matic}},
  titleaddon = {(v.2) [computer program]},
  url = {https://screencast-o-matic.com/},
  urldate = {2016-05-16}}
%</driver|bib>
%<*driver>
\end{Verbatim}
\end{bibexbox}

\printbibliography

\section*{Licence}

\begin{markdown*}{hybrid=true}
%</driver>
%<readme>
%<readme>## Licence
%<readme>
%<*driver|readme>
Copyright 2016 University of Bath.

This work consists of the documented LaTeX file biblatex-bath.dtx and a Makefile.

The text files contained in this work may be distributed and/or modified
under the conditions of the [LaTeX Project Public License (LPPL)][lppl],
either version 1.3c of this license or (at your option) any later
version.

This work is `maintained' (as per LPPL maintenance status) by [Alex Ball][me].

[lppl]: http://www.latex-project.org/lppl.txt "LaTeX Project Public License (LPPL)"
[me]: https://github.bath.ac.uk/ab318/bathbib "Alex Ball"
%</driver|readme>
%<*driver>
\end{markdown*}

\end{document}
%</driver>
% \iffalse
%<*bbx>
% \fi
%
% \section{Implementation: bibliography style}
%
% \subsection{Preliminaries}
%
% For ease of maintenance, we will patch some definitions with \textsf{xpatch}
% instead of writing out our own in full.
%
%    \begin{macrocode}
\RequirePackage{xpatch}
%    \end{macrocode}
%
% Language support may be widened in future, but for now we support the
% following:
%
%    \begin{macrocode}
\DeclareLanguageMapping{english}{british-bath}
\DeclareLanguageMapping{british}{british-bath}
\DeclareLanguageMapping{american}{british-bath}
%    \end{macrocode}
%
% We begin by loading the default author--year style.
%
%    \begin{macrocode}
\RequireBibliographyStyle{authoryear}
\ExecuteBibliographyOptions{maxcitenames=3,maxbibnames=9999,isbn=false,giveninits=true,dashed=false}
%    \end{macrocode}
%
% We provide some additional bibliography strings.
%
%    \begin{macrocode}
\NewBibliographyString{online}
%    \end{macrocode}
%
% We allow the bibliography look more like the Bib\TeX\ default.
%
%    \begin{macrocode}
\setlength{\bibitemsep}{1em plus 0.2em minus 0.2em}
\renewcommand*{\bibfont}{\normalfont\normalsize}

%    \end{macrocode}
%
% \subsection{Name handling}
%
% Names are usually reversed. There are no spaces between initials.
%
%    \begin{macrocode}
\DeclareNameAlias{author}{family-given}
\DeclareNameAlias{editor}{family-given}
\renewcommand*{\bibinitdelim}{}
%    \end{macrocode}
%
% The handling of CJK names is based on code supplied to TeX.sx by user Moewe in
% answer to \href{http://tex.stackexchange.com/a/320738/16293}{question 320738}.
%
% The CJK part is printed after the anglicized name. If the name is also
% annotated as `cjk' (see `Data Annotations' in the \textsf{biblatex} manual),
% it is always printed in family-given order with no intermediate punctuation.
%
%    \begin{macrocode}
\newbibmacro*{name:cjk-given-family}[3]{%
  \ifitemannotation{cjk}{%
    \usebibmacro{name:delim}{#2#1#3}%
    \usebibmacro{name:hook}{#2#1#3}%
    \mkbibnamefamily{#1}\isdot
    \ifdefvoid{#2}{}{\bibnamedelimd\mkbibnamegiven{#2}}%
    \ifdefvoid{#3}{}{\bibnamedelimd\mkbibnamecjk{#3}}%
  }{%
    \usebibmacro{name:delim}{#2#1#3}%
    \usebibmacro{name:hook}{#2#1#3}%
    \ifdefvoid{#2}{}{\mkbibnamegiven{#2}\isdot\bibnamedelimd}%
    \mkbibnamefamily{#1}\isdot
    \ifdefvoid{#3}{}{\bibnamedelimd\mkbibnamecjk{#3}}%
  }%
}
\newbibmacro*{name:cjk-family-given}[3]{%
  \ifitemannotation{cjk}{%
    \usebibmacro{name:delim}{#2#1#3}%
    \usebibmacro{name:hook}{#2#1#3}%
    \mkbibnamefamily{#1}\isdot
    \ifdefvoid{#2}{}{\bibnamedelimd\mkbibnamegiven{#2}}%
    \ifdefvoid{#3}{}{\bibnamedelimd\mkbibnamecjk{#3}}%
  }{%
    \usebibmacro{name:delim}{#1}%
    \usebibmacro{name:hook}{#1}%
    \mkbibnamefamily{#1}\isdot
    \ifboolexpe{%
      test {\ifdefvoid{#2}}
      and
      test {\ifdefvoid{#3}}}
      {}
      {\revsdnamepunct}%
    \ifdefvoid{#2}{}{\bibnamedelimd\mkbibnamegiven{#2}\isdot}%
    \ifdefvoid{#3}{}{\bibnamedelimd\mkbibnamecjk{#3}}%
  }%
}

\DeclareNameFormat{given-family}{%
  \ifdefvoid{\namepartcjk}{%
    \ifgiveninits{%
      \usebibmacro{name:given-family}
        {\namepartfamily}
        {\namepartgiveni}
        {\namepartprefix}
        {\namepartsuffix}%
    }{%
      \usebibmacro{name:given-family}
        {\namepartfamily}
        {\namepartgiven}
        {\namepartprefix}
        {\namepartsuffix}%
    }%
  }{%
    \ifgiveninits{%
      \usebibmacro{name:cjk-given-family}
        {\namepartfamily}
        {\namepartgiveni}
        {\namepartcjk}%
    }{%
      \usebibmacro{name:cjk-given-family}
        {\namepartfamily}
        {\namepartgiven}
        {\namepartcjk}%
    }%
  }%
  \usebibmacro{name:andothers}%
}

\DeclareNameFormat{family-given}{%
  \ifdefvoid{\namepartcjk}{%
    \ifgiveninits{%
      \usebibmacro{name:family-given}
        {\namepartfamily}
        {\namepartgiveni}
        {\namepartprefix}
        {\namepartsuffix}%
    }{%
      \usebibmacro{name:family-given}
        {\namepartfamily}
        {\namepartgiven}
        {\namepartprefix}
        {\namepartsuffix}%
    }%
  }{%
    \ifgiveninits{%
      \usebibmacro{name:cjk-family-given}
        {\namepartfamily}
        {\namepartgiveni}
        {\namepartcjk}%
    }{%
      \usebibmacro{name:cjk-family-given}
        {\namepartfamily}
        {\namepartgiven}
        {\namepartcjk}%
    }%
  }
  \usebibmacro{name:andothers}%
}

%    \end{macrocode}
%
% With collections, editors appear in natural order between `In' and the title,
% followed by `ed.'
%
%    \begin{macrocode}
\renewbibmacro*{byeditor}{%
  \ifnameundef{editor}
    {}
    {\printnames[byeditor]{editor}%
     \setunit{\addcomma\space}%
     \usebibmacro{editorstrg}%
     \clearname{editor}%
     \newunit}%
  \usebibmacro{byeditorx}}

\renewbibmacro*{byeditorx}{%
  \ifnameundef{editora}
    {}
    {\printnames[byeditora]{editora}%
     \setunit{\addcomma\space}%
     \usebibmacro{typestrg}{editora}{editor}%
     \clearname{editora}%
     \newunit}%
  \ifnameundef{editorb}
    {}
    {\printnames[byeditorb]{editorb}%
     \setunit{\addcomma\space}%
     \usebibmacro{typestrg}{editorb}{editor}%
     \clearname{editorb}%
     \newunit}%
  \ifnameundef{editorc}
    {}
    {\printnames[byeditorc]{editorc}%
     \setunit{\addcomma\space}%
     \usebibmacro{typestrg}{editorc}{editor}%
     \clearname{editorc}%
     \newunit}}

\renewbibmacro*{bytranslator}{%
  \ifnameundef{translator}
    {}
    {\printnames[bytranslator]{translator}%
     \newunit
     \bibstring{translator}%
     \clearname{translator}}}

\renewbibmacro*{byeditor+others}{%
  \ifnameundef{editor}
    {}
    {\printnames[byeditor]{editor}%
     \setunit{\addcomma\space}%
     \usebibmacro{editorstrg}%
     \clearname{editor}%
     \newunit}%
  \usebibmacro{byeditorx}%
  \usebibmacro{bytranslator+others}}

\renewbibmacro*{bytranslator+others}{%
  \ifnameundef{translator}
    {}
    {\printnames[bytranslator]{translator}%
     \newunit
     \bibstring{translator}%
     \clearname{translator}%
     \newunit}%
  \usebibmacro{withothers}}

\newbibmacro*{typestrg}[2]{%
  \iffieldundef{#1type}
    {\bibstring{#2}}
    {\ifbibxstring{\thefield{#1type}}
       {\bibstring{\thefield{#1type}}}
       {\printtext{\thefield{#1type}}}}}

%    \end{macrocode}
%
% \subsection{Titles}
%
% Most titles are set in italics, but some are set roman and unquoted.
%
%    \begin{macrocode}
\DeclareFieldFormat{sentencecase}{\MakeSentenceCase*{#1}}
\DeclareFieldFormat{title}{\mkbibemph{#1}}
\DeclareFieldFormat
  [article,inbook,incollection,inproceedings]%
  {title}{#1}
\DeclareFieldFormat
  [patent,thesis,unpublished]%
  {title}{\mkbibemph{#1}}

%    \end{macrocode}
%
% Online resources are clearly tarred and feathered with an `[Online]' label.
% The \texttt{isonline} prints this label if the resource has a URL and does
% nothing otherwise.
%
%    \begin{macrocode}
\newbibmacro*{isonline}{%
  \ifboolexpr{
    test {\iffieldundef{url}}
    and
    test {\iffieldundef{doi}}
  }{}{%
    \bibstring{online}%
  }%
}

%    \end{macrocode}
%
% Titles are converted to sentence case, and the \texttt{titleaddon} field
% follows with no intervening punctuation.
%
%    \begin{macrocode}
\renewbibmacro*{title}{%
  \ifboolexpr{
    test {\iffieldundef{title}}
    and
    test {\iffieldundef{subtitle}}
  }{}{%
    \printtext[title]{%
      \printfield[sentencecase]{title}%
      \setunit{\subtitlepunct}%
      \printfield[sentencecase]{subtitle}%
      \setunit{\space}%
    }%
    \printfield{titleaddon}%
    \ifboolexpr{
      test {\iffieldundef{journaltitle}}
      and
      test {\iffieldundef{booktitle}}
      and
      not test {\ifentrytype{software}}
    }{%
      \setunit*{\space}%
      \usebibmacro{isonline}%
    }{}%
  }%
}

\renewbibmacro*{booktitle}{%
  \ifboolexpr{
    test {\iffieldundef{booktitle}}
    and
    test {\iffieldundef{booksubtitle}}
  }{}{%
    \printtext[booktitle]{%
      \printfield[sentencecase]{booktitle}%
      \setunit{\subtitlepunct}%
      \printfield[sentencecase]{booksubtitle}%
      \setunit{\space}%
    }%
    \printfield{booktitleaddon}
    \setunit*{\space}%
    \usebibmacro{isonline}%
  }%
}

\renewbibmacro*{maintitle}{%
  \ifboolexpr{
    test {\iffieldundef{maintitle}}
    and
    test {\iffieldundef{mainsubtitle}}
  }{}{
    \printtext[maintitle]{%
      \printfield[sentencecase]{maintitle}%
      \setunit{\subtitlepunct}%
      \printfield[sentencecase]{mainsubtitle}%
      \setunit{\space}%
    }%
    \printfield{maintitleaddon}%
  }%
}

\renewcommand*{\subtitlepunct}{\addcolon\space}

\xpatchbibmacro{labeltitle}{%
  \printfield{title}%
}{%
  \usebibmacro{title}\setunit{\addcomma\space}\clearfield{titleaddon}%
}{}{}

%    \end{macrocode}
%
% \subsection{Dates}
%
% Dates are preceded by a comma, and are given bare rather than in parentheses.
%
%    \begin{macrocode}
\xpatchbibmacro{date+extrayear}{%
  \printtext[parens]%
}{%
  \setunit*{\addcomma\space}%
  \printtext%
}{}{}

%    \end{macrocode}
%
% No attempt is made to recover from missing dates in the bibliography.
%
%    \begin{macrocode}
\DeclareLabeldate{%
  \field{date}
  \field{year}
}
%    \end{macrocode}
%
% \subsection{Publishers}
%
% We patch the secondary publication macros so they fall back to using the
% publisher list (if provided) if the intended list is empty.
%
%    \begin{macrocode}
\renewbibmacro*{institution+location+date}{%
  \printlist{location}%
  \iflistundef{institution}{%
    \iflistundef{publisher}{%
      \setunit*{\addcomma\space}%
    }{%
      \setunit*{\addcolon\space}%
      \printlist{publisher}%
    }%
  }{%
    \setunit*{\addcolon\space}%
    \printlist{institution}%
  }%
  \setunit*{\addcomma\space}%
  \usebibmacro{date}%
  \newunit}
\renewbibmacro*{organization+location+date}{%
  \printlist{location}%
  \iflistundef{organization}{%
    \iflistundef{publisher}{%
      \setunit*{\addcomma\space}%
    }{%
      \setunit*{\addcolon\space}%
      \printlist{publisher}%
    }%
  }{%
    \setunit*{\addcolon\space}%
    \printlist{organization}%
  }%
  \setunit*{\addcomma\space}%
  \usebibmacro{date}%
  \newunit}
%    \end{macrocode}
%
% \subsection{Page numbers}
%
%    \begin{macrocode}
\renewcommand*{\ppspace}{}
\DeclareNumChars{ab}
%    \end{macrocode}
%
% \subsection{URLs}
%
% URLs are prefaced by a `from' statement, and the URL date is enclosed in
% brackets rather than parentheses.
%
%    \begin{macrocode}
\DeclareFieldFormat{url}{\bibstring{urlfrom}\addcolon\space\url{#1}}
\DeclareFieldFormat{doi}{\bibstring{urlfrom}\addcolon\space\url{http://dx.doi.org/#1}}
\DeclareFieldFormat{urldate}{\mkbibbrackets{\bibstring{urlseen}\space#1}}
\renewbibmacro*{doi+eprint+url}{%
  \iftoggle{bbx:eprint}
    {\usebibmacro{eprint}}
    {}%
  \newunit\newblock
  \iftoggle{bbx:url}
    {\usebibmacro{url+urldate}}
    {}}
\renewbibmacro*{url}{%
  \iffieldundef{doi}%
    {\printfield{url}}%
    {\printfield{doi}}%
}

%    \end{macrocode}
%
% \subsection{Articles}
%
% Compared with the standard styles, the main difference in the driver is the
% omission of `in'.
%
%    \begin{macrocode}
\xpatchbibdriver{article}{%
  \usebibmacro{in:}\usebibmacro{journal+issuetitle}%
}{%
  \usebibmacro{journal+issuetitle}%
}{}{}
%    \end{macrocode}
%
% The journal title is followed by a comma. The issue number is separated from
% the volume by parentheses rather than a dot.
%
%    \begin{macrocode}
\renewbibmacro*{journal+issuetitle}{%
  \usebibmacro{journal}%
  \setunit*{\space}%
  \usebibmacro{isonline}%
  \setunit*{\addcomma\space}%
  \iffieldundef{series}
    {}
    {\newunit
     \printfield{series}%
     \setunit{\addcomma\space}}%
  \usebibmacro{volume+number+eid}%
  \setunit{\addspace}%
  \usebibmacro{issue+date}%
  \setunit{\addcolon\space}%
  \usebibmacro{issue}%
  \newunit}
\renewbibmacro*{volume+number+eid}{%
  \printfield{volume}%
  \printfield[parens]{number}%
  \setunit{\addcomma\space}%
  \printfield{eid}}

%    \end{macrocode}
%
% \subsection{Works in collections}
%
% Compared with the standard styles, the main difference is that the editors
% precede the booktitle.
%
%    \begin{macrocode}
\xpatchbibdriver{incollection}{%
  \usebibmacro{maintitle+booktitle}%
}{%
  \usebibmacro{byeditor}%
  \newunit\newblock
  \usebibmacro{maintitle+booktitle}%
  \usebibmacro{bytranslator+others}%
}{}{}
\xpatchbibdriver{inproceedings}{%
  \usebibmacro{maintitle+booktitle}%
  \newunit\newblock
  \usebibmacro{event+venue+date}%
  \newunit\newblock
  \usebibmacro{byeditor+others}%
}{%
  \usebibmacro{byeditor}%
  \newunit\newblock
  \usebibmacro{maintitle+booktitle}%
  \usebibmacro{bytranslator+others}%
  \newunit
  \usebibmacro{event+venue+date}%
}{}{}

%    \end{macrocode}
%
% \subsection{Online works}
%
% Compared with the standard styles, the main difference is that the
% organization's address is printed.
%
%    \begin{macrocode}
\xpatchbibdriver{online}{%
  \printlist{organization}%
}{%
  \usebibmacro{organization+location+date}%
}{}{}

%    \end{macrocode}
%
% \subsection{Reports}
%
% Compared with the standard styles, the main difference is that the type and
% number are printed in parentheses after the publisher.
%
%    \begin{macrocode}
\newbibmacro{type+number}{%
  \ifboolexpr{
    test {\iffieldundef{type}}
    and
    test {\iffieldundef{number}}
  }{}{%
    \printtext[parens]{%
      \printfield{type}%
      \setunit*{\addspace}%
      \printfield{number}%
    }%
  }%
}
\DeclareBibliographyDriver{report}{%
  \usebibmacro{bibindex}%
  \usebibmacro{begentry}%
  \usebibmacro{author}%
  \setunit{\printdelim{nametitledelim}}\newblock
  \usebibmacro{title}%
  \newunit
  \printlist{language}%
  \newunit\newblock
  \usebibmacro{byauthor}%
  \newunit\newblock
  \printfield{version}%
  \newunit
  \printfield{note}%
  \newunit\newblock
  \usebibmacro{institution+location+date}%
  \setunit{\addcomma\space}%
  \usebibmacro{type+number}%
  \newunit\newblock
  \usebibmacro{chapter+pages}%
  \newunit
  \printfield{pagetotal}%
  \newunit\newblock
  \iftoggle{bbx:isbn}
    {\printfield{isrn}}
    {}%
  \newunit\newblock
  \usebibmacro{doi+eprint+url}%
  \newunit\newblock
  \usebibmacro{addendum+pubstate}%
  \setunit{\bibpagerefpunct}\newblock
  \usebibmacro{pageref}%
  \newunit\newblock
  \iftoggle{bbx:related}
    {\usebibmacro{related:init}%
     \usebibmacro{related}}
    {}%
  \usebibmacro{finentry}}

%    \end{macrocode}
%
% \subsection{Legislation}
%
% Legislation is mostly formatted like a report, except there is no comma
% between the publisher and the type\slash number.
%
%    \begin{macrocode}
\newbibmacro{type+number}{%
  \ifboolexpr{
    test {\iffieldundef{type}}
    and
    test {\iffieldundef{number}}
  }{}{%
    \printtext[parens]{%
      \printfield{type}%
      \setunit*{\addspace}%
      \printfield{number}%
    }%
  }%
}
\DeclareBibliographyDriver{legislation}{%
  \usebibmacro{bibindex}%
  \usebibmacro{begentry}%
  \usebibmacro{author}%
  \setunit{\printdelim{nametitledelim}}\newblock
  \usebibmacro{title}%
  \newunit
  \printlist{language}%
  \newunit\newblock
  \usebibmacro{byauthor}%
  \newunit\newblock
  \printfield{version}%
  \newunit
  \printfield{note}%
  \newunit\newblock
  \usebibmacro{institution+location+date}%
  \setunit{\space}%
  \usebibmacro{type+number}%
  \newunit\newblock
  \usebibmacro{chapter+pages}%
  \newunit
  \printfield{pagetotal}%
  \newunit\newblock
  \iftoggle{bbx:isbn}
    {\printfield{isrn}}
    {}%
  \newunit\newblock
  \usebibmacro{doi+eprint+url}%
  \newunit\newblock
  \usebibmacro{addendum+pubstate}%
  \setunit{\bibpagerefpunct}\newblock
  \usebibmacro{pageref}%
  \newunit\newblock
  \iftoggle{bbx:related}
    {\usebibmacro{related:init}%
     \usebibmacro{related}}
    {}%
  \usebibmacro{finentry}}

%    \end{macrocode}
%
% \iffalse
%</bbx>
%<*dbx>
% \fi
%
% \section{Implementation: data model}
%
%    \begin{macrocode}
\DeclareDatamodelConstant[type=list]{nameparts}{prefix,family,suffix,given,cjk}
%    \end{macrocode}
% \iffalse
%</dbx>
%<*lbx>
% \fi
%
% \section{Implementation: British English language file}
%
%    \begin{macrocode}
\InheritBibliographyExtras{british}
\DeclareBibliographyExtras{%
  \protected\def\mkbibdatelong#1#2#3{%
    \iffieldundef{#3}
      {}
      {\thefield{#3}%
       \iffieldundef{#2}{}{\nobreakspace}}%
    \iffieldundef{#2}
      {}
      {\mkbibmonth{\thefield{#2}}%
       \iffieldundef{#1}{}{\space}}%
    \iffieldbibstring{#1}
      {\bibstring{\thefield{#1}}}
      {\dateeraprintpre{#1}\stripzeros{\thefield{#1}}}}%
  \protected\def\mkbibdateshort#1#2#3{%
    \iffieldundef{#3}
      {}
      {\mkdayzeros{\thefield{#3}}%
       \iffieldundef{#2}{}{/}}%
    \iffieldundef{#2}
      {}
      {\mkmonthzeros{\thefield{#2}}%
       \iffieldundef{#1}{}{/}}%
    \iffieldbibstring{#1}
      {\bibstring{\thefield{#1}}}
      {\dateeraprintpre{#1}\mkyearzeros{\thefield{#1}}}}%
  }

\DeclareBibliographyStrings{%
  inherit          = {british},
  urlseen          = {{Accessed}{Accessed}},
  online           = {{[Online]}{[Online]}},
  january          = {{January}{January}},
  february         = {{February}{February}},
  march            = {{March}{March}},
  april            = {{April}{April}},
  may              = {{May}{May}},
  june             = {{June}{June}},
  july             = {{July}{July}},
  august           = {{August}{August}},
  september        = {{September}{September}},
  october          = {{October}{October}},
  november         = {{November}{November}},
  december         = {{December}{December}}
}
%    \end{macrocode}
% \iffalse
%</lbx>
%<*cbx>
% \fi
%
% \section{Implementation: citation file}
%
%    \begin{macrocode}
\RequireCitationStyle{authoryear}
%    \end{macrocode}
% \iffalse
%</cbx>
% \fi

