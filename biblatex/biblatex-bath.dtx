% \iffalse meta-comment
%<*internal>
\iffalse
%</internal>
%<*readme>
# biblatex-bath: Harvard referencing style as recommended by the University of Bath Library
%</readme>
%<*internal>
\fi
\def\nameofplainTeX{plain}
\ifx\fmtname\nameofplainTeX\else
  \expandafter\begingroup
\fi
%</internal>
%<*install>
\input docstrip.tex
\keepsilent
\askforoverwritefalse

\nopreamble\nopostamble

\usedir{doc/latex/\jobname}
\generate{
  \file{README.md}{\from{\jobname.dtx}{readme}}
  \file{\jobname.bib}{\from{\jobname.dtx}{bib}}
}

\preamble
----------------------------------------------------------------
biblatex-bath --- Harvard referencing style as recommended by the University of Bath Library
Author:  Alex Ball
E-mail:  a.j.ball@bath.ac.uk
License: Released under the LaTeX Project Public License v1.3c or later
See:     http://www.latex-project.org/lppl.txt
----------------------------------------------------------------

\endpreamble
\postamble

Copyright (C) 2018 by University of Bath
\endpostamble

\usedir{tex/latex/\jobname}
\generate{
  \file{bath.bbx}{\from{\jobname.dtx}{bbx}}
  \file{bath.cbx}{\from{\jobname.dtx}{cbx}}
  \file{bath.dbx}{\from{\jobname.dtx}{dbx}}
  \file{english-bath.lbx}{\from{\jobname.dtx}{lbx}}
  \file{british-bath.lbx}{\from{\jobname.dtx}{lbx-gb}}
}
%</install>
%<install>\endbatchfile
%<*internal>
\usedir{source/latex/\jobname}
\generate{
  \file{\jobname.ins}{\from{\jobname.dtx}{install}}
}
\ifx\fmtname\nameofplainTeX
  \expandafter\endbatchfile
\else
  \expandafter\endgroup
\fi
%</internal>
%<*driver>
\ProvidesFile{biblatex-bath.dtx}
    [2018/10/17 v3 Harvard referencing style as recommended by the University of Bath Library]

\documentclass[10pt,a4paper]{article}
\usepackage[british]{babel}
\usepackage[hmargin=3cm,vmargin=2.5cm]{geometry}
\frenchspacing

% For typesetting the CJK example

\usepackage{iftex}
\ifPDFTeX
  \usepackage{CJKutf8}
\else
  \ifLuaTeX
    \usepackage{luatexja-fontspec}
    \setmainjfont{IPAexGothic}
  \else
    \ifXeTeX
    \usepackage{ctex}
    \fi
  \fi
\fi

% For typesetting the examples

\newcounter{example}
\newcommand{\eg}[1]{\protect\hfill\normalfont\textit{#1}}

\usepackage{xpatch,csquotes,xcolor,xparse,multicol,fancyvrb}
\xdefinecolor{Green}{rgb}{0,.5,0}
\xdefinecolor{Slate}{RGB}{80,86,94}
\xdefinecolor{BathStone}{RGB}{213,211,185}
\colorlet{ok}{Green}
\colorlet{todo}{red}
\colorlet{hacked}{orange}
\colorlet{manual}{purple}
\RecustomVerbatimEnvironment
  {Verbatim}{Verbatim}
  {commentchar=\%}

\usepackage[tightLists=false]{markdown}
\markdownSetup{rendererPrototypes={%
    link = {\href{#3}{#1}}%
}}

\usepackage{fontawesome}[2015/07/07]
\newcommand{\booksym}{\makebox[1em][c]{\faicon{book}}}
\newcommand{\cogsym}{\makebox[1em][c]{\faicon{cog}}}
\makeatletter
\newcommand{\hangfrom}[1]{%
  \setbox\@tempboxa\hbox{{#1}}%
  \hangindent \wd\@tempboxa\noindent\box\@tempboxa}
\makeatother
\newenvironment{tips}{%
  \begin{list}{\makebox[2em][c]{\faLightbulbO}}{%
    \setlength{\leftmargin}{2em}
    \setlength{\labelwidth}{2em}
    \setlength{\labelsep}{0pt}}
}{\end{list}}
\newenvironment{info}{%
  \begin{list}{\makebox[2em][c]{\faInfoCircle}}{%
    \setlength{\leftmargin}{2em}
    \setlength{\labelwidth}{2em}
    \setlength{\labelsep}{0pt}}
}{\end{list}}
\newenvironment{hacks}{%
  \begin{list}{\makebox[2em][c]{\faWrench}}{%
    \setlength{\leftmargin}{2em}
    \setlength{\labelwidth}{2em}
    \setlength{\labelsep}{0pt}}
}{\end{list}}

\usepackage{tcolorbox}
\tcbuselibrary{listings,breakable,skins,xparse}
\colorlet{Option}{violet}
\newcommand*{\key}[1]{\textcolor{Option}{\ttfamily #1}}
\lstloadlanguages{[LaTeX]TeX}
\lstdefinestyle{dtxlatex}%
  { columns=fullflexible
  , basicstyle=\ttfamily
  , language={[LaTeX]TeX}
  , texcsstyle=*\color{red!75!black}
  , moretexcs=
    { printbibliography
    , textcite
    , autocite
    , noop
    , addbibresource
    , assignrefcontextentries
    , newrefcontext
    }
  , moredelim=**[s][\color{violet}]{[}{]}
  , moredelim=**[s][\color{blue!75!black}]{\{}{\}}
  , mathescape
  }
\lstset{style=dtxlatex}
\tcbset
  { colframe = Slate
  , colback = BathStone!25
  , listing options =
    { style = tcblatex
    , style = dtxlatex
    , basicstyle=\ttfamily\small
    }
  }
\NewTColorBox{bibexbox}{D(){ok}d<>m}%
  {bicolor
  ,colframe = #1
  ,colback = #1!5!white
  ,colbacklower = white
  ,fontlower = \footnotesize
  ,before upper = {\hangfrom{\booksym\space}\biburlsetup}
  ,after upper = {\par\hangfrom{\cogsym\space}\fullcite{#3}}
  ,IfNoValueTF={#2}{}%
    {overlay = {
      \node[anchor=south east,text=teal] at (frame.south east) {#2};
      }
    }
  }

% For typesetting the implementation
\usepackage{doc}
\makeatletter
\newwrite\ydocwrite
\def\ydocfname{\jobname.listing}
\def\ydoc@catcodes{%
  \let\do\@makeother
  \dospecials
  \catcode`\\=\active
  \catcode`\^^M=\active
  \catcode`\ =\active
}
\def\macrocode{%
  \begingroup
  \ydoc@catcodes
  \macro@code
}
\def\endmacrocode{}
\begingroup
\endlinechar\m@ne
\@firstofone{%
\catcode`\|=0\relax
\catcode`\(=1\relax
\catcode`\)=2\relax
\catcode`\*=14\relax
\catcode`\{=12\relax
\catcode`\}=12\relax
\catcode`\ =12\relax
\catcode`\%=12\relax
\catcode`\\=\active
\catcode`\^^M=\active
\catcode`\ =\active
}*
|gdef|macro@code#1^^M%    \end{macrocode}(*
|endgroup|expandafter|macro@@code|expandafter(|ydoc@removeline#1|noexpand|lastlinemacro)*
)*
|gdef|ydoc@removeline#1^^M(|noexpand|firstlinemacro)*
|gdef|ydoc@defspecialmacros(*
|def^^M(|noexpand|newlinemacro)*
|def (|noexpand|spacemacro)*
|def\(|noexpand|bslashmacro)*
)*
|gdef|ydoc@defrevspecialmacros(*
|def|newlinemacro(|noexpand^^M)*
|def|spacemacro(|noexpand )*
|def|bslashmacro(|noexpand\)*
)*
|endgroup
\def\macro@@code#1{%
  {\ydoc@defspecialmacros
  \xdef\themacrocode{#1}}%
  \PrintMacroCode
  \end{macrocode}%
}
\def\PrintMacroCode{%
  \begingroup
  \let\firstlinemacro\empty
  \let\lastlinemacro\empty
  \def\newlinemacro{^^J}%
  \let\bslashmacro\bslash
  \let\spacemacro\space
  \immediate\openout\ydocwrite=\ydocfname\relax
  \immediate\write\ydocwrite{\themacrocode}%
  \immediate\closeout\ydocwrite
  \let\input\@input
  \tcbinputlisting{enhanced,breakable,size=small,listing only,listing file=\ydocfname}%
  \endgroup
}
\makeatother

% Documentation

\usepackage[backend=biber,bibencoding=utf8,hyperref=false,isbn=false,style=bath,sorting=ynt]{biblatex}
\addbibresource{biblatex-bath.bib}
\assignrefcontextentries[]{*}
\makeatletter
\DeclareCiteCommand{\fullcite}
  {\usebibmacro{prenote}}
  {\usedriver
     {\defcounter{maxnames}{\blx@maxbibnames}}
     {\thefield{entrytype}}}
  {\multicitedelim}
  {\usebibmacro{postnote}}
\makeatother
\xpretobibmacro{finentry}{%
  \ifboolexpr{ test {\ifcitation} and not test {\iffootnote} }{%
    \finentrypunct
  }{}%
}{}{}

\usepackage{readprov}
\usepackage[british,cleanlook]{isodate}

\usepackage[colorlinks,citecolor=black]{hyperref}

\sloppy

\title{biblatex-bath: Harvard referencing style as recommended by the University of Bath Library}
\author{%
  Maintainer: Alex Ball\thanks{%
    To contact the maintainer about this package, please visit the repository
    where the code is hosted: \url{https://github.com/alex-ball/bathbib}.%
  }%
}
\date{Package \UseVersionOf{biblatex-bath.dtx} --\printdateTeX{\UseDateOf{biblatex-bath.dtx}}}

\begin{document}
\maketitle

\section{Introduction}

\begin{markdown*}{hybrid=true}
%</driver>
%<*driver|readme>

This package provides a [biblatex] style to format reference lists in the
[Harvard style][bath-harvard] recommended by the University of Bath Library.

## Installation

You can use this style simply by copying all the `.bbx`, `.cbx`, `.dbx` and
`.lbx` files into your working directory, that is, the directory holding the
main `.tex` file for your document. If you want the style to be available for
all your documents without having to copy the files over each time, you can
install them using the instructions below.

### Managed way

The latest stable release of biblatex-bath has been packaged for TeX Live and
MiKTeX. If you are running TeX Live and have `tlmgr` installed, you can install
the package simply by running `tlmgr install biblatex-bath`. If you are running
MiKTeX, you can install the package by running `mpm --install=biblatex-bath`.
Both `tlmgr` and `mpm` have GUI versions that you might find friendlier.

### Automated way

A makefile is provided which you can use with the Make utility on
UNIX-like systems:

  * Running `make source` generates the derived files
      - `README.md`
      - `bath.bbx`, `bath.cbx`, `bath.dbx`
      - `english-bath.lbx`, `british-bath.lbx`
      - `biblatex-bath.bib`
      - `biblatex-bath.ins`
  * Running `make` generates the above files and also `biblatex-bath.pdf`.
  * Running `make inst` installs the files in the user's TeX tree.
    You can undo this with `make uninst`.
  * Running `make install` installs the files in the local TeX tree.
    You can undo this with `make uninstall`.

### Manual way

You only need to follow the first two steps if you have made your own
changes to the .dtx file. The compiled files you need are included in
the zip archive.

 1. Run `luatex biblatex-bath.dtx` to generate the source files.

 2. Compile `biblatex-bath.dtx` with [LuaLaTeX] and [Biber] to generate the
    documentation. You will need, among other things, the [luatexja],
    [adobemapping] and [ipaex] packages installed; this is just for the
    documentation, not for the biblatex style itself.

 3. If you are using TeX Live, find your home TeX tree using the following
    command at the command prompt/terminal:

    ```
    kpsewhich -var-value=TEXMFHOME
    ```

    If you are using MikTeX, consult the MikTeX manual entry for [integrating
    local additions](http://docs.miktex.org/manual/localadditions.html). You
    can use one of the roots (TeX trees) already defined – preferably one of
    the User roots – or set up a new one.

 4. Move the files to your TeX tree as follows:
      - `source/latex/biblatex-bath`:
        `biblatex-bath.dtx`,
        (`biblatex-bath.ins`)
      - `tex/latex/biblatex-bath`:
        `bath.bbx`,
        `bath.cbx`,
        `bath.dbx`,
        `english-bath.lbx`,
        `british-bath.lbx`
      - `doc/latex/biblatex-bath`:
        `biblatex-bath.pdf`,
        `README.md`

 5. You may then have to update your installation's file name database
    before TeX and friends can see the files.

[bath-harvard]: https://library.bath.ac.uk/referencing/harvard-bath
[biblatex]: http://ctan.org/pkg/biblatex
[LuaLaTeX]: http://ctan.org/pkg/lualatex-doc
[Biber]: http://ctan.org/pkg/biber
[luatexja]: http://ctan.org/pkg/luatexja
[adobemapping]: http://ctan.org/pkg/adobemapping
[ipaex]: http://ctan.org/pkg/ipaex
%</driver|readme>
%<*driver>
\end{markdown*}

\section{Using the style}

To use the style, specify it when you load \textsf{biblatex}. To get the sorting
order of your \emph{citations} right, specify the \texttt{ynt} sorting scheme.
To avoid possible inconsistencies (e.g.\@ in how disambiguation letters are
assigned), force all entries to use the global reference context:

\begin{tcblisting}{listing only}
\usepackage[style=bath,sorting=ynt]{biblatex}
\assignrefcontextentries[]{*}
\end{tcblisting}

Remember also to specify your \texttt{.bib} file in the preamble:

\begin{tcblisting}{listing only}
\addbibresource{file.bib}
\end{tcblisting}

To get the sorting order of your \emph{references} right, create a new reference
context just before you print your bibliography, using the \texttt{nyt} sorting
scheme:

\begin{tcblisting}{listing only}
\newrefcontext[sorting=nyt]
\printbibliography
\end{tcblisting}

If you forget these extra steps, the \texttt{nyt} sorting scheme will be used
throughout.

To make a citation in the text, use the key that corresponds to the entry in your \texttt{.bib} file:

\begin{tcblisting}{}
While collections can be supplemented by other means \autocite{williams1997edd},
the absence of an invisible collection amongst historians is noted by
\textcite[556]{stieg1981inh}. It may be, as \textcite{burchard1965hhl} points
out, that they have no assistants or are reluctant to delegate, or it may be
down to economic factors \autocite{adams2009tc1, adams2014tc2, adams2017tc3,
gb.pa2014}\dots
\end{tcblisting}

Please refer to the documentation for \href{http://www.ctan.org/pkg/biblatex}{\textsf{biblatex}}
for the full range of commands available for in-text citations.

It is strongly recommended to use Biber in place of BibTeX to process your
references, as the style relies on Biber functionality to deal with some of the
more exotic types of entry.


\section{Change history}

\subsection{Version 1 to version 2}

If you have previously used version 1 of this style, there are some changes
you need to be aware of.

\begin{itemize}
\item
  The code for styling legal references has been completely overhauled, so you can
  represent them more semantically in your bibliography file. The examples below
  show the fields you should now use. As far as possible they have been made
  consistent with the \href{http://ctan.org/pkg/oscola}{\textsf{biblatex-oscola}}
  style, a specialist style implementing the \emph{Oxford Standard for the Citation
  of Legal Authorities}, to enable the same entries to be used with both styles.

  For the sake of backwards compatibility, the version 1 semantics still work,
  but should be considered deprecated; support may be withdrawn in future (major)
  versions of this style.

\item
  In version 1, the \texttt{titleaddon} field was printed unadorned, so you had
  to supply your own brackets. This was to allow greater freedom to `hack' entries.

  With version 2, you should not need to resort to such dirty tricks, so the
  style now applies the brackets for you, in common with styles such as
  \href{http://ctan.org/pkg/biblatex-apa}{\textsf{biblatex-apa}}. Again, for the
  sake of backwards compatibility, before it does this it strips off any brackets
  you may have added by hand around your field value.

\item
  Since version 1 was released, there have been improvements made both to the
  Harvard (Bath) style itself and to \textsf{biblatex} internals. You may therefore
  notice some changes in how certain entries are styled, and if you have made your
  own adaptations to the style, these may no longer work with version 2.
\end{itemize}

There is now a companion Bib\TeX\ style available (\texttt{bathx} in version 2+
of the \textsf{bath-bst} package) which is compatible with this style in the
sense that it will render the same \texttt{.bib} file the same way, so long as some
adjustments are made that won't affect the \textsf{biblatex} rendering (mainly
concerning the handling of dates).

\subsection{Version 2 to version 3}

If you have previously used version 2 of this style, there are some changes
you need to be aware of.

\begin{itemize}
\item
  Journal titles are now coerced to sentence case, so any capital letters you
  want to keep need to be protected with braces.
\item
  Entries of type \texttt{unpublished} are now explicitly marked as unpublished.
\item
  With entry types that usually expect an \texttt{institution} or
  \texttt{organization} instead of a \texttt{publisher}, if you gave a
  \texttt{publisher} in earlier versions the \texttt{institution}\slash
  \texttt{organization} would be ignored. Now they are printed before the
  publication block.
\item
  Some changes to the Harvard (Bath) style cannot be applied automatically
  (e.g.~the update to how standards are referenced), so you may need to update
  your .bib file accordingly.
\end{itemize}

\section{Examples}

The examples below are shown in three parts.
The first, marked with \faBook, shows an extract from the
\href{https://library.bath.ac.uk/referencing/harvard-bath}{\emph{Referencing guide: Harvard Bath}} or
\href{https://library.bath.ac.uk/images/referencing}{\emph{Referencing images}}.
The second, marked with \faCog, shows the reference as formatted by \textsf{biblatex}.
The last shows how the reference was entered in the \texttt{.bib} file.
The bottom right corner shows the source of the example: `RX' indicates the `Reference examples (A-Z)' section of the Guide; `RL' indicates the `Organise a reference list' section of the Guide; `RI' indicates \emph{Referencing images}.

% Some examples are highlighted in
% \tcbox[colframe=hacked,colback=hacked!5!white,nobeforeafter,size=fbox,tcbox raise base]{orange}.
% This indicates that some fields have been `abused' to achieve the right effect;
% in other words, they contain information that would normally be entered in another field.
% Particular care should be taken with such items when switching between different styles,
% though of course any item might need adjustment to take account of differing conventions.

You may notice that the examples tend to use BibTeX aliases and conventions for
fields (e.g.\@ \texttt{address}, \texttt{journal}, \texttt{year}) rather than the
native \textsf{biblatex} fields. This is simply to ease transitions to and from
BibTeX, and not a symptom of recalcitrance on the part of the author.


\subsection{Books and book chapters}

\subsubsection*{Book with author(s)}

\begin{bibexbox}<RX>{rang.etal2012rdp}
  Rang, H.P., Dale, M.M., Ritter, J.M., Flower, R.J. and Henderson, G., 2012. \emph{Rang and Dale's pharmacology}. 7th ed. Edinburgh:\@ Elsevier Churchill Livingstone.
  \tcblower
\begin{Verbatim}
%</driver>
%<*driver|bib>
@book{rang.etal2012rdp,
  author = {Rang, H. P. and Dale, M. M. and Ritter, J. M. and Flower, R. J. and Henderson, G.},
  year = {2012},
  title = {Rang and {Dale's} Pharmacology},
  edition = {7},
  address = {Edinburgh},
  publisher = {Elsevier Churchill Livingstone}}
%</driver|bib>
%<*driver>
\end{Verbatim}
\end{bibexbox}

\begin{bibexbox}<RX>{ou1972em}
  Open University, 1972. \emph{Electricity and magnetism}. Bletchley:\@ Open University Press.
  \tcblower
\begin{Verbatim}
%</driver>
%<*driver|bib>
@book{ou1972em,
  author = {{Open University}},
  year = {1972},
  title = {Electricity and Magnetism},
  address = {Bletchley},
  publisher = {Open University Press}}
%</driver|bib>
%<*driver>
\end{Verbatim}
\end{bibexbox}


\subsubsection*{Book with editor(s) instead of author(s)}

\begin{bibexbox}<RX>{rothman.etal2008me}
  Rothman, K.J., Greenland, S. and Lash, T.L., eds, 2008. \emph{Modern epidemiology}. 3rd ed. Philadelphia, Pa.:\@ Lippincott Williams \& Wilkins.
  \tcblower
\begin{Verbatim}
%</driver>
%<*driver|bib>
@book{rothman.etal2008me,
  editor = {Kenneth J. Rothman and Sander Greenland and Timothy L. Lash},
  year = {2008},
  title = {Modern Epidemiology},
  edition = {3},
  address = {Philadelphia, Pa.},
  publisher = {Lippincott Williams \& Wilkins}}
%</driver|bib>
%<*driver>
\end{Verbatim}
\end{bibexbox}


\subsubsection*{Electronic book}

\begin{bibexbox}<RX>{haynes2014crc}
  Haynes, W.M., ed.\@, 2014. \emph{CRC handbook of chemistry and physics} [Online]. 94th ed. Boca Raton, Fla.:\@ CRC Press/Taylor and Francis. Available from:\@ \url{http://www.hbcpnetbase.com} [Accessed 16 June 2016].
  \tcblower
\begin{Verbatim}
%</driver>
%<*driver|bib>
@book{haynes2014crc,
  editor = {Haynes, W. M.},
  year = {2014},
  title = {{CRC} Handbook of Chemistry and Physics},
  edition = {94},
  address = {Boca Raton, Fla.},
  publisher = {CRC Press/{Taylor and Francis}},
  url = {http://www.hbcpnetbase.com},
  urldate = {2016-06-16}}
%</driver|bib>
%<*driver>
\end{Verbatim}
\end{bibexbox}

\begin{bibexbox}<RX>{hodds2016re}
  Hodds, J., 2016. \emph{Referencing ebooks} [Kindle version 4.18]. Bath:\@ University of Bath.
  \tcblower
\begin{Verbatim}
%</driver>
%<*driver|bib>
@book{hodds2016re,
  author = {Hodds, J.},
  year = {2016},
  title = {Referencing ebooks},
  titleaddon = {Kindle version 4.18},
  address = {Bath},
  publisher = {University of Bath}}
%</driver|bib>
%<*driver>
\end{Verbatim}
\end{bibexbox}


\subsubsection*{One chapter\slash paper from a collection (by different authors) in an edited book}

\begin{bibexbox}<RL>{burchard1965hhl}
  Burchard, J.E., 1965. How humanists use a library. In:\@ C.F.J. Overhage and J.R. Harman, eds. \emph{Intrex:\@ report on a planning conference and information transfer experiments}. Cambridge, Mass.:\@ MIT Press, pp.41--87.
  \tcblower
\begin{Verbatim}
%</driver>
%<*driver|bib>
@incollection{burchard1965hhl,
  author = {Burchard, J. E.},
  year = {1965},
  title = {How Humanists use a Library},
  editor = {C. F. J. Overhage and J. R. Harman},
  booktitle = {Intrex},
  booksubtitle = {Report on a Planning Conference and Information Transfer Experiments},
  address = {Cambridge, Mass.},
  publisher = {MIT Press},
  pages = {41-87}}
%</driver|bib>
%<*driver>
\end{Verbatim}
\end{bibexbox}

\begin{bibexbox}<RX>{reid1967ptp}
  Reid, D.R., 1967. Physical testing of polymer films. In:\@ S.H. Pinner, ed. \emph{Modern packaging films}. London:\@ Butterworths, pp.143--183.
  \tcblower
\begin{Verbatim}
%</driver>
%<*driver|bib>
@incollection{reid1967ptp,
  author = {D. R. Reid},
  year = {1967},
  title = {Physical Testing of Polymer Films},
  editor = {S. H. Pinner},
  booktitle = {Modern Packaging Films},
  address = {London},
  publisher = {Butterworths},
  pages = {143-183}}
%</driver|bib>
%<*driver>
\end{Verbatim}
\end{bibexbox}


\subsection{Articles and periodicals}

\subsubsection*{Journal article}

\begin{bibexbox}<RL>{stieg1981cer}
  Stieg, M.F., 1981a. Continuing education and the reference librarian in the academic and research library. \emph{Library journal}, 105(22), pp.2547--2551.
  \tcblower
\begin{Verbatim}
%</driver>
%<*driver|bib>
@article{stieg1981cer,
  author = {Stieg, M. F.},
  year = {1981},
  title = {Continuing Education and the Reference Librarian in the Academic and Research Library},
  journal = {Library Journal},
  volume = {105},
  number ={22},
  pages = {2547-2551}}
%</driver|bib>
%<*driver>
\end{Verbatim}
\end{bibexbox}

\begin{bibexbox}<RL>{stieg1981inh}
  Stieg, M.F., 1981b. The information needs of historians. \emph{College and research libraries}, 42(6), pp.549--560.
  \tcblower
\begin{Verbatim}
%</driver>
%<*driver|bib>
@article{stieg1981inh,
  author = {Stieg, M. F.},
  year = {1981},
  title = {The Information Needs of Historians},
  journal = {College and Research Libraries},
  volume = {42},
  number ={6},
  pages = {549-560}}
%</driver|bib>
%<*driver>
\end{Verbatim}
\end{bibexbox}

\begin{bibexbox}<RX>{newman2010mcb}
  Newman, R., 2010. Malaria control beyond 2010. \emph{Brit.\@ Med.\@ J.}, 341(7765), pp.157--208.
  \tcblower
\begin{Verbatim}
%</driver>
%<*driver|bib>
@article{newman2010mcb,
  author = {Newman, R.},
  year = {2010},
  title = {Malaria control beyond 2010},
  journal = {{Brit.\@ Med.\@ J.\isdot}},
  volume = {341},
  number = {7765},
  pages = {157-208}}
%</driver|bib>
%<*driver>
\end{Verbatim}
\end{bibexbox}


\subsubsection*{Electronic journal article}

\begin{bibexbox}<RX>{williams1997edd}
  Williams, F., 1997. Electronic document delivery:\@ a trial in an academic library. \emph{Ariadne} [Online], 10. Available from:\@ \url{http://www.ariadne.ac.uk/issue10/edd/} [Accessed 5 December 1997].
  \tcblower
\begin{Verbatim}
%</driver>
%<*driver|bib>
@article{williams1997edd,
  author = {Williams, F.},
  year = {1997},
  title = {Electronic Document Delivery},
  subtitle = {A Trial in an Academic Library},
  journal = {Ariadne},
  volume = {10},
  url = {http://www.ariadne.ac.uk/issue10/edd/},
  urldate = {1997-12-05}}
%</driver|bib>
%<*driver>
\end{Verbatim}
\end{bibexbox}

\begin{tips}
\item
If citing an article yet to be officially published, use the \texttt{pubstate}
field with the keyword \texttt{inpress} for `in press' and
\texttt{inpreparation} or \texttt{submitted} (whichever is more accurate)
for `preprint'.
\end{tips}

\begin{bibexbox}<RX>{liontou.etal2019dra}
  Liontou, C., Kontopodis, E., Oikonomidis, N., Maniotis, C., Tassopoulos, A., Tsiafoutis, I., Lazaris, E. and Koutouzis, M., 2019. Distal radial access:\@ a review article. \emph{Cardiovascular revascularization medicine} [Online], in press. Available from: \url{https://www.sciencedirect.com/science/article/pii/S1553838919303367} [Accessed 19 June 2019].
  \tcblower
\begin{Verbatim}
%</driver>
%<*driver|bib>
@article{liontou.etal2019dra,
  author = {Liontou, C. and Kontopodis, E. and Oikonomidis, N. and Maniotis, C. and
    Tassopoulos, A. and Tsiafoutis, I. and Lazaris, E. and Koutouzis, M.},
  year = {2019},
  title = {Distal Radial Access: a Review Article},
  journal = {Cardiovascular Revascularization Medicine},
  pubstate = {inpress},
  url = {https://www.sciencedirect.com/science/article/pii/S1553838919303367},
  urldate = {2019-06-19}}
%</driver|bib>
%<*driver>
\end{Verbatim}
\end{bibexbox}


\subsubsection*{Preprint in a digital repository}

\begin{bibexbox}<RX>{shah.corrick2016hsc}
  Shah, I. and Corrick, I., 2016. \emph{How should central banks respond to non-neutral inflation expectations?} Bath:\@ University of Bath. \emph{OPUS} [Online]. Available from:\@ \url{http://opus.bath.ac.uk} [Accessed 4 May 2016].
  \tcblower
\begin{Verbatim}
%</driver>
%<*driver|bib>
@report{shah.corrick2016hsc,
  author = {Shah, I. and Corrick, I.},
  year = {2016},
  title = {How should central banks respond to non-neutral inflation expectations?},
  address = {Bath},
  institution = {University of Bath},
  library = {OPUS},
  url = {http://opus.bath.ac.uk},
  urldate = {2016-05-04}}
%</driver|bib>
%<*driver>
\end{Verbatim}
\end{bibexbox}

\begin{tips}
\item
The Harvard (Bath) style treats preprints as grey literature, and so the
\texttt{report} entry type is the best match semantically. Use \texttt{library}
to specify the name of the preprint repository.

In standard \textsf{biblatex}, you would typically use the natural entry type
for the work and provide preprint details using the \texttt{eprinttype},
\texttt{eprintclass} and \texttt{eprint} fields.
\end{tips}


\subsubsection*{Newspaper article}

\begin{bibexbox}<RX>{haurant2004bbh}
  Haurant, S., 2004. Britain's borrowing hits £1 trillion. \emph{The Guardian}, 29 July, p.16c.
  \tcblower
\begin{Verbatim}
%</driver>
%<*driver|bib>
@article{haurant2004bbh,
  author = {Haurant, S.},
  date = {2004-07-29},
  title = {Britain's Borrowing Hits \pounds 1 Trillion},
  journal = {{The Guardian}},
  pages = {16c}}
%</driver|bib>
%<*driver>
\end{Verbatim}
\end{bibexbox}

\begin{bibexbox}<RX>{independent1992pub}
  The Independent, 1992. Picking up the bills. \emph{The Independent}, 4 June, p.28a.
  \tcblower
\begin{Verbatim}
%</driver>
%<*driver|bib>
@article{independent1992pub,
  author = {{The Independent}},
  date = {1992-06-04},
  title = {Picking Up the Bills},
  journal = {{The Independent}},
  pages = {28a}}
%</driver|bib>
%<*driver>
\end{Verbatim}
\end{bibexbox}


\subsection{Conference papers}

\subsubsection*{Conference paper (when proceedings have a named editor)}

\begin{bibexbox}<RX>{crawford1965oim}
  Crawford, G.I., 1965. Oxygen in metals. In:\@ J.M.A. Lenihan and S.J. Thompson, eds. \emph{Activation analysis:\@ proceedings of a NATO Advanced Study Institute}, 2--4 August 1964, Glasgow. London:\@ Academic Press, pp.113--118.
  \tcblower
\begin{Verbatim}
%</driver>
%<*driver|bib>
@inproceedings{crawford1965oim,
  author = {Crawford, G. I.},
  year = {1965},
  title = {Oxygen in Metals},
  editor = {J. M. A. Lenihan and S. J. Thompson},
  booktitle = {Activation Analysis},
  booksubtitle = {Proceedings of a {NATO} {Advanced} {Study} {Institute}},
  eventdate = {1964-08-02/1964-08-04},
  venue = {Glasgow},
  address = {London},
  publisher = {Academic Press},
  pages = {113-118}}
%</driver|bib>
%<*driver>
\end{Verbatim}
\end{bibexbox}


\subsubsection*{Conference paper (when proceedings have no named editor or are part of a major series)}

\begin{bibexbox}<RX>{soper1972rbc}
  Soper, D., 1972. Review of bracken control experiments with asulam. \emph{Proceedings of the 11th British Weed Control Conference}, 15--17 November 1972, Brighton. Brighton:\@ University of Sussex, pp.24--31.
  \tcblower
\begin{Verbatim}
%</driver>
%<*driver|bib>
@inproceedings{soper1972rbc,
  author = {Soper, D.},
  year = {1972},
  title = {Review of Bracken Control Experiments with Asulam},
  booktitle = {Proceedings of the 11th {British} {Weed} {Control} {Conference}},
  eventdate = {1972-11-15/1972-11-17},
  venue = {Brighton},
  address = {Brighton},
  publisher = {University of Sussex},
  pages = {24-31}}
%</driver|bib>
%<*driver>
\end{Verbatim}
\end{bibexbox}


\subsection{Grey literature}

\subsubsection*{Thesis/dissertation}

\begin{bibexbox}<RX>{burrell1973ist}
  Burrell, J.G., 1973. \emph{The importance of school tours in education}. Thesis (M.A.). Queen's University, Belfast.
  \tcblower
\begin{Verbatim}
%</driver>
%<*driver|bib>
@thesis{burrell1973ist,
  author = {Burrell, J. G.},
  year = {1973},
  title = {The Importance of School Tours in Education},
  type = {Thesis \parentext{M.A.}},
  school = {Queen's University, Belfast}}
%</driver|bib>
%<*driver>
\end{Verbatim}
\end{bibexbox}


\subsubsection*{Report}

\begin{bibexbox}<RX>{unesco1993gip}
  UNESCO, 1993. \emph{General information programme and UNISIST}\@. (PGI-93/WS/22). Paris:\@ UNESCO.
  \tcblower
\begin{Verbatim}
%</driver>
%<*driver|bib>
@report{unesco1993gip,
  author = {{UNESCO}},
  year = {1993},
  title = {General Information Programme and {UNISIST}},
  address = {Paris},
  institution = {UNESCO},
  number = {PGI-93/WS/22}}
%</driver|bib>
%<*driver>
\end{Verbatim}
\end{bibexbox}

\begin{bibexbox}<RX>{bre2007dqb}
  BRE, 2007. \emph{Designing quality buildings:\@ a BRE guide}. (Report 497). Bracknell:\@ BRE.
  \tcblower
\begin{Verbatim}
%</driver>
%<*driver|bib>
@techreport{bre2007dqb,
  author = {{BRE}},
  year = {2007},
  title = {Designing Quality Buildings: a {BRE} Guide},
  address = {Bracknell},
  institution = {BRE},
  type = {Report},
  number = {497}}
%</driver|bib>
%<*driver>
\end{Verbatim}
\end{bibexbox}


\subsubsection*{Standard}

\begin{bibexbox}<RX>{bs5605:1990}
  BSI, 1990. \emph{BS 5605:1990 Recommendations for citing and referencing
  published material.} London: BSI.
  \tcblower
\begin{Verbatim}
%</driver>
%<*driver|bib>
@standard{bs5605:1990,
  author = {{BSI}},
  year = {1990},
  title = {{BS}~5605:1990 {Recommendations} for Citing and Referencing Published Material},
  address = {London},
  organization = {BSI}}
%</driver|bib>
%<*driver>
\end{Verbatim}
\end{bibexbox}

\begin{bibexbox}<RX>{astm.d1655}
  ASTM, 2019. \emph{ASTM D1655 - 19 Standard specification for aviation
  turbine fuels.} West Conshohocken, Pa.: ASTM.
  \tcblower
\begin{Verbatim}
%</driver>
%<*driver|bib>
@standard{astm.d1655,
  author = {{ASTM}},
  year = {2019},
  title = {{ASTM~D1655} - 19 {Standard} Specification for Aviation Turbine Fuels},
  address = {West Conshohocken, Pa.},
  organization = {ASTM}}
%</driver|bib>
%<*driver>
\end{Verbatim}
\end{bibexbox}


\subsubsection*{Patent}

\begin{bibexbox}<RX>{pm1981opa}
  Phillipp Morris Inc., 1981. \emph{Optical perforating apparatus and system}. European patent application 0021165A1. 1981-01-07.
  \tcblower
\begin{Verbatim}
%</driver>
%<*driver|bib>
@patent{pm1981opa,
  author = {{Phillipp Morris Inc.}},
  year = {1981},
  title = {Optical perforating apparatus and system},
  type = {European patent application},
  number = {0021165A1. 1981-01-07}}
%</driver|bib>
%<*driver>
\end{Verbatim}
\end{bibexbox}

\begin{info}\item
Note that \textsf{biblatex} has special localisation keys for full patents and
patent requests\slash applications.
\end{info}


\subsubsection*{Unpublished written material and personal communications}

\begin{bibexbox}<RX>{harris2013fgr}
  Harris, G., 2013. \emph{Focus group recommendations:\@ internal task group
  report}. Unpublished.
  \tcblower
\begin{Verbatim}
%</driver>
%<*driver|bib>
@unpublished{harris2013fgr,
  author = {Harris, G.},
  year = {2013},
  title = {Focus group recommendations: internal task group report}}
%</driver|bib>
%<*driver>
\end{Verbatim}
\end{bibexbox}

\begin{bibexbox}<RX>{hadley2015bir}
  Hadley, S., 2015. \emph{Biomechanics:\@ introductory reading, BM289:\@ sport
  biomechanics}. University of Bath. Unpublished.
  \tcblower
\begin{Verbatim}
%</driver>
%<*driver|bib>
@unpublished{hadley2015bir,
  author = {Hadley, S.},
  year = {2015},
  title = {Biomechanics: introductory reading, {BM289}: sport biomechanics},
  howpublished = {University of Bath}}
%</driver|bib>
%<*driver>
\end{Verbatim}
\end{bibexbox}

\begin{bibexbox}<RX>{thomas2015wcr}
  Thomas, D., 2015. Word count and referencing style. \emph{Frequently
  asked questions discussion board:\@ PHYS 2011:\@ housing studies.}
  University of Bath. Unpublished.
  \tcblower
\begin{Verbatim}
%</driver>
%<*driver|bib>
@unpublished{thomas2015wcr,
  author = {Thomas, D.},
  year = {2015},
  title = {Word count and referencing style},
  booktitle = {Frequently asked questions discussion board: {PHYS} 2011: housing studies},
  howpublished = {University of Bath}}
%</driver|bib>
%<*driver>
\end{Verbatim}
\end{bibexbox}


\subsection{Audiovisual materials}

\subsubsection*{Image}

\begin{bibexbox}<RI>{nasa2015nat}
   NASA, 2015, \emph{NASA astronaut Tim Kopra on Dec.\@ 21 spacewalk} [Online]. Washington:\@ NASA. Available from:\@ \url{http://www.nasa.gov/image-feature/nasa-astronaut-tim-kopra-on-dec-21-spacewalk} [Accessed 7 January 2015].
  \tcblower
\begin{Verbatim}
%</driver>
%<*driver|bib>
@image{nasa2015nat,
  author = {{NASA}},
  year = {2015},
  title = {{NASA} Astronaut {Tim} {Kopra} on {Dec.\@} 21 Spacewalk},
  address = {Washington},
  publisher = {NASA},
  url = {http://www.nasa.gov/image-feature/nasa-astronaut-tim-kopra-on-dec-21-spacewalk},
  urldate = {2015-01-07}}
%</driver|bib>
%<*driver>
\end{Verbatim}
\end{bibexbox}

\begin{tips}\item
You can use either the \texttt{publisher} or the \texttt{organization} field to
record the source of the image.
\end{tips}

\begin{bibexbox}<RI>{iliff2006rcb}
   Iliff, D., 2006. \emph{Royal Crescent in Bath, England - July 2006} [Online]. San Francisco:\@ Wikimedia Foundation. Available from:\@ \url{https://commons.wikimedia.org/wiki/File:Royal_Crescent_in_Bath,_England_-_July_2006.jpg} [Accessed 7 January 2016].
  \tcblower
\begin{Verbatim}
%</driver>
%<*driver|bib>
@image{iliff2006rcb,
  author = {D. Iliff},
  year = {2006},
  title = {{Royal} {Crescent} in {Bath,} {England} - {July} 2006},
  address = {San Francisco},
  organization = {Wikimedia Foundation},
  url = {https://commons.wikimedia.org/wiki/File:Royal_Crescent_in_Bath,_England_-_July_2006.jpg},
  urldate = {2015-01-07}}
%</driver|bib>
%<*driver>
\end{Verbatim}
\end{bibexbox}

\begin{bibexbox}<RI>{anon1946peb}
  Anon., 1946. \emph{Prototype electric bicycle displayed at the \enquote{Britain Can Make It} exhibition organised by the Council of Industrial Design and held at the Victoria and Albert Museum, London, 1946. Designed by B.~G. Bowden} [Photograph]. At:\@ London. Design Council Slide Collection.
  \tcblower
\begin{Verbatim}
%</driver>
%<*driver|bib>
@image{anon1946peb,
  author = {Anon\adddot},
  year = {1946},
  title = {Prototype electric bicycle displayed at the \enquote{Britain Can Make It} exhibition
    organised by the {Council} of {Industrial} {Design} and held at the {Victoria} and {Albert}
    {Museum}, {London}, 1946. {Designed} by {B.~G.} {Bowden}},
  titleaddon = {Photograph},
  address = {London},
  library = {Design Council Slide Collection}}
%</driver|bib>
%<*driver>
\end{Verbatim}
\end{bibexbox}

\begin{tips}\item
Use the \texttt{library} field to record the archive and register number.
If the image is on display rather than archived,
use \texttt{institution} instead for the museum, gallery or building.
If you also wish to provide the name of the organisation that published the image,
use the \texttt{publisher} field if the location you provide relates to the publisher,
but the \texttt{organization} field if the location relates to the archive.
\end{tips}


\subsubsection*{Map}

\begin{bibexbox}<RX>{andrews.dury1773wilts}
  Andrews, J. and Dury, A., 1773. \emph{Map of Wiltshire}, 1 inch to 2 miles. Devizes:\@ Wiltshire Record Society.
  \tcblower
\begin{Verbatim}
%</driver>
%<*driver|bib>
@manual{andrews.dury1773wilts,
  author = {Andrews, J. and Dury, A.},
  year = {1773},
  title = {Map of {Wiltshire}},
  series = {1 inch to 2 miles},
  address = {Devizes},
  publisher = {Wiltshire Record Society}}
%</driver|bib>
%<*driver>
\end{Verbatim}
\end{bibexbox}

\begin{info}\item
The \texttt{book} entry type would also work for this reference.
\end{info}

\subsubsection*{Film, video or DVD}

\begin{bibexbox}<RX>{macbeth1948}
  \emph{Macbeth}, 1948. Film.\@ Directed by Orson Welles. USA:\@ Republic Pictures.
  \tcblower
\begin{Verbatim}
%</driver>
%<*driver|bib>
@video{macbeth1948,
  year = {1948},
  title = {Macbeth},
  type = {Film},
  note = {Directed by Orson Welles},
  address = {USA},
  publisher = {Republic Pictures}}
%</driver|bib>
%<*driver>
\end{Verbatim}
\end{bibexbox}

\begin{info}\item
In the above entry, the following would also work instead of using \texttt{note}:

\begin{tcolorbox}%
  [ colframe = Slate
  , colback = white
  , fontupper = \footnotesize
  ]
\begin{Verbatim}
  editor = {Orson Welles},
  editortype = {director},
\end{Verbatim}
\end{tcolorbox}

\item
You can also use \texttt{movie} as an alias for \texttt{video}.
\end{info}


\subsubsection*{Streamed video (YouTube, TED Talks, etc.)}

\begin{bibexbox}<RX>{moran2016sol}
  Moran, C., 2016. \emph{Save our libraries} [Online]. Available from:\@ \url{https://www.youtube.com/watch?v=gKTfCz4JtVE&feature=youtu.be} [Accessed 29 April 2016].
  \tcblower
\begin{Verbatim}
%</driver>
%<*driver|bib>
@video{moran2016sol,
  author = {Moran, C.},
  year = {2016},
  title = {Save Our Libraries},
  url = {https://www.youtube.com/watch?v=gKTfCz4JtVE&feature=youtu.be},
  urldate = {2016-04-29}}
%</driver|bib>
%<*driver>
\end{Verbatim}
\end{bibexbox}

\begin{bibexbox}<RI>{uob2015wie}
   University of Bath, 2015. \emph{What is engineering?} [Online]. Available from:\@ \url{https://www.youtube.com/watch?v=NoyZarq-Zbo} [Accessed 12 January 2016].
  \tcblower
\begin{Verbatim}
%</driver>
%<*driver|bib>
@video{uob2015wie,
  author = {{University of Bath}},
  year = {2015},
  title = {What is Engineering?},
  url = {https://www.youtube.com/watch?v=NoyZarq-Zbo},
  urldate = {2016-01-12}}
%</driver|bib>
%<*driver>
\end{Verbatim}
\end{bibexbox}

\begin{bibexbox}<RI>{chakrabarti2016hac}
   Chakrabarti, V., 2016. \emph{How architecture and city planning can combat social inequality} [Online]. Available from:\@ \url{https://www.curbed.com/2016/5/5/11593058/vishaan-chakrabarti-pau-curbed-appeal-podcast} [Accessed 28 March 2019].
  \tcblower
\begin{Verbatim}
%</driver>
%<*driver|bib>
@audio{chakrabarti2016hac,
  author = {Chakrabarti, V.},
  year = {2016},
  title = {How Architecture and City Planning Can Combat Social Inequality},
  url = {https://www.curbed.com/2016/5/5/11593058/vishaan-chakrabarti-pau-curbed-appeal-podcast},
  urldate = {2019-03-28}}
%</driver|bib>
%<*driver>
\end{Verbatim}
\end{bibexbox}


\subsubsection*{Television or radio broadcast}

\begin{bibexbox}<RX>{rsfo2006ep5}
  \emph{Rick Stein's French odyssey:\@ Episode 5}, 2006. TV. BBC2, 23 August. 20.30 hrs.
  \tcblower
\begin{Verbatim}
%</driver>
%<*driver|bib>
@video{rsfo2006ep5,
  date = {2006-08-23T20:30:00},
  title = {Rick {Stein's} {French} Odyssey},
  subtitle = {{Episode} 5},
  type = {TV},
  publisher = {BBC2}}
%</driver|bib>
%<*driver>
\end{Verbatim}
\end{bibexbox}

\begin{tips}\item
Use \texttt{type} for the medium and \texttt{publisher} for the channel.
\end{tips}

\begin{bibexbox}<RX>{archers20060823}
  \emph{The Archers}, 2006. Radio. BBC Radio 4, 23 August. 19.02 hrs.
  \tcblower
\begin{Verbatim}
%</driver>
%<*driver|bib>
@audio{archers20060823,
  date = {2006-08-23T19:02:00},
  title = {The {Archers}},
  type = {Radio},
  publisher = {BBC Radio 4}}
%</driver|bib>
%<*driver>
\end{Verbatim}
\end{bibexbox}

\begin{info}\item
You can also use \texttt{music} as an alias for \texttt{audio}.
\end{info}


\subsubsection*{Music score}

\begin{bibexbox}<RX>{beethoven1950symph1}
  Beethoven, L. van, 1950. \emph{Symphony no.1 in C, Op.21}. Harmondsworth:\@ Penguin.
  \tcblower
\begin{Verbatim}
%</driver>
%<*driver|bib>
@book{beethoven1950symph1,
  author = {Ludwig van Beethoven},
  year = {1950},
  title = {Symphony no.1 in {C,} {Op.21}},
  address = {Harmondsworth},
  publisher = {Penguin}}
%</driver|bib>
%<*driver>
\end{Verbatim}
\end{bibexbox}


\subsection{Digital media}

\subsubsection*{Website\slash webpage}

\begin{bibexbox}<RX>{holland2002gci}
  Holland, M., 2002. \emph{Guide to citing internet sources} [Online]. Poole:\@ Bournemouth University. Available from:\@ \url{http://www.bournemouth.ac.uk/library/using/guide_to_citing_internet_sourc.html} [Accessed 4 November 2002].
  \tcblower
\begin{Verbatim}
%</driver>
%<*driver|bib>
@online{holland2002gci,
  author = {Holland, M.},
  year = {2002},
  title = {Guide to Citing Internet Sources},
  address = {Poole},
  organization = {Bournemouth University},
  url = {http://www.bournemouth.ac.uk/library/using/guide_to_citing_internet_sourc.html},
  urldate = {2002-11-04}}
%</driver|bib>
%<*driver>
\end{Verbatim}
\end{bibexbox}


\subsubsection*{Email discussion lists (jiscmail\slash listserv etc.)}

\begin{bibexbox}<RX>{clark2004euk}
  Clark, T., 5 July 2004. A European UK Libraries Plus? \emph{Lis-link} [Online]. Available from:\@ \url{lis-link@jiscmail.ac.uk} [Accessed 30 July 2004].
  \tcblower
\begin{Verbatim}
%</driver>
%<*driver|bib>
@letter{clark2004euk,
  author = {Clark, T.},
  date = {2004-07-05},
  title = {A {European} {UK} {Libraries} {Plus}?},
  journal = {Lis-link},
  url = {lis-link@jiscmail.ac.uk},
  urldate = {2004-07-30}}
%</driver|bib>
%<*driver>
\end{Verbatim}
\end{bibexbox}

\begin{tips}\item
Use the \texttt{journal} field to specify the mailing list. If you omit the
\texttt{journal} field, the entry will be formatted as an unpublished work.
\end{tips}


\subsubsection*{Database}

\begin{bibexbox}<RX>{bvd2008bt}
  Bureau van Dijk, 2008. \emph{BT Group plc company report}. \emph{FAME} [Online]. London:\@ Bureau van Dijk. Available from:\@ \url{http://www.portal.euromonitor.com} [Accessed 6 November 2014].
  \tcblower
\begin{Verbatim}
%</driver>
%<*driver|bib>
@online{bvd2008bt,
  author = {{Bureau van Dijk}},
  year = {2008},
  title = {{BT} {Group} PLC Company Report},
  library = {FAME},
  address = {London},
  organization = {Bureau van Dijk},
  url = {http://www.portal.euromonitor.com},
  urldate = {2014-11-06}}
%</driver|bib>
%<*driver>
\end{Verbatim}
\end{bibexbox}

\begin{tips}\item
Use the \texttt{title} field for the entry title,
and the \texttt{library} field for the name of the database itself.
\end{tips}


\subsubsection*{Dataset}

\begin{bibexbox}<RX>{wilson2013rgc}
  Wilson, D., 2013. \emph{Real geometry and connectedness via triangular description:\@ CAD example bank} [Online]. Bath:\@ University of Bath. Available from:\@ \url{https://doi.org/10.15125/BATH-00069} [Accessed 20 April 2016].
  \tcblower
\begin{Verbatim}
%</driver>
%<*driver|bib>
@online{wilson2013rgc,
  author = {Wilson, D.},
  year = {2013},
  title = {Real Geometry and Connectedness via Triangular Description},
  subtitle = {{CAD} Example Bank},
  address = {Bath},
  organization = {University of Bath},
  doi = {10.15125/BATH-00069},
  urldate = {2016-04-20}}
%</driver|bib>
%<*driver>
\end{Verbatim}
\end{bibexbox}

\begin{tips}\item
You can also use \texttt{dataset} as an alias for \texttt{online}.
\end{tips}


\subsubsection*{Computer program}

\begin{bibexbox}<RX>{screencasto}
  @screencasto, n.d. \emph{Screencast-O-Matic} (v.2) [computer program]. Available from:\@ \url{https://screencast-o-matic.com/} [Accessed 16 May 2016].
  \tcblower
\begin{Verbatim}
%</driver>
%<*driver|bib>
@software{screencasto,
  author = {@screencasto},
  title = {{Screencast-O-Matic}},
  version = {2},
  titleaddon = {computer program},
  url = {https://screencast-o-matic.com/},
  urldate = {2016-05-16}}
%</driver|bib>
%<*driver>
\end{Verbatim}
\end{bibexbox}


\subsection{Works in languages other than English}

\subsubsection*{Work in translation}

\begin{bibexbox}<RX>{aristotle2007ne}
  Aristotle, 2007. \emph{Nicomachean ethics} (W.D. Ross, Trans.). South Dakota:\@ NuVisions.
  \tcblower
\begin{Verbatim}
%</driver>
%<*driver|bib>
@book{aristotle2007ne,
  author = {Aristotle},
  year = {2007},
  title = {Nicomachean Ethics},
  translator = {W. D. Ross},
  address = {South Dakota},
  publisher = {NuVisions}}
%</driver|bib>
%<*driver>
\end{Verbatim}
\end{bibexbox}


\subsubsection*{Work in the Roman alphabet}

\begin{bibexbox}<RX>{esquivel2003cap}
  Esquivel, L., 2003. \emph{Como agua para chocolate} [Like water for chocolate]. Barcelona:\@ Debolsillo.
  \tcblower
\begin{Verbatim}
%</driver>
%<*driver|bib>
@book{esquivel2003cap,
  author = {Esquivel, L.},
  year = {2003},
  title = {Como Agua para Chocolate},
  titleaddon = {Like water for chocolate},
  address = {Barcelona},
  publisher = {Debolsillo}}
%</driver|bib>
%<*driver>
\end{Verbatim}
\end{bibexbox}

\begin{tips}\item
Use the \texttt{titleaddon} field to supply the English translation of the title.
\end{tips}

\begin{bibexbox}<RX>{thurfjell1975vhv}
  Thurfjell, W., 1975. Vart har våran doktor tagit vägen? [Where has our doctor gone?]. \emph{Läkartidningen}, 72, p.789.
  \tcblower
\begin{Verbatim}
%</driver>
%<*driver|bib>
@article{thurfjell1975vhv,
  author = {Thurfjell, W.},
  year = {1975},
  title = {Vart har våran doktor tagit vägen?},
  titleaddon = {Where has our doctor gone?},
  journal = {Läkartidningen},
  volume = {72},
  pages = {789}}
%</driver|bib>
%<*driver>
\end{Verbatim}
\end{bibexbox}


\subsubsection*{Work in a non-Roman alphabet}

\begin{bibexbox}<RX>{hua1999qys1}
  Hua, L. 華林甫, 1999.  Qingdai yilai Sanxia diqu shuihan zaihai de chubu yanjiu 清代以來三峽地區水旱災害的初步硏 [A preliminary study of floods and droughts in the Three Gorges region since the Qing dynasty]. \emph{Zhongguo shehui kexue} 中國社會科學, 1, pp.168--79.
  \tcblower
\begin{Verbatim}
%</driver>
%<*driver|bib>
@article{hua1999qys1,
  author = {given=Linfu, family=Hua, cjk=華林甫},
  year = {1999},
  title = {Qingdai yilai {Sanxia} diqu shuihan zaihai de chubu yanjiu
    {清代以來三峽地區水旱災害的初步硏}},
  titleaddon = {A preliminary study of floods and droughts in the {Three} {Gorges} region since
    the {Qing} dynasty},
  journal = {Zhongguo shehui kexue \textup{中國社會科學}},
  volume = {1},
  pages = {168-79}}
%</driver|bib>
%<*driver>
\end{Verbatim}
\end{bibexbox}

\begin{tips}\item
To supply a transliterated version of an author name, specify the name using the
name parts \texttt{family}, \texttt{given} and \texttt{cjk}. Please note that if
you do supply a \texttt{cjk} component, any \texttt{prefix} or \texttt{suffix}
component you may supply will be ignored.
\end{tips}

\begin{bibexbox}<RX>{hua1999qys2}
  Hua, L., 1999. Qingdai yilai Sanxia diqu shuihan zaihai de chubu yanjiu [A preliminary study of floods and droughts in the Three Gorges region since the Qing dynasty]. \emph{Zhongguo shehui kexue}, 1, pp.168--79.
  \tcblower
\begin{Verbatim}
%</driver>
%<*driver|bib>
@article{hua1999qys2,
  author = {Hua, Linfu},
  year = {1999},
  title = {Qingdai yilai {Sanxia} diqu shuihan zaihai de chubu yanjiu},
  titleaddon = {A preliminary study of floods and droughts in the {Three} {Gorges} region since
    the {Qing} dynasty},
  journal = {Zhongguo shehui kexue},
  volume = {1},
  pages = {168-79}}
%</driver|bib>
%<*driver>
\end{Verbatim}
\end{bibexbox}

\begin{tips}\item
Although not a feature of the Harvard (Bath) Style, if you want to suppress the
punctuation between the family name and the initial (and thereby be more
faithful to the original orthography), you can specify this using
\textsf{biblatex}'s data annotations feature, using the keyword \texttt{cjk}:
\end{tips}

\begin{bibexbox}{hua2001foo}
  Hua L. 華林甫, 2001. \emph{Lorem ipsum}.
  \tcblower
\begin{Verbatim}
%</driver>
%<*driver|bib>
@book{hua2001foo,
  author = {given=Linfu, family=Hua, cjk=華林甫},
  author+an = {1=cjk},
  year = {2001},
  title = {Lorem ipsum}}
%</driver|bib>
%<*driver>
\end{Verbatim}
\end{bibexbox}

\begin{bibexbox}<RX>{pamporov2006rvb}
  Pamporov, A., 2006. \emph{Romskoto vsekidnevie v Balgariya} [Roma everyday life in Bulgaria]. Veliko Tarnovo: Faber.
  \tcblower
\begin{Verbatim}
%</driver>
%<*driver|bib>
@book{pamporov2006rvb,
  author = {Pamporov, A.},
  year = {2006},
  title = {Romskoto vsekidnevie v {Balgariya}},
  titleaddon = {Roma everyday life in Bulgaria},
  address = {Veliko Tarnovo},
  publisher = {Faber}}
%</driver|bib>
%<*driver>
\end{Verbatim}
\end{bibexbox}


\subsection{Legal references: UK legislation and parliamentary reports}

% UK Primary Legislation

\subsubsection*{Act of Parliament (UK Statutes) before 1963}

\begin{bibexbox}<RX>{gb.wa1735}
  \emph{Witchcraft Act 1735} (9 Geo.2, c.5).
  \tcblower
\begin{Verbatim}
%</driver>
%<*driver|bib>
@legislation{gb.wa1735,
  title = {Witchcraft {Act}},
  year = {1735},
  series = {9 Geo.2},
  chapter = {5}}
%</driver|bib>
%<*driver>
\end{Verbatim}
\end{bibexbox}

\begin{info}\item
You could instead combine the series and chapter in the \texttt{number} field:

\begin{tcolorbox}%
  [ colframe = Slate
  , colback = white
  , fontupper = \footnotesize
  ]
\begin{Verbatim}
  number = {9 Geo.2, c.5}
\end{Verbatim}
\end{tcolorbox}
\end{info}

\subsubsection*{Act of Parliament (UK Statutes) 1963 onwards}

\begin{bibexbox}<RX>{gb.pa2014}
  \emph{Pensions Act 2014}, c.19. London:\@ TSO.
  \tcblower
\begin{Verbatim}
%</driver>
%<*driver|bib>
@legislation{gb.pa2014,
  title = {Pensions {Act}},
  year = {2014},
  chapter = {19},
  address = {London},
  publisher = {TSO}}
%</driver|bib>
%<*driver>
\end{Verbatim}
\end{bibexbox}


\subsubsection*{House of Commons/House of Lords bill}

\begin{bibexbox}<RX>{gb.bill1987/88-66}
  Great Britain.\@ Parliament.\@ House of Commons, 1988. \emph{Local government finance bill}. (Bills | 1987/88, 66). London:\@ HMSO.
  \tcblower
\begin{Verbatim}
%</driver>
%<*driver|bib>
@legislation{gb.bill1987/88-66,
  author = {{Great Britain. Parliament. House of Commons}},
  year = {1988},
  title = {Local Government Finance Bill},
  address = {London},
  publisher = {HMSO},
  series = {{Bills | 1987/88}},
  number = {66}}
%</driver|bib>
%<*driver>
\end{Verbatim}
\end{bibexbox}


% UK secondary legislation

\subsubsection*{Statutory instrument}

\begin{bibexbox}<RX>{gb.hmr2012}
  \emph{The Human Medicines Regulations 2012} [Online], No.1916, United Kingdom:\@ HMSO. Available from:\@ \url{http://www.legislation.gov.uk/uksi/2012/1916/pdfs/uksi_20121916_en.pdf} [Accessed 17 April 2016].
  \tcblower
\begin{Verbatim}
%</driver>
%<*driver|bib>
@legislation{gb.hmr2012,
  entrysubtype = {secondary},
  title = {The {Human} {Medicines} {Regulations}},
  year = {2012},
  number = {No.1916},
  address = {United Kingdom},
  publisher = {HMSO},
  url = {http://www.legislation.gov.uk/uksi/2012/1916/pdfs/uksi_20121916_en.pdf},
  urldate = {2016-04-17}}
%</driver|bib>
%<*driver>
\end{Verbatim}
\end{bibexbox}

\begin{tips}\item
Use the \texttt{entrysubtype} `secondary' to put the number in the right place.
\end{tips}

% Parliamentary reports

\subsubsection*{House of Commons paper}

\begin{tips}\item
Use this form for reports of House of Commons select committees.
\end{tips}

\begin{bibexbox}<RX>{gb.hc2003/04-30}
  Great Britain.\@ Parliament.\@ House of Commons, 2004. \emph{National Savings investment deposits:\@ account 2002--2003}. (HC 2003/04, 30). London:\@ National Audit Office.
  \tcblower
\begin{Verbatim}
%</driver>
%<*driver|bib>
@report{gb.hc2003/04-30,
  author = {{Great Britain. Parliament. House of Commons}},
  year = {2004},
  title = {National {Savings} Investment Deposits},
  subtitle = {Account 2002--2003},
  address = {London},
  publisher = {National Audit Office},
  series = {HC 2003/04},
  number = {30}}
%</driver|bib>
%<*driver>
\end{Verbatim}
\end{bibexbox}


\subsubsection*{House of Lords paper}

\begin{tips}\item
Use this form for reports of House of Lords select committees.
\end{tips}

\begin{bibexbox}<RX>{gb.hl1986/87-66}
  Great Britain.\@ Parliament.\@ House of Lords, 1987. \emph{Social fund (maternity and funeral expenses) bill}. (HL 1986/87, (66)). London:\@ HMSO.
  \tcblower
\begin{Verbatim}
%</driver>
%<*driver|bib>
@report{gb.hl1986/87-66,
  author = {{Great Britain. Parliament. House of Lords}},
  year = {1987},
  title = {Social Fund (Maternity and Funeral Expenses) Bill},
  address = {London},
  publisher = {HMSO},
  series = {HL 1986/87},
  number = {66}}
%</driver|bib>
%<*driver>
\end{Verbatim}
\end{bibexbox}

\begin{hacks}\item
For joint committees, you will have to hack this slightly, putting the session
years in \texttt{series} and the HL and HC numbers in \texttt{number}.
\end{hacks}


\subsubsection*{Command paper}

\begin{bibexbox}<RX>{gb.cm6041}
  Great Britain.\@ Ministry of Defence, 2004. \emph{Delivering security in a changing world:\@ defence white paper}. (Cm.\@ 6041). London:\@ TSO.
  \tcblower
\begin{Verbatim}
%</driver>
%<*driver|bib>
@report{gb.cm6041,
  author = {{Great Britain. Ministry of Defence}},
  year = {2004},
  title = {Delivering Security in a Changing World},
  subtitle = {Defence White Paper},
  address = {London},
  publisher = {TSO},
  series = {Cm},
  number = {6041}}
%</driver|bib>
%<*driver>
\end{Verbatim}
\end{bibexbox}


\subsection{Legal references: EU legislation and reports}

% European legislation

\subsubsection*{EU regulation or directive, decision, recommendation or opinion}

\begin{bibexbox}<RX>{eu.dir1984/2003}
  Council Regulation (EC) 1984/2003 of 8 April 2003 introducing a system for the statistical monitoring of trade in bluefin tuna, swordfish and big eye tuna within the Community [2003] \emph{OJ} L295.
  \tcblower
\begin{Verbatim}
%</driver>
%<*driver|bib>
@legislation{eu.dir1984/2003,
  title = {Council {Regulation} ({EC}) 1984/2003 of 8 {April} 2003 Introducing a System for
    the Statistical Monitoring of Trade in Bluefin Tuna, Swordfish and Big Eye Tuna within
    the {Community}},
  shorttitle = {Council {Regulation} ({EC}) 1984/2003},
  year = {2003},
  journal = {OJ},
  series = {L},
  volume = {295}}
%</driver|bib>
%<*driver>
\end{Verbatim}
\end{bibexbox}

\begin{info}\item
The \texttt{shorttitle} will be used in citations instead of the full title.
\end{info}


% European reports

\subsubsection*{EU publication}

\begin{bibexbox}<RX>{ec2015gra}
  European Commission, 2015. \emph{General report on the activities of the European Union 2014}. Luxembourg:\@ Publications Office of the European Union.
  \tcblower
\begin{Verbatim}
%</driver>
%<*driver|bib>
@report{ec2015gra,
  author = {{European Commission}},
  year = {2015},
  title = {General Report on the Activities of the {European} {Union} 2014},
  address = {Luxembourg},
  publisher = {Publications Office of the European Union}}
%</driver|bib>
%<*driver>
\end{Verbatim}
\end{bibexbox}


\subsection{Legal references: case reports}

\subsubsection*{Legal case study}

\begin{bibexbox}<RX>{seldon-v-c.w.j2012}
  \emph{Seldon v.~Clarkson Wright \& Jakes}. [2012]. UKSC 16.
  \tcblower
\begin{Verbatim}
%</driver>
%<*driver|bib>
@jurisdiction{seldon-v-c.w.j2012,
  title = {Seldon v.~{Clarkson} {Wright} \& {Jakes}},
  year = {2012},
  journal = {UKSC},
  pages = {16}}
%</driver|bib>
%<*driver>
\end{Verbatim}
\end{bibexbox}

\begin{info}\item
Generally speaking, the year should be in square brackets if it is essential to the citation
(unless it is a Scottish case, in which case it is printed bare), and in parentheses if it is
not.
\end{info}
\begin{tips}\item
By default, the style assumes the year is essential if and only if a volume number is
\emph{not} provided; to override this, you can use the \key{year-essential} option:

\begin{tcolorbox}%
  [ colframe = Slate
  , colback = white
  , fontupper = \footnotesize
  ]
  \begin{Verbatim}
  options = {year-essential=true},
  \end{Verbatim}
\end{tcolorbox}

To use Scottish style for a case, you can either use the \key{scottish-style} option or
the keyword \texttt{sc}:

\begin{tcolorbox}%
  [ colframe = Slate
  , colback = white
  , fontupper = \footnotesize
  ]
  \begin{Verbatim}
  options = {scottish-style},
  keywords = {sc},
  \end{Verbatim}
\end{tcolorbox}
\end{tips}

\begin{hacks}\item
This should cover most cases, but legal references tend to enforce their own conventions
no matter what the rest of the reference list is doing,
and it is out of scope for this style to cater for every variation.
Therefore if you need a different format (e.g.\@ for an American case),
you may prefer to format the reference more-or-less by hand:

\begin{tcolorbox}%
  [ colframe = Slate
  , colback = white
  , fontupper = \footnotesize
  ]
  \begin{Verbatim}
  title = {Seldon v.~{Clarkson} {Wright} \& {Jakes}},
  sortyear = {2012},
  note = {[2012]. UKSC 16}
  \end{Verbatim}
\end{tcolorbox}
\end{hacks}

\subsubsection*{Judgment of the European Court of Justice}

\begin{bibexbox}<RX>{srl.etal-v-comm2005}
  \emph{Alessandrini Srl and others v.~Commission} (C-295/03 P) [2005] ECR I--5700.
  \tcblower
\begin{Verbatim}
%</driver>
%<*driver|bib>
@jurisdiction{srl.etal-v-comm2005,
  title = {Alessandrini {Srl} and others v.\@ {Commission}},
  number = {C-295/03 P},
  year = {2005},
  journal = {ECR},
  volume = {I},
  pages = {5700}}
%</driver|bib>
%<*driver>
\end{Verbatim}
\end{bibexbox}

\begin{tips}\item
Use the \texttt{number} field (or the non-standard \texttt{casenmuber} field)
for the case number. For Commission Decisions, use the (non-standard)
\texttt{casenumber} or (\textsf{biblatex-oscola}) \texttt{userb} field for the
Commission case number, \texttt{number} for the formal decision number, and give
`Commission' as the \texttt{institution}.
\end{tips}


\newrefcontext[sorting=nyt]
\printbibliography[heading=bibnumbered]

\section{Licence}

\begin{markdown*}{hybrid=true}
%</driver>
%<readme>
%<readme>## Licence
%<readme>
%<*driver|readme>
Copyright 2016-2018 University of Bath.

This work consists of the documented LaTeX file biblatex-bath.dtx and a Makefile.

The text files contained in this work may be distributed and/or modified
under the conditions of the [LaTeX Project Public License (LPPL)][lppl],
either version 1.3c of this license or (at your option) any later
version.

This work is `maintained' (as per LPPL maintenance status) by [Alex Ball][me].

[lppl]: http://www.latex-project.org/lppl.txt "LaTeX Project Public License (LPPL)"
[me]: https://github.com/alex-ball/bathbib "Alex Ball"
%</driver|readme>
%<*driver>
\end{markdown*}

\newpage
\lstset
  { aboveskip=0pt
  , belowskip=0pt
  , numbers=left
  , numberstyle=\color{gray}\footnotesize\itshape
  , firstnumber=last
  , basicstyle=\ttfamily\footnotesize
  , breaklines=true
  }%
\MakeShortVerb{\|}%
\DocInput{\jobname.dtx}

\end{document}
%</driver>
%<*bib>
@book{adams2009tc1,
  author = {Adams, Gomez},
  year = {2009},
  title = {Test Citation One},
  address = {London},
  publisher = {Imperial College Bookstall}}
@book{adams2014tc2,
  author = {Adams, Gomez},
  year = {2014},
  title = {Test Citation Two},
  address = {Oxford},
  publisher = {Oxford University Press}}
@book{adams2017tc3,
  author = {Adams, Gomez},
  year = {2017},
  title = {Test Citation Three},
  address = {Cambridge},
  publisher = {Cambridge University Press}}
%</bib>
%<*bbx>
% \fi
%
% \section{Implementation: bibliography style}
%
% \setcounter{lstnumber}{16}
%
% \subsection{Preliminaries}
%
% For ease of maintenance, we will patch some definitions with \textsf{xpatch}
% instead of writing out our own in full.
%
%    \begin{macrocode}
\RequirePackage{xpatch}
%    \end{macrocode}
%
% Some string analysis is required.
%
%    \begin{macrocode}
\RequirePackage{xstring}
%    \end{macrocode}
%
% Language support may be widened in future, but for now we support British and
% American English. Adapted language files have the following suffix:
%
%    \begin{macrocode}
\DeclareLanguageMappingSuffix{-bath}
%    \end{macrocode}
%
% We begin by loading the default author--year style.
%
%    \begin{macrocode}
\RequireBibliographyStyle{authoryear}
\ExecuteBibliographyOptions{%
  maxcitenames=3,maxbibnames=9999,isbn=false,giveninits=true,dashed=false,
  alldates=comp,labeldate=year}
\ExecuteBibliographyOptions[audio,video,music,movie]{%
  useeditor=false}
%    \end{macrocode}
%
% We provide some additional bibliography strings.
%
%    \begin{macrocode}
\NewBibliographyString{%
  online, hours, at, unpublished, legalchapter,
  director, performer, reader, conductor,
  directors, performers, readers, conductors,
  bydirector, byperformer, byreader, byconductor,
}
%    \end{macrocode}
%
% We allow the bibliography look more like the Bib\TeX\ default.
%
%    \begin{macrocode}
\setlength{\bibitemsep}{1em plus 0.2em minus 0.2em}
\renewcommand*{\bibfont}{\normalfont\normalsize}

%    \end{macrocode}
%
% \subsection{Name handling}
%
% Names are usually reversed. There are no spaces between initials.
%
%    \begin{macrocode}
\DeclareNameAlias{author}{family-given}
\DeclareNameAlias{editor}{family-given}
\DeclareNameAlias{bookeditor}{given-family}
\renewcommand*{\bibinitdelim}{}
%    \end{macrocode}
%
% The handling of CJK names is based on code supplied to TeX.sx by user Moewe in
% answer to \href{http://tex.stackexchange.com/a/320738/16293}{question 320738}.
%
% The CJK part is printed after the anglicized name. If the name is also
% annotated as `cjk' (see `Data Annotations' in the \textsf{biblatex} manual),
% it is always printed in family-given order with no intermediate punctuation.
%
%    \begin{macrocode}
\newbibmacro*{name:cjk-given-family}[3]{%
  \ifitemannotation{cjk}{%
    \usebibmacro{name:delim}{#2#1#3}%
    \usebibmacro{name:hook}{#2#1#3}%
    \mkbibnamefamily{#1}\isdot
    \ifdefvoid{#2}{}{\bibnamedelimd\mkbibnamegiven{#2}}%
    \ifdefvoid{#3}{}{\bibnamedelimd\mkbibnamecjk{#3}}%
  }{%
    \usebibmacro{name:delim}{#2#1#3}%
    \usebibmacro{name:hook}{#2#1#3}%
    \ifdefvoid{#2}{}{\mkbibnamegiven{#2}\isdot\bibnamedelimd}%
    \mkbibnamefamily{#1}\isdot
    \ifdefvoid{#3}{}{\bibnamedelimd\mkbibnamecjk{#3}}%
  }%
}
\newbibmacro*{name:cjk-family-given}[3]{%
  \ifitemannotation{cjk}{%
    \usebibmacro{name:delim}{#2#1#3}%
    \usebibmacro{name:hook}{#2#1#3}%
    \mkbibnamefamily{#1}\isdot
    \ifdefvoid{#2}{}{\bibnamedelimd\mkbibnamegiven{#2}}%
    \ifdefvoid{#3}{}{\bibnamedelimd\mkbibnamecjk{#3}}%
  }{%
    \usebibmacro{name:delim}{#1}%
    \usebibmacro{name:hook}{#1}%
    \mkbibnamefamily{#1}\isdot
    \ifboolexpe{%
      test {\ifdefvoid{#2}}
      and
      test {\ifdefvoid{#3}}}
      {}
      {\revsdnamepunct}%
    \ifdefvoid{#2}{}{\bibnamedelimd\mkbibnamegiven{#2}\isdot}%
    \ifdefvoid{#3}{}{\bibnamedelimd\mkbibnamecjk{#3}}%
  }%
}

\DeclareNameFormat{given-family}{%
  \ifdefvoid{\namepartcjk}{%
    \ifgiveninits{%
      \usebibmacro{name:given-family}
        {\namepartfamily}
        {\namepartgiveni}
        {\namepartprefix}
        {\namepartsuffix}%
    }{%
      \usebibmacro{name:given-family}
        {\namepartfamily}
        {\namepartgiven}
        {\namepartprefix}
        {\namepartsuffix}%
    }%
  }{%
    \ifgiveninits{%
      \usebibmacro{name:cjk-given-family}
        {\namepartfamily}
        {\namepartgiveni}
        {\namepartcjk}%
    }{%
      \usebibmacro{name:cjk-given-family}
        {\namepartfamily}
        {\namepartgiven}
        {\namepartcjk}%
    }%
  }%
  \usebibmacro{name:andothers}%
}

\DeclareNameFormat{family-given}{%
  \ifdefvoid{\namepartcjk}{%
    \ifgiveninits{%
      \usebibmacro{name:family-given}
        {\namepartfamily}
        {\namepartgiveni}
        {\namepartprefix}
        {\namepartsuffix}%
    }{%
      \usebibmacro{name:family-given}
        {\namepartfamily}
        {\namepartgiven}
        {\namepartprefix}
        {\namepartsuffix}%
    }%
  }{%
    \ifgiveninits{%
      \usebibmacro{name:cjk-family-given}
        {\namepartfamily}
        {\namepartgiveni}
        {\namepartcjk}%
    }{%
      \usebibmacro{name:cjk-family-given}
        {\namepartfamily}
        {\namepartgiven}
        {\namepartcjk}%
    }%
  }
  \usebibmacro{name:andothers}%
}

%    \end{macrocode}
%
% With videos, names in credits are printed in full.
%
%    \begin{macrocode}
\DeclareNameFormat{given-family:full}{%
  \usebibmacro{name:given-family}
    {\namepartfamily}
    {\namepartgiven}
    {\namepartprefix}
    {\namepartsuffix}%
  \usebibmacro{name:andothers}}

\renewbibmacro*{byauthor}[1][byauthor]{%
  \ifboolexpr{
    test \ifuseauthor
    or
    test {\ifnameundef{author}}
  }{}
  {\usebibmacro{bytypestrg}{author}{author}%
    \setunit{\addspace}%
    \printnames[#1]{author}}}

\renewbibmacro*{byeditor}[1][byeditor]{%
  \ifnameundef{editor}
  {}
  {\usebibmacro{bytypestrg}{editor}{editor}%
    \setunit{\addspace}%
    \printnames[#1]{editor}%
    \newunit}%
  \ifstrequal{#1}{byeditor}{%
    \usebibmacro{byeditora}%
    \usebibmacro{byeditorb}%
    \usebibmacro{byeditorc}
  }{%
    \usebibmacro{byeditora}[#1]%
    \usebibmacro{byeditorb}[#1]%
    \usebibmacro{byeditorc}[#1]}}

\newbibmacro*{byeditora}[1][byeditora]{%
  \ifnameundef{editora}
  {}
  {\usebibmacro{bytypestrg}{editora}{editor}%
    \setunit{\addspace}%
    \printnames[#1]{editora}%
    \newunit}}
\newbibmacro*{byeditorb}[1][byeditorb]{%
  \ifnameundef{editorb}
  {}
  {\usebibmacro{bytypestrg}{editorb}{editor}%
    \setunit{\addspace}%
    \printnames[#1]{editorb}%
    \newunit}}
\newbibmacro*{byeditorc}[1][byeditorc]{%
  \ifnameundef{editorc}
  {}
  {\usebibmacro{bytypestrg}{editorc}{editor}%
    \setunit{\addspace}%
    \printnames[#1]{editorc}%
    \newunit}}

\renewbibmacro*{bytranslator}[1][bytranslator]{%
  \ifnameundef{translator}
  {}
  {\setunit{\addspace}%
    \printtext[parens]{%
    \printnames[#1]{translator}%
    \newunit
    \bibstring{translator}%
    \clearname{translator}}}}

\renewbibmacro*{byeditor+others}[1][byeditor]{%
  \ifnameundef{editor}
  {}
  {\usebibmacro{byeditor+othersstrg}%
    \setunit{\addspace}%
    \printnames[#1]{editor}%
    \clearname{editor}%
    \newunit}%
  \ifstrequal{#1}{byeditor}{%
    \usebibmacro{byeditorx}%
    \usebibmacro{bytranslator+others}%
  }{%
    \usebibmacro{byeditora}[#1]%
    \usebibmacro{byeditorb}[#1]%
    \usebibmacro{byeditorc}[#1]%
    \usebibmacro{bytranslator+others}[#1]}}

\renewbibmacro*{bytranslator+others}[1][bytranslator]{%
  \ifnameundef{translator}
  {\usebibmacro{withothers}}
  {\setunit{\addspace}%
    \printtext[parens]{%
    \printnames[bytranslator]{translator}%
    \newunit
    \bibstring{translator}%
    \clearname{translator}%
    \newunit
    \usebibmacro{withothers}}}}

%    \end{macrocode}
%
% With collections, editors appear in natural order between `In' and the title,
% followed by `ed.'
%
%    \begin{macrocode}
\newbibmacro*{bookeditor}{%
  \ifnameundef{editor}{}{%
    \printnames[bookeditor]{editor}%
    \setunit*{\addcomma\space}%
    \usebibmacro{editor+othersstrg}%
    \clearname{editor}%
  }}

%    \end{macrocode}
%
% \subsection{Titles}
%
% Most titles are set in italics, but some are set roman and unquoted.
%
%    \begin{macrocode}
\DeclareFieldFormat{sentencecase}{\MakeSentenceCase*{#1}}
\DeclareFieldFormat{midsentencecase}{\MakeSentenceCase*{{}#1}}
\DeclareFieldFormat{title}{\mkbibemph{#1}}
\DeclareFieldFormat
  [article,inbook,incollection,inproceedings]%
  {title}{#1}
\DeclareFieldFormat
  [patent,thesis]%
  {title}{\mkbibemph{#1}}
\DeclareFieldFormat
  [unpublished]%
  {title}{\iffieldundef{booktitle}{\mkbibemph{#1}}{#1}}

%    \end{macrocode}
%
% Online resources are clearly tarred and feathered with an `[Online]' label.
% The \texttt{isonline} macro prints this label if the resource has a URL and
% does nothing otherwise. We add a safeguard to stop it being used repeatedly.
%
%    \begin{macrocode}
\newtoggle{bbx:onlineshown}
\newbibmacro*{isonline}{%
  \ifboolexpr{(
      test {\iffieldundef{url}}
      and
      not test {\ifentrytype{online}}
    ) or
    togl {bbx:onlineshown}
  }{}{%
    \bibstring[\mkbibbrackets]{online}%
    \toggletrue{bbx:onlineshown}}}

%    \end{macrocode}
%
% The |titleaddon| field follows the title after a space and wrapped in
% brackets. Version 1 of this style encouraged people to supply the brackets
% manually, so we strip them off if they have been supplied.
%
%    \begin{macrocode}
\DeclareFieldFormat{titleaddon}{\mkbibbrackets{%
  \IfBeginWith{#1}{[}{%
    \IfEndWith{#1}{]}{%
      \StrBetween{#1}{[}{]}%
    }{#1}%
  }{#1}%
}}

%    \end{macrocode}
%
% We need to supply a new |title| macro. The standard version hard-codes the
% case used (we use sentence case, not title case) and puts default unit
% punctuation between the |title| and |titleaddon| fields. We also need to add
% conditional code for printing the automatic `[Online]' label.
%
%    \begin{macrocode}
\renewbibmacro*{title}{%
  \printtext{%
    \ifboolexpr{
      test {\iffieldundef{title}}
      and
      test {\iffieldundef{subtitle}}
    }{}{%
      \printtext[title]{%
        \printfield[sentencecase]{title}%
        \setunit{\subtitlepunct}%
        \printfield[midsentencecase]{subtitle}%
        \setunit{\addspace}%
      }%
      \printfield{version}%
      \clearfield{version}%
      \setunit*{\addspace}%
      \printfield{titleaddon}%
      \ifboolexpr{
        test {\iffieldundef{journaltitle}}
        and
        test {\iffieldundef{booktitle}}
        and
        test {\iffieldundef{library}}
        and
        not test {\ifentrytype{software}}
      }{%
        \setunit*{\addspace}%
        \usebibmacro{isonline}%
      }{}%
    }%
  }%
}

%    \end{macrocode}
%
% Similar changes are needed for the |booktitle| and |maintitle| macros.
%
%    \begin{macrocode}
\renewbibmacro*{booktitle}{%
  \ifboolexpr{
    test {\iffieldundef{booktitle}}
    and
    test {\iffieldundef{booksubtitle}}
  }{}{%
    \printtext[booktitle]{%
      \printfield[sentencecase]{booktitle}%
      \setunit{\subtitlepunct}%
      \printfield[midsentencecase]{booksubtitle}%
      \setunit{\addspace}%
    }%
    \printfield{booktitleaddon}
    \setunit*{\addspace}%
    \usebibmacro{isonline}%
  }%
}

\renewbibmacro*{maintitle}{%
  \ifboolexpr{
    test {\iffieldundef{maintitle}}
    and
    test {\iffieldundef{mainsubtitle}}
  }{}{
    \printtext[maintitle]{%
      \printfield[sentencecase]{maintitle}%
      \setunit{\subtitlepunct}%
      \printfield[midsentencecase]{mainsubtitle}%
      \setunit{\addspace}%
    }%
    \printfield{maintitleaddon}%
  }%
}

%    \end{macrocode}
%
% Subtitles are set off with a colon
%
%    \begin{macrocode}
\renewcommand*{\subtitlepunct}{\addcolon\space}

%    \end{macrocode}
%
% In the standard author--year styles, if a title is promoted to the head of
% a reference (due to missing authors/editors), the subtitle and titleaddon
% fields are discarded. We want to show the subtitle directly after the title,
% and show titleaddon after the |date+extradate| macro.
%
%    \begin{macrocode}
\providetoggle{bbx:labelistitle}
\renewbibmacro*{labeltitle}{%
  \iffieldundef{label}{%
    \ifboolexpr{
      test {\iffieldundef{title}}
      and
      test {\iffieldundef{subtitle}}
    }{}{%
      \printtext[title]{%
        \printfield[sentencecase]{title}%
        \setunit{\subtitlepunct}%
        \printfield[midsentencecase]{subtitle}}%
      \clearfield{title}\clearfield{subtitle}%
      \toggletrue{bbx:labelistitle}}%
  }{%
    \printfield{label}%
  }%
}
\DeclareDelimFormat{yearlabeltitleaddondelim}{\addspace}
\newbibmacro*{labeltitleaddon}{%
  \iftoggle{bbx:labelistitle}{%
    \setunit{\printdelim{yearlabeltitleaddondelim}}%
    \printfield{version}%
    \clearfield{version}%
    \setunit*{\addspace}%
    \printfield{titleaddon}%
    \clearfield{titleaddon}%
    \ifboolexpr{
      test {\iffieldundef{journaltitle}}
      and
      test {\iffieldundef{booktitle}}
      and (
        test {\iffieldundef{library}}
        or
        test {\ifentrytype{image}}
      ) and
      not test {\ifentrytype{software}}
    }{%
      \setunit*{\addspace}%
      \usebibmacro{isonline}%
    }{}%
  }{}%
}
\xapptobibmacro{author}{\usebibmacro{labeltitleaddon}}{}{}
\xapptobibmacro{bbx:editor}{\usebibmacro{labeltitleaddon}}{}{}
\xapptobibmacro{bbx:translator}{\usebibmacro{labeltitleaddon}}{}{}

%    \end{macrocode}
%
% \subsection{Dates}
%
% If the main publication date is missing, we fall back immediately to `n.d.\@'
% rather than use URL date or anything like that. However, for some resources
% (ISO standards, Acts of Parliament), the date is part of the label and should
% not be repeated, so we declare an option for removing the `n.d.' Unless
% already set, the option is inserted if |sortyear| is used.
%
%    \begin{macrocode}
\DeclareLabeldate{%
  \field{date}
  \field{year}
  \literal{nodate}
}
\newtoggle{bbx:nonodate}
\DeclareBibliographyOption[boolean]{nonodate}[true]{%
  \settoggle{bbx:nonodate}{#1}}
\DeclareTypeOption[boolean]{nonodate}[true]{%
  \settoggle{bbx:nonodate}{#1}}
\DeclareEntryOption[boolean]{nonodate}[true]{%
  \settoggle{bbx:nonodate}{#1}}
%    \end{macrocode}
%
% Unless already set, the \key{nonodate} option is inserted if |sortyear| is
% used. We accomplish this with source maps; the first one works where options
% (not including |nonodate|) have been set, the second where no options have
% been set.
%
%    \begin{macrocode}
\DeclareStyleSourcemap{
  \maps[datatype=bibtex]{
    \map[overwrite=true]{
      \step[notmatch=\regexp{nonodate}, fieldsource=options, final]
      \step[fieldsource=sortyear, final]
      \step[fieldset=options, append, fieldvalue={,nonodate}]
    }
    \map[overwrite=true]{
      \step[notfield=options, final]
      \step[fieldsource=sortyear, final]
      \step[fieldset=options, fieldvalue={nonodate}]
    }
  }
}

%    \end{macrocode}
%
% The punctuation before the label year is controlled by the following
% commands, and should be a comma. The exceptions to this are |legislation|
% entries, where there should just be a space, and English/Welsh legal case
% reports, where there should be a period.
%
%    \begin{macrocode}
\DeclareDelimFormat{nameyeardelim}{\addcomma\space}
\DeclareDelimFormat[parencite,bib,biblist]{nameyeardelim}{\addcomma\space}
\newcommand{\dononameyeardelim}{%
  \ifentrytype{legislation}{%
    \addspace
  }{%
    \ifentrytype{jurisdiction}{%
      \ifboolexpr{
        togl {bbx:eu-oj}
        or
        test {\iffieldequalstr{journaltitle}{ECR}}
        or
        test {\iffieldequalstr{type}{ECR}}
      }{%
        \addspace
      }{%
        \ifboolexpr{
          test {\ifkeyword{sc}}
          or
          togl {bbx:scotstyle}
        }{%
          \addcomma\space
        }{%
          \addperiod\space}}%
    }{%
      \addcomma\space}}}
\DeclareDelimFormat{nonameyeardelim}{\dononameyeardelim}
\DeclareDelimFormat[bib,biblist]{nonameyeardelim}{\dononameyeardelim}
\DeclareDelimFormat[parencite]{nonameyeardelim}{%
  \ifboolexpr{
    test {\ifentrytype{jurisdiction}}
    or
    test {\ifentrytype{legislation}}
  }{\addspace}{\addcomma\space}}

%    \end{macrocode}
%
% The punctuation after the label year is controlled by the following
% commands, and should be a comma. The exceptions to this are |legislation|
% and |jurisdiction| entries, where there should just be a space if the label
% is the title.
%
%    \begin{macrocode}
\DeclareDelimFormat{nametitledelim}{%
  \ifboolexpr{
    (
      test {\ifentrytype{jurisdiction}}
      or
      test {\ifentrytype{legislation}}
    ) and
    togl {bbx:labelistitle}
  }{\addspace}{\addcomma\space}}
\DeclareDelimFormat[bib,biblist]{nametitledelim}{%
  \ifboolexpr{
    (
      test {\ifentrytype{jurisdiction}}
      or
      test {\ifentrytype{legislation}}
    ) and
    togl {bbx:labelistitle}
  }{\addspace}{\labelnamepunct}}


%    \end{macrocode}
%
% We allow the date macro to print the time as well.
%
%    \begin{macrocode}
\renewbibmacro*{date}{%
  \printdate
  \setunit*{\bibdatetimesep}
  \printtime
}
\DeclareFieldFormat{time}{#1~\bibstring{hours}}

%    \end{macrocode}
%
% We provide a new date merging option that moves the year but leaves the
% month and day in place, and set this as the default. Note that dates are
% preceded by a comma, and are given bare rather than in parentheses.
%
% We start by patching the error message of the existing option code, then
% add a per-type version of the option.
%
%    \begin{macrocode}
\xpatchcmd{\KV@blx@opt@pre@mergedate}{%
  'true' (=compact)%
}{%
  'year', 'true' (=year)%
}{}{}
\DeclareTypeOption[boolean]{mergedate}[true]{%
  \ifcsdef{bbx@opt@mergedate@#1}{%
    \csuse{bbx@opt@mergedate@#1}%
  }{%
    \PackageError{biblatex}
       {Invalid option 'mergedate=#1'}
       {Valid values are 'maximum', 'compact', 'basic', 'minimum',\MessageBreak
        'year', 'true' (=year), and 'false'.}}}
%    \end{macrocode}
%
% We define a configurable field format for date labels to replace the
% hard-coded parentheses in the options from the standard author--year style.
% We also amend the logic for printing the label date so the |nonodate| option
% is respected.
%
%    \begin{macrocode}
\DeclareFieldFormat{datelabel}{#1}
\xpatchcmd{\bbx@opt@mergedate@maximum}{%
  \iffieldundef{labelyear}%
}{%
  \ifboolexpr{
    togl {bbx:nonodate}
    and
    not test {\iflabeldateisdate}}%
}{}{}
\xpatchcmd{\bbx@opt@mergedate@maximum}{%
  \printtext[parens]%
}{%
  \printtext[datelabel]%
}{}{}
\xpatchcmd{\bbx@opt@mergedate@compact}{%
  \iffieldundef{labelyear}%
}{%
  \ifboolexpr{
    togl {bbx:nonodate}
    and
    not test {\iflabeldateisdate}}%
}{}{}
\xpatchcmd{\bbx@opt@mergedate@compact}{%
  \printtext[parens]%
}{%
  \printtext[datelabel]%
}{}{}
\xpatchcmd{\bbx@opt@mergedate@basic}{%
  \iffieldundef{labelyear}%
}{%
  \ifboolexpr{
    togl {bbx:nonodate}
    and
    not test {\iflabeldateisdate}}%
}{}{}
\xpatchcmd{\bbx@opt@mergedate@basic}{%
  \printtext[parens]%
}{%
  \printtext[datelabel]%
}{}{}
\xpatchcmd{\bbx@opt@mergedate@minimum}{%
  \iffieldundef{labelyear}%
}{%
  \ifboolexpr{
    togl {bbx:nonodate}
    and
    not test {\iflabeldateisdate}}%
}{}{}
\xpatchcmd{\bbx@opt@mergedate@minimum}{%
  \printtext[parens]%
}{%
  \printtext[datelabel]%
}{}{}
\xpatchcmd{\bbx@opt@mergedate@false}{%
  \iffieldundef{labelyear}%
}{%
  \ifboolexpr{
    togl {bbx:nonodate}
    and
    not test {\iflabeldateisdate}}%
}{}{}
\xpatchcmd{\bbx@opt@mergedate@false}{%
  \printtext[parens]%
}{%
  \printtext[datelabel]%
}{}{}

%    \end{macrocode}
%
% Lastly, here is our new (default) option, which always merges the year, and
% only the year, with the label date. Other date and time componenents are
% displayed later in the reference. Therefore we clear the year from the date
% used for the label, but leave the month and day alone.
%
%    \begin{macrocode}
\def\bbx@opt@mergedate@year{%
  \renewbibmacro*{date+extradate}{%
    \iffieldundef{labelyear}{}{%
      \ifboolexpr{
        togl {bbx:nonodate}
        and
        not test {\iflabeldateisdate}
      }{}{%
        \printtext[datelabel]{\printlabeldateextra}%
      }%
      \iflabeldateisdate{%
        \clearfield{year}%
      }{}}}
  \renewbibmacro*{issue+date}{%
    \ifboolexpr{
      test {\iffieldundef{issue}}
      and
      test {\iffieldundef{month}}
    }{}{%
      \ifboolexpr{(
        test {\iffieldundef{volume}}
        and
        test {\iffieldundef{number}}
        ) and
        test {\iffieldundef{eid}}
      }{%
        \newunit
        \printfield{issue}%
      }{%
        \printtext[parens]{%
          \printfield{issue}%
        }%
      }
      \setunit{\addcomma\space}%
      \printdate
    }%
    \newunit
  }%
}%

\def\bbx@opt@mergedate@true{\bbx@opt@mergedate@year}
\ExecuteBibliographyOptions{mergedate}

%    \end{macrocode}
%
% The problem with moving years but leaving months and days behind is that the
% regular date range macros in |biblatex.sty| do nothing at all if no year
% is printed. We therefore need to patch the macros with extra routines for
% printing year-free date ranges: the rather extravagently named
% |\mknoyeardaterangefull| and |\mknoyeardaterangetrunc|.
%
%    \begin{macrocode}
\newrobustcmd*{\mknoyeardaterangefull}[2]{%
  \iffieldundef{#2month}{}{%
    \datecircaprint
    \printtext[#2date]{%
    \iffieldundef{#2season}{%
      \csuse{mkbibdate#1}{}{#2month}{#2day}%
      \blx@printtime{#2}{}%
    }{%
      \csuse{mkbibseasondate#1}{}{#2season}}%
    \dateuncertainprint
    \iffieldundef{#2endmonth}{}{%
      \iffieldequalstr{#2endmonth}{}{%
        \mbox{\bibdaterangesep}%
      }{%
        \bibdaterangesep
        \enddatecircaprint
        \iffieldundef{#2season}{%
          \csuse{mkbibdate#1}{}{#2endmonth}{#2endday}%
          \blx@printtime{#2}{end}%
        }{%
          \csuse{mkbibseasondate#1}{}{#2endseason}}%
        \enddateuncertainprint}}}}}
%    \end{macrocode}
%
% There is a potential problem for |\mknoyeardaterangetrunc|, in that if
% the year and endyear are missing, it cannot tell if they are the same, so
% if the months are the same but the years are different, the range would
% be erroneously compressed. However, the only reason the year should
% be missing is that it is in the label, so we test |labelyear| instead.
%
%    \begin{macrocode}
\newrobustcmd*{\mknoyeardaterangetrunc}[2]{%
  \iffieldundef{#2month}{}{%
    \datecircaprint
    \printtext[#2date]{%
      \iffieldundef{#2season}{%
        \ifboolexpr{
          test {\iffieldsequal{labelyear}{labelendyear}}
          and
          test {\iffieldsequal{#2month}{#2endmonth}}
        }{%
          \csuse{mkbibdate#1}{}{}{#2day}%
        }{%
          \csuse{mkbibdate#1}{}{#2month}{#2day}}%
      }{%
        \csuse{mkbibseasondate#1}{}{#2season}}%
      \dateuncertainprint
      \iffieldundef{#2endmonth}{}{%
        \iffieldequalstr{#2endmonth}{}{%
          \mbox{\bibdaterangesep}%
        }{%
          \bibdaterangesep
          \enddatecircaprint
          \iffieldundef{#2season}{%
            \csuse{mkbibdate#1}{}{#2endmonth}{#2endday}%
          }{%
            \csuse{mkbibseasondate#1}{}{#2endseason}}%
          \enddateuncertainprint}}}}}
%    \end{macrocode}
%
% Now we patch the four date range commands. The |extra| commands,
% which print disambiguating labels as well, should only print those
% labels if the year is present, so they can use the same non-year
% date range functions as the non-|extra| commands.
%
%    \begin{macrocode}
\xpatchcmd{\mkdaterangefull}{%
  \iffieldundef{#2year} {\blx@nounit}%
}{%
  \iffieldundef{#2year} {\mknoyeardaterangefull{#1}{#2}}%
}{}{\wlog{WARNING: biblatex-bath failed to patch mkdaterangefull}}
\xpatchcmd{\mkdaterangetrunc}{%
  \iffieldundef{#2year} {\blx@nounit}%
}{%
  \iffieldundef{#2year} {\mknoyeardaterangetrunc{#1}{#2}}%
}{}{\wlog{WARNING: biblatex-bath failed to patch mkdaterangetrunc}}
\xpatchcmd{\mkdaterangefullextra}{%
  \iffieldundef{#2year} {\blx@nounit}%
}{%
  \iffieldundef{#2year} {\mknoyeardaterangefull{#1}{#2}}%
}{}{\wlog{WARNING: biblatex-bath failed to patch mkdaterangefullextra}}
\xpatchcmd{\mkdaterangetruncextra}{%
  \iffieldundef{#2year} {\blx@nounit}%
}{%
  \iffieldundef{#2year} {\mknoyeardaterangetrunc{#1}{#2}}%
}{}{\wlog{WARNING: biblatex-bath failed to patch mkdaterangetruncextra}}

%    \end{macrocode}
%
% \subsection{Versions}
%
% Versions are printed with `v.\@' in parentheses.
%
%    \begin{macrocode}
\DeclareFieldFormat{version}{\mkbibparens{\bibstring{version}#1}}
%    \end{macrocode}
%
% \subsection{Types}
%
% Types are always printed longhand.
%
%    \begin{macrocode}
\DeclareFieldFormat{type}{\ifbibstring{#1}{\biblstring{#1}}{#1}}
%    \end{macrocode}
%
% \subsection{Events}
%
% Events are printed as date then venue with no intermediate punctuation.
%
%    \begin{macrocode}
\renewbibmacro*{event+venue+date}{%
  \printfield{eventtitle}%
  \setunit*{\addspace}%
  \printfield{eventtitleaddon}%
  \ifboolexpr{
    test {\iffieldundef{venue}}
    and
    test {\iffieldundef{eventyear}}
  }
    {}
    {\setunit{\addcomma\space}%
     \printeventdate
     \setunit*{\addcomma\space}%
     \printfield{venue}%
     \newunit}}

%    \end{macrocode}
%
% \subsection{Publishers}
%
% We patch the secondary publication macros so they will use the
% publisher list instead, if provided.
%
%    \begin{macrocode}
\letbibmacro{plain:institution+location+date}{institution+location+date}
\renewbibmacro*{institution+location+date}{%
  \iflistundef{publisher}{%
    \usebibmacro{plain:institution+location+date}%
  }{%
    \printlist{institution}%
    \newunit
    \usebibmacro{publisher+location+date}}}
\letbibmacro{plain:organization+location+date}{organization+location+date}
\renewbibmacro*{organization+location+date}{%
  \iflistundef{publisher}{%
    \usebibmacro{plain:organization+location+date}%
  }{%
    \printlist{organization}%
    \newunit
    \usebibmacro{publisher+location+date}}}

%    \end{macrocode}
%
% We use the \texttt{library} field for databases and preprint repositories.
%
%    \begin{macrocode}
\DeclareFieldFormat{library}{\mkbibemph{#1}}
\newbibmacro*{library}{%
  \iffieldundef{library}{}{%
    \printfield{library}%
    \setunit*{\addspace}%
    \usebibmacro{isonline}%
  }%
}

%    \end{macrocode}
%
% \subsection{Page numbers}
%
%    \begin{macrocode}
\renewcommand*{\ppspace}{}
\DeclareNumChars{ab}
%    \end{macrocode}
%
% \subsection{URLs}
%
% URLs are prefaced by a `from' statement, and the URL date is enclosed in
% brackets rather than parentheses.
%
%    \begin{macrocode}
\DeclareFieldFormat{url}{\bibsentence\bibstring{urlfrom}\addcolon\space\url{#1}}
\DeclareFieldFormat{doi}{\bibsentence\bibstring{urlfrom}\addcolon\space\url{https://doi.org/#1}}
\DeclareFieldFormat{urldate}{\mkbibbrackets{\bibstring{urlseen}\space#1}}
\renewbibmacro*{doi+eprint+url}{%
  \iftoggle{bbx:eprint}
    {\usebibmacro{eprint}}
    {}%
  \newunit\newblock
  \iftoggle{bbx:url}
    {\usebibmacro{url+urldate}}
    {}}
\renewbibmacro*{url}{%
  \iffieldundef{doi}%
    {\printfield{url}}%
    {\printfield{doi}}%
}

%    \end{macrocode}
%
% \subsection{Articles}
%
% Compared with the standard styles, the main differences in the driver are the
% omission of `in' and the position of the publication state.
%
%    \begin{macrocode}
\xpatchbibdriver{article}{%
  \usebibmacro{in:}\usebibmacro{journal+issuetitle}%
}{%
  \usebibmacro{journal+issuetitle}%
}{}{}
\xpatchbibdriver{article}{%
  \usebibmacro{addendum+pubstate}%
}{%
  \printfield{addendum}%
}{}{}
%    \end{macrocode}
%
% The journal title is in sentence case rather than title case.
%
%    \begin{macrocode}
\xpatchbibmacro{journal}{%
  \printfield[titlecase]{journaltitle}%
}{%
  \printfield[sentencecase]{journaltitle}%
}{}{}
\xpatchbibmacro{journal}{%
  \printfield[titlecase]{journalsubtitle}%
}{%
  \printfield[sentencecase]{journalsubtitle}%
}{}{}
%    \end{macrocode}
%
% The journal title is followed by a comma. The issue number is separated from
% the volume by parentheses rather than a dot. The pubstate is in the volume
% position.
%
%    \begin{macrocode}
\renewbibmacro*{journal+issuetitle}{%
  \usebibmacro{journal}%
  \setunit*{\addspace}%
  \usebibmacro{isonline}%
  \setunit*{\addcomma\space}%
  \iffieldundef{series}
    {}
    {\newunit
     \printfield{series}%
     \setunit{\addcomma\space}}%
  \usebibmacro{volume+number+eid}%
  \setunit{\addspace}%
  \usebibmacro{issue+date}%
  \setunit{\addcolon\space}%
  \usebibmacro{issue}%
  \setunit{\addcomma\space}%
  \printfield{pubstate}%
  \newunit}
\renewbibmacro*{volume+number+eid}{%
  \printfield{volume}%
  \printfield[parens]{number}%
  \setunit{\addcomma\space}%
  \printfield{eid}}

%    \end{macrocode}
%
% \subsection{Books}
%
% Compared with the standard styles, the main difference is that the series is
% separated by a comma rather than a period. This also affects proceedings and
% certain forms of grey literature.
%
%    \begin{macrocode}
\xpatchbibdriver{book}{%
  \newunit\newblock
  \usebibmacro{series+number}%
}{%
  \setunit{\addcomma\space}%
  \usebibmacro{series+number}%
}{}{}

\xpatchbibdriver{collection}{%
  \newunit\newblock
  \usebibmacro{series+number}%
}{%
  \setunit{\addcomma\space}%
  \usebibmacro{series+number}%
}{}{}

\xpatchbibdriver{inbook}{%
  \newunit\newblock
  \usebibmacro{series+number}%
}{%
  \setunit{\addcomma\space}%
  \usebibmacro{series+number}%
}{}{}

\xpatchbibdriver{incollection}{%
  \newunit\newblock
  \usebibmacro{series+number}%
}{%
  \setunit{\addcomma\space}%
  \usebibmacro{series+number}%
}{}{}

\xpatchbibdriver{inproceedings}{%
  \newunit\newblock
  \usebibmacro{series+number}%
}{%
  \setunit{\addcomma\space}%
  \usebibmacro{series+number}%
}{}{}

\xpatchbibdriver{proceedings}{%
  \newunit\newblock
  \usebibmacro{series+number}%
}{%
  \setunit{\addcomma\space}%
  \usebibmacro{series+number}%
}{}{}

%    \end{macrocode}
%
% \subsection{Works in collections}
%
% Compared with the standard styles, the main difference is that the editors
% precede the booktitle.
%
%    \begin{macrocode}
\xpatchbibdriver{incollection}{%
  \usebibmacro{in:}%
  \usebibmacro{maintitle+booktitle}%
  \newunit\newblock
  \usebibmacro{byeditor+others}%
}{%
  \ifnameundef{editor}{}{\usebibmacro{in:}}%
  \usebibmacro{bookeditor}%
  \newunit\newblock
  \usebibmacro{maintitle+booktitle}%
  \usebibmacro{byeditor+others}%
}{}{}

\xpatchbibdriver{inproceedings}{%
  \usebibmacro{in:}%
  \usebibmacro{maintitle+booktitle}%
  \newunit\newblock
  \usebibmacro{event+venue+date}%
  \newunit\newblock
  \usebibmacro{byeditor+others}%
}{%
  \ifnameundef{editor}{}{\usebibmacro{in:}}%
  \usebibmacro{bookeditor}%
  \newunit\newblock
  \usebibmacro{maintitle+booktitle}%
  \usebibmacro{byeditor+others}%
  \newunit
  \usebibmacro{event+venue+date}%
}{}{}

%    \end{macrocode}
%
% \subsection{Online works}
%
% Compared with the standard styles, the main difference is that the
% organization's address is printed.
%
%    \begin{macrocode}
\xpatchbibdriver{online}{%
  \printlist{organization}%
}{%
  \usebibmacro{library}%
  \newunit\newblock
  \usebibmacro{organization+location+date}%
}{}{}

%    \end{macrocode}
%
% \subsection{Reports}
%
% Compared with the standard styles, the main differences are that we use a
% special macro for printing the |type|, |series| and |number| in parentheses,
% and we support the |library| field.
%
%    \begin{macrocode}
\DeclareFieldFormat{forceparens}{(#1)}
\newbibmacro{series+type+number}{%
  \ifboolexpr{
    test {\iffieldundef{series}}
    and
    test {\iffieldundef{type}}
    and
    test {\iffieldundef{number}}
  }{}{%
    \printtext[parens]{%
      \printfield{series}%
      \IfStrEqCase{\thefield{series}}{%
        {C}{\printunit*{\adddot\space}}%
        {Cd}{\printunit*{\adddot\space}}%
        {Cmd}{\printunit*{\adddot\space}}%
        {Cmnd}{\printunit*{\adddot\space}}%
        {Cm}{\printunit*{\adddot\space}}%
      }{%
        \setunit*{\addcomma\space}}%
      \printfield{type}%
      \setunit*{\addspace}%
      \IfBeginWith{\thefield{series}}{HL}{%
        \printfield[forceparens]{number}%
      }{%
        \printfield{number}%
      }}}}

\DeclareBibliographyDriver{report}{%
  \usebibmacro{bibindex}%
  \usebibmacro{begentry}%
  \usebibmacro{author}%
  \setunit{\printdelim{nametitledelim}}\newblock
  \usebibmacro{title}%
  \newunit
  \printlist{language}%
  \newunit\newblock
  \usebibmacro{byauthor}%
  \newunit\newblock
  \usebibmacro{series+type+number}%
  \newunit\newblock
  \printfield{note}%
  \newunit\newblock
  \usebibmacro{institution+location+date}%
  \newunit\newblock
  \usebibmacro{chapter+pages}%
  \newunit
  \printfield{pagetotal}%
  \newunit\newblock
  \iftoggle{bbx:isbn}
    {\printfield{isrn}}
    {}%
  \newunit\newblock
  \usebibmacro{library}%
  \newunit\newblock
  \usebibmacro{doi+eprint+url}%
  \newunit\newblock
  \usebibmacro{addendum+pubstate}%
  \setunit{\bibpagerefpunct}\newblock
  \usebibmacro{pageref}%
  \newunit\newblock
  \iftoggle{bbx:related}
    {\usebibmacro{related:init}%
     \usebibmacro{related}}
    {}%
  \usebibmacro{finentry}}

%    \end{macrocode}
%
% \subsection{Manuals}
%
% The \texttt{manual} driver is useful for a variety of technical publications,
% such as standards, patents and maps. (Patents have their own driver, in fact,
% but we set things up so the \texttt{manual} driver can be used instead if
% needed for compatibility.)
%
%    \begin{macrocode}
\newbibmacro{manual:series+type+number}{%
  \iffieldundef{series}{%
    \newunit\newblock
    \printfield{type}%
    \setunit{\addspace}%
    \printfield{number}%
  }{%
    \setunit{\addcomma\space}%
    \usebibmacro{series+number}%
    \newunit\newblock
    \printfield{type}%
  }%
}
\xpatchbibdriver{manual}{%
  \newunit\newblock
  \usebibmacro{series+number}%
  \newunit\newblock
  \printfield{type}%
}{%
  \usebibmacro{manual:series+type+number}%
}{}{}
\xpatchbibdriver{manual}{%
  \printlist{organization}%
  \newunit
  \usebibmacro{publisher+location+date}%
}{%
  \usebibmacro{organization+location+date}%
}{}{}

%    \end{macrocode}
%
% \subsection{Standards}
%
% A special quirk with standards is that they are often best known by their
% number, so it comes at the head of the reference if no authors are specified.
%
% To achieve this cleanly with correct sorting, we copy the number into
% |sortkey| as a simple fix (providing neither |author| nor |sortkey| are
% already specified). We then move the number to |label| so that citations
% come out correctly and the date (if given) is put in the right place.
%
%    \begin{macrocode}
\DeclareStyleSourcemap{%
  \maps[datatype=bibtex]{%
    \map[overwrite=false]{
      \pertype{standard}
      \step[notfield=author,
        fieldsource=number,
        final]
      \step[fieldset=sortkey,
        origfieldval]
    }
    \map[overwrite=false]{
      \pertype{standard}
      \step[notfield=author,
      fieldsource=number,
      fieldtarget=label]
    }
  }}
\ExecuteBibliographyOptions[standard]{useeditor=false}

%    \end{macrocode}
%
% \subsection{Audiovisual materials}
%
% The \texttt{audio} and \texttt{video} drivers are based on \texttt{misc}.
%
%    \begin{macrocode}
\DeclareBibliographyDriver{audio}{%
  \usebibmacro{bibindex}%
  \usebibmacro{begentry}%
  \usebibmacro{author/editor+others/translator+others}%
  \setunit{\printdelim{nametitledelim}}\newblock
  \usebibmacro{title}%
  \newunit
  \printlist{language}%
  \newunit\newblock
  \printfield{type}%
  \newunit\newblock
  \usebibmacro{byauthor}[given-family:full]%
  \newunit\newblock
  \usebibmacro{byeditor+others}[given-family:full]%
  \newunit\newblock
  \printfield{howpublished}%
  \newunit\newblock
  \printfield{note}%
  \newunit\newblock
  \usebibmacro{publisher+location+date}%
  \newunit\newblock
  \usebibmacro{doi+eprint+url}%
  \newunit\newblock
  \usebibmacro{addendum+pubstate}%
  \setunit{\bibpagerefpunct}\newblock
  \usebibmacro{pageref}%
  \newunit\newblock
  \iftoggle{bbx:related}
    {\usebibmacro{related:init}%
     \usebibmacro{related}}
    {}%
  \usebibmacro{finentry}}
\DeclareBibliographyAlias{movie}{audio}
\DeclareBibliographyAlias{music}{audio}
\DeclareBibliographyAlias{video}{audio}
%    \end{macrocode}
%
% \subsection{Case reports}
%
% We provide a test to see if a work was published in the Official Journal of
% the EU; if so, the title is not italic (unless reporting a decision of the
% European Commission).
%
%    \begin{macrocode}
\newtoggle{bbx:eu-oj}
\newbibmacro*{eucheck}{%
  \IfBeginWith{\thefield{journaltitle}}{OJ}{%
    \toggletrue{bbx:eu-oj}%
  }{}}
\DeclareFieldFormat[jurisdiction,legislation]{title}{%
  \ifboolexpr{
    togl{bbx:eu-oj}
    and
    not test {\iffieldequalstr{type}{Commission Decision}}
  }{#1}{\mkbibemph{#1}}}

%    \end{macrocode}
%
% The year is printed in square brackets, unless the case is Scottish, in which
% case the year is printed bare, or a volume number is present, in which case the
% year is printed in parentheses. With ECR and OJ cases, however, the year is
% always printed in square brackets despite the presence of a volume number.
% We provide a \key{year-essential} option for forcing the brackets/parentheses
% choice, and \key{scottish-style} for activating Scottish style.
%
%    \begin{macrocode}
\newtoggle{bbx:scotstyle}
\DeclareEntryOption[boolean]{scottish-style}[true]{%
  \settoggle{bbx:scotstyle}{#1}}
\newtoggle{bbx:year-essential}
\DeclareEntryOption[boolean]{year-essential}[true]{%
  \settoggle{bbx:year-essential}{#1}}
\DeclareFieldFormat[jurisdiction]{datelabel}{%
  \ifboolexpr{
    test {\iffieldundef{volume}}
    or
    togl {bbx:year-essential}
    or
    togl {bbx:eu-oj}
    or
    test {\iffieldequalstr{journaltitle}{ECR}}
  }{%
    \ifboolexpr{
      test {\ifkeyword{sc}}
      or
      togl {bbx:scotstyle}
    }{%
      #1%
    }{%
      \mkbibbrackets{#1}}%
  }{%
    \mkbibparens{#1}}}

%    \end{macrocode}
%
% A case number should normally go after the title and before the year, but
% Version 1 of this style used |number| with |type| at the end of the reference.
% (It relied on the case number being given in the |issue| field.)
% To maintain backwards compatibility, we rely on a `trick' that case numbers
% are usually only present in EU cases where a page number is always provided.
% (And we support the \textsf{oscola} semantics for numbering European
% Commission cases.) We have to be a bit careful to ensure the last print
% command prints something otherwise |(no)nameyeardelim| will not be triggered.
%
%    \begin{macrocode}
\DeclareFieldFormat[jurisdiction]{issue}{\mkbibparens{#1}}
\newbibmacro*{casenumber}{%
  \iffieldundef{issue}{%
    \ifboolexpr{
      test {\iffieldundef{pages}}
      or
      test {\iffieldundef{number}}
    }{}{%
      \iffieldundef{userb}{%
        \printfield[parens]{number}%
        \clearfield{number}%
      }{%
        \printfield[parens]{userb}%
        \setunit{\addspace}%
        \printfield{type}%
        \setunit*{\addspace}%
        \printfield{number}%
        \clearfield{type}\clearfield{number}}}%
  }{%
    \printfield{issue}}}
%    \end{macrocode}
%
% For ease, if `Commission' is given as the |institution|, this is deleted but
% the |type| is set to `Commission Decision'.
%
%    \begin{macrocode}
\DeclareStyleSourcemap{
  \maps[datatype=bibtex]{
    \map[overwrite=false]{
      \step[match=\regexp{Commission}, fieldsource=institution, final]
      \step[fieldset=type, fieldvalue={Commission Decision}]
      \step[fieldset=institution, null]
    }
    \map[overwrite=false]{
      \step[fieldsource=casenumber, final]
      \step[notfield=number, fieldsource=casenumber, fieldtarget=number]
      \step[fieldsource=casenumber, fieldtarget=userb]
    }
  }
}

%    \end{macrocode}
%
% Law reports have a special way of printing journals, volume and page numbers.
% Here are several macros implementing this: a deprecated one using
% |type|/|number| and three proper ones using |journaltitle|, for UK journals,
% ECR and OJ respectively. The |reporter| macro chooses between them (again,
% this is a little messy for backwards compatibility).
%
%    \begin{macrocode}
\newbibmacro{jurisdiction:type+number}{%
  \usebibmacro{journal}%
  \setunit*{\addspace}%
  \ifboolexpr{
    test {\iffieldundef{type}}
    and
    test {\iffieldundef{number}}
  }{}{%
    \printfield{volume}%
    \setunit*{\addperiod\space}%
    \printfield{type}%
    \setunit*{\addspace}%
    \printfield{number}}}
\DeclareFieldFormat[jurisdiction]{journaltitle}{%
  \iftoggle{bbx:eu-oj}{\mkbibemph{#1}}{#1}}
\DeclareFieldFormat[jurisdiction,legislation]{volume}{#1}
\DeclareFieldFormat[jurisdiction,legislation]{pages}{#1}
\newbibmacro{journal+volume+pages}{%
  \printfield{volume}%
  \setunit{\addperiod\space}%
  \printfield{journaltitle}%
  \setunit*{\addspace}%
  \printfield{pages}%
}
\newbibmacro{eu:journal+volume+pages}{%
  \printfield{journaltitle}%
  \setunit{\addspace}%
  \printfield{volume}%
  \setunit*{\printtext{--\allowbreak}}%
  \printfield{pages}%
}
\newbibmacro{eu:journal+series+volume+pages}{%
  \printfield{journaltitle}%
  \setunit{\addspace}%
  \printfield{series}%
  \clearfield{series}%
  \printfield{volume}%
  \setunit*{\printtext{/}}%
  \printfield{pages}%
}
\newbibmacro{reporter}{%
  \iffieldundef{journaltitle}{%
    \usebibmacro{jurisdiction:type+number}%
  }{%
    \iffieldequalstr{journaltitle}{ECR}{%
      \usebibmacro{eu:journal+volume+pages}%
    }{%
      \iffieldequalstr{journaltitle}{OJ}{%
        \iffieldundef{series}{%
          \usebibmacro{jurisdiction:type+number}%
        }{%
          \usebibmacro{eu:journal+series+volume+pages}%
        }%
      }{%
        \usebibmacro{journal+volume+pages}%
        }}}}

%    \end{macrocode}
%
% It's not a feature of Harvard (Bath), but OSCOLA puts the institution in
% parentheses at the end.
%
%    \begin{macrocode}
\DeclareFieldFormat[jurisdiction]{institution}{\mkbibparens{#1}}

\DeclareBibliographyDriver{jurisdiction}{%
  \savebibmacro{title}%
  \xapptobibmacro{labeltitle}{%
    \setunit*{\addspace}%
    \usebibmacro{casenumber}%
  }{}{}%
  \usebibmacro{bibindex}%
  \usebibmacro{begentry}%
  \usebibmacro{eucheck}%
  \usebibmacro{author}%
  \setunit{\printdelim{nametitledelim}}\newblock
  \usebibmacro{title}%
  \newunit
  \printlist{language}%
  \newunit\newblock
  \usebibmacro{byauthor}%
  \newunit\newblock
  \printfield{note}%
  \setunit{\addspace}%
  \usebibmacro{reporter}%
  \newunit
  \printfield{pagetotal}%
  \newunit\newblock
  \printfield{institution}%
  \newunit\newblock
  \iftoggle{bbx:isbn}
    {\printfield{isrn}}
    {}%
  \newunit\newblock
  \usebibmacro{doi+eprint+url}%
  \newunit\newblock
  \usebibmacro{addendum+pubstate}%
  \setunit{\bibpagerefpunct}\newblock
  \usebibmacro{pageref}%
  \newunit\newblock
  \iftoggle{bbx:related}
    {\usebibmacro{related:init}%
     \usebibmacro{related}}
    {}%
  \usebibmacro{finentry}%
  \restorebibmacro{title}}

%    \end{macrocode}
%
% \subsection{Legislation}
%
% Legislation is mostly formatted like a report, except there is no
% |nametitledelim| there is no |date+extradate|. There is also some variation
% in how the series\slash number\slash chapter information is punctuated.
%
%    \begin{macrocode}
\DeclareFieldFormat[legislation]{datelabel}{%
  \iftoggle{bbx:eu-oj}{%
    \mkbibbrackets{#1}%
  }{#1}}
\DeclareFieldFormat[legislation]{labeldate}{%
  \iftoggle{bbx:labelistitle}{\printtext[title]{#1}}{#1}}
\DeclareFieldFormat[legislation]{chapter}{\biblcsstring{legalchapter}#1}
\newbibmacro*{journal+series+volume+number+chapter+pages}{%
  \iftoggle{bbx:eu-oj}{%
    \setunit{\addspace}%
    \usebibmacro{eu:journal+series+volume+pages}%
  }{%
    \iffieldequalstr{entrysubtype}{secondary}{%
      \setunit{\addcomma\space}%
      \printfield{number}%
      \clearfield{number}%
      \printunit{\addcomma\space}%
    }{%
      \ifboolexpr{
        test {\iffieldundef{series}}
        and
        test {\iffieldundef{type}}
      }{%
        \iffieldundef{number}{%
          \setunit{\addcomma\space}%
          \printfield{chapter}%
        }{%
          \setunit{\addspace}%
          \printtext[parens]{%
            \printfield{number}%
            \setunit*{\addcomma\space}%
            \printfield{chapter}}}%
      }{%
        \iffieldundef{chapter}{}{\setunit{\addspace}}%
        \printtext[parens]{%
          \printfield{series}%
          \setunit{\addcomma\space}%
          \printfield{type}%
          \setunit*{\addspace}%
          \printfield{number}%
          \setunit*{\addcomma\space}%
          \printfield{chapter}}}}}}

\DeclareBibliographyDriver{legislation}{%
  \usebibmacro{bibindex}%
  \usebibmacro{begentry}%
  \usebibmacro{eucheck}%
  \usebibmacro{author}%
  \setunit{\printdelim{nametitledelim}}\newblock
  \usebibmacro{title}%
  \newunit
  \printlist{language}%
  \newunit\newblock
  \usebibmacro{byauthor}%
  \newunit\newblock
  \usebibmacro{journal+series+volume+number+chapter+pages}
  \newunit\newblock
  \printfield{note}%
  \newunit\newblock
  \usebibmacro{institution+location+date}%
  \newunit\newblock
  \printfield{pagetotal}%
  \newunit\newblock
  \iftoggle{bbx:isbn}
    {\printfield{isrn}}
    {}%
  \newunit\newblock
  \usebibmacro{doi+eprint+url}%
  \newunit\newblock
  \usebibmacro{addendum+pubstate}%
  \setunit{\bibpagerefpunct}\newblock
  \usebibmacro{pageref}%
  \newunit\newblock
  \iftoggle{bbx:related}
    {\usebibmacro{related:init}%
     \usebibmacro{related}}
    {}%
  \usebibmacro{finentry}}

%    \end{macrocode}
%
% \subsection{Letters}
%
% This driver is used for emails. It is based loosely on the one for articles.
% The particular foible with this type is that the date must be printed complete
% at the start.
%
%    \begin{macrocode}
\DeclareFieldFormat[letter]{title}{\iffieldundef{journaltitle}{\emph{#1}}{#1}}
\ExecuteBibliographyOptions[letter]{mergedate=maximum}
\DeclareBibliographyDriver{letter}{%
  \usebibmacro{bibindex}%
  \usebibmacro{begentry}%
  \usebibmacro{author/translator+others}%
  \setunit{\printdelim{nametitledelim}}\newblock
  \usebibmacro{title}%
  \newunit
  \printlist{language}%
  \newunit\newblock
  \usebibmacro{byauthor}%
  \newunit\newblock
  \usebibmacro{bytranslator+others}%
  \newunit\newblock
  \usebibmacro{journal+issuetitle}%
  \newunit\newblock
  \printfield{howpublished}%
  \newunit
  \printfield{note}%
  \newunit\newblock
  \usebibmacro{doi+eprint+url}%
  \newunit\newblock
  \usebibmacro{addendum+pubstate}%
  \setunit{\bibpagerefpunct}\newblock
  \usebibmacro{pageref}%
  \newunit\newblock
  \iftoggle{bbx:related}
    {\usebibmacro{related:init}%
     \usebibmacro{related}}
    {}%
  \usebibmacro{finentry}%
}

%    \end{macrocode}
%
% \subsection{Software}
%
% The driver for software entries is based on misc. The main difference is in
% how the type is printed.
%
%    \begin{macrocode}
\DeclareFieldFormat[software]{type}{\mkbibbrackets{#1}}
\DeclareBibliographyDriver{software}{%
  \usebibmacro{bibindex}%
  \usebibmacro{begentry}%
  \usebibmacro{author/editor+others/translator+others}%
  \setunit{\printdelim{nametitledelim}}\newblock
  \usebibmacro{title}%
  \setunit{\addspace}
  \printfield{type}%
  \newunit
  \printlist{language}%
  \newunit\newblock
  \usebibmacro{byauthor}%
  \newunit\newblock
  \usebibmacro{byeditor+others}%
  \newunit\newblock
  \printfield{howpublished}%
  \newunit\newblock
  \printfield{note}%
  \newunit\newblock
  \usebibmacro{organization+location+date}%
  \newunit\newblock
  \usebibmacro{doi+eprint+url}%
  \newunit\newblock
  \usebibmacro{addendum+pubstate}%
  \setunit{\bibpagerefpunct}\newblock
  \usebibmacro{pageref}%
  \newunit\newblock
  \iftoggle{bbx:related}
    {\usebibmacro{related:init}%
     \usebibmacro{related}}
    {}%
  \usebibmacro{finentry}}

%    \end{macrocode}
%
% \subsection{Images}
%
% The driver for image is based on misc. The main difference is the
% support for the \texttt{library} field.
%
%    \begin{macrocode}
\DeclareFieldFormat[image]{library}{#1}
\DeclareBibliographyDriver{image}{%
  \usebibmacro{bibindex}%
  \usebibmacro{begentry}%
  \usebibmacro{author/editor+others/translator+others}%
  \setunit{\printdelim{nametitledelim}}\newblock
  \usebibmacro{title}%
  \newunit
  \printlist{language}%
  \newunit\newblock
  \usebibmacro{byauthor}%
  \newunit\newblock
  \usebibmacro{byeditor+others}%
  \newunit\newblock
  \printfield{howpublished}%
  \newunit\newblock
  \printfield{type}%
  \newunit
  \printfield{note}%
  \newunit\newblock
  \usebibmacro{organization+location+date+library}%
  \newunit\newblock
  \usebibmacro{doi+eprint+url}%
  \newunit\newblock
  \usebibmacro{addendum+pubstate}%
  \setunit{\bibpagerefpunct}\newblock
  \usebibmacro{pageref}%
  \newunit\newblock
  \iftoggle{bbx:related}
    {\usebibmacro{related:init}%
     \usebibmacro{related}}
    {}%
  \usebibmacro{finentry}}
%    \end{macrocode}
%
% The following code works slightly differently
%
%    \begin{macrocode}
\newbibmacro*{organization+location+date+library}{%
  \ifboolexpr{
    test {\iffieldundef{library}}
    or
    not test {\iflistundef{publisher}}
  }{%
    \printlist{location}%
    \setunit*{\addcolon\space}%
    \clearfield{location}%
  }{}%
  \iflistundef{publisher}{%
    \printlist{organization}%
  }{%
    \printlist{publisher}%
  }%
  \setunit{\addcomma\space}%
  \usebibmacro{date}%
  \newunit
  \iffieldundef{library}{%
    \iffieldundef{institution}{}{%
      \bibsentence
      \bibstring{at}%
      \setunit{\addcolon\space}%
      \printlist{location}%
      \setunit*{\addperiod\space}%
      \printfield{institution}%
    }%
  }{%
    \bibsentence
    \bibstring{at}%
    \setunit{\addcolon\space}%
    \printlist{location}%
    \setunit*{\addperiod\space}%
    \printfield{library}%
  }%
}

%    \end{macrocode}
%
% \subsection{Unpublished}
%
% This driver is used for miscellaneous unpublished written material. It differs
% from the standard version by supporting |maintitle| and |booktitle|, and
% including a label at the end. The unpublished label is delegated to a macro
% in case future versions of the style place conditions on whether it is displayed.
%
%    \begin{macrocode}
\newbibmacro*{isunpublished}{%
  \bibstring{unpublished}%
}
\DeclareBibliographyDriver{unpublished}{%
  \usebibmacro{bibindex}%
  \usebibmacro{begentry}%
  \usebibmacro{author}%
  \setunit{\printdelim{nametitledelim}}\newblock
  \usebibmacro{title}%
  \newunit
  \printlist{language}%
  \newunit\newblock
  \usebibmacro{byauthor}%
  \newunit\newblock
  \ifnameundef{editor}{}{\usebibmacro{in:}}%
  \usebibmacro{bookeditor}%
  \newunit\newblock
  \usebibmacro{maintitle+booktitle}%
  \usebibmacro{byeditor+others}%
  \newunit\newblock
  \printfield{howpublished}%
  \newunit\newblock
  \printfield{type}%
  \newunit\newblock
  \usebibmacro{event+venue+date}%
  \newunit\newblock
  \printfield{note}%
  \newunit\newblock
  \usebibmacro{location+date}%
  \newunit\newblock
  \usebibmacro{isunpublished}%
  \newunit\newblock
  \iftoggle{bbx:url}
    {\usebibmacro{url+urldate}}
    {}%
  \newunit\newblock
  \usebibmacro{addendum+pubstate}%
  \setunit{\bibpagerefpunct}\newblock
  \usebibmacro{pageref}%
  \newunit\newblock
  \iftoggle{bbx:related}
    {\usebibmacro{related:init}%
     \usebibmacro{related}}
    {}%
  \usebibmacro{finentry}}

%    \end{macrocode}
%
% \subsection{Aliases}
%
% We define some handy semantic aliases.
%
%    \begin{macrocode}
\DeclareBibliographyAlias{standard}{manual}
\DeclareBibliographyAlias{dataset}{online}
%    \end{macrocode}
%
% \iffalse
%</bbx>
%<*dbx>
% \fi
%
% \section{Implementation: data model}
%
% \setcounter{lstnumber}{16}
%
%    \begin{macrocode}
\DeclareDatamodelConstant[type=list]{nameparts}{prefix,family,suffix,given,cjk}
%    \end{macrocode}
% \iffalse
%</dbx>
%<*lbx>
% \fi
%
% \section{Implementation: General English language localization}
%
% \setcounter{lstnumber}{16}
%
% The strings are mostly the same except for the following changes. Note that
% month names are never abbreviated.
%
%    \begin{macrocode}
\InheritBibliographyExtras{english}
\DeclareBibliographyStrings
{inherit          = {english}
,editors          = {{editors}{eds}}
,version          = {{version~}{v\adddot}}
,inpreparation    = {{preprint}{preprint}}
,submitted        = {{preprint}{preprint}}
,urlseen          = {{Accessed}{Accessed}}
,january          = {{January}{January}}
,february         = {{February}{February}}
,march            = {{March}{March}}
,april            = {{April}{April}}
,may              = {{May}{May}}
,june             = {{June}{June}}
,july             = {{July}{July}}
,august           = {{August}{August}}
,september        = {{September}{September}}
,october          = {{October}{October}}
,november         = {{November}{November}}
,december         = {{December}{December}}
,patreq           = {{patent application}{pat\adddot\ appl\adddot}}
,patreqde         = {{German patent application}{German pat\adddot\ appl\adddot}}
,patreqeu         = {{European patent application}{European pat\adddot\ appl\adddot}}
,patreqfr         = {{French patent application}{French pat\adddot\ appl\adddot}}
,patrequk         = {{British patent application}{British pat\adddot\ appl\adddot}}
,patrequs         = {{U.S\adddotspace patent application}{U.S\adddotspace pat\adddot\ appl\adddot}}
%    \end{macrocode}
%
% These are the new strings we define in this style.
%
%    \begin{macrocode}
,online           = {{Online}{Online}}
,hours            = {{hours}{hrs\adddot}}
,at               = {{at}{at}}
,unpublished      = {{unpublished}{unpublished}}
,legalchapter     = {{chapter}{c\adddot}}
,director         = {{director}{dir\adddot}}
,directors        = {{directors}{dir\adddot}}
,bydirector       = {{directed by}{directed by}}
,performer        = {{}{}}
,performers       = {{}{}}
,byperformer      = {{}{}}
,reader           = {{reader}{reader}}
,readers          = {{readers}{readers}}
,byreader         = {{read by}{read by}}
,conductor        = {{conductor}{cond\adddot}}
,conductors       = {{conductors}{cond\adddot}}
,byconductor      = {{conducted by}{conducted by}}
}
%    \end{macrocode}
% \iffalse
%</lbx>
%<*lbx-gb>
% \fi
%
% \section{Implementation: British English language localization}
%
% \setcounter{lstnumber}{16}
%
% We use the standard version with a few changes. We display the day as a
% cardinal number instead of an ordinal. The date is separated from the time
% by a period, not just a space, and the time separator is a period instead of a
% colon.
%
%    \begin{macrocode}
\InheritBibliographyExtras{british}
\DeclareBibliographyExtras{%
  \protected\def\mkbibdatelong#1#2#3{%
    \iffieldundef{#3}
      {}
      {\thefield{#3}%
       \iffieldundef{#2}{}{\nobreakspace}}%
    \iffieldundef{#2}
      {}
      {\mkbibmonth{\thefield{#2}}%
       \iffieldundef{#1}{}{\space}}%
    \iffieldbibstring{#1}
      {\bibstring{\thefield{#1}}}
      {\dateeraprintpre{#1}\stripzeros{\thefield{#1}}}}%
  \renewrobustcmd*{\bibdatetimesep}{\addperiod\space}%
  \renewrobustcmd*{\bibtimesep}{\addperiod}%
  }

\DeclareBibliographyStrings
{inherit          = {english}
}
%    \end{macrocode}
% \iffalse
%</lbx-gb>
%<*cbx>
% \fi
%
% \section{Implementation: citation style}
%
% \setcounter{lstnumber}{16}
%
% The standard |authoryear-comp| style is a close match for what we need.
%
%    \begin{macrocode}
\RequireCitationStyle{authoryear-comp}
%    \end{macrocode}
%
% This sets \key{uniquename} to |full|, but that conflicts with
% \key{giveninits} set by the bibliography style, so we set it to |init|
% instead. If left alone, \textsf{biblatex} would do this anyway, but if we do
% it explicitly, we avoid the warning message.
%
%    \begin{macrocode}
\ExecuteBibliographyOptions{uniquename=init}
%    \end{macrocode}
%
% Compressed citations are delimited with a semicolon, just like
% non-compressed citations.
%
%    \begin{macrocode}
\renewcommand*{\compcitedelim}{\addsemicolon\space}
%    \end{macrocode}
%
% We need to suppress the display of `n.d.' in citations if the \key{nonodate}
% toggle is true.
%
%    \begin{macrocode}
\xpatchbibmacro{cite:labeldate+extradate}{%
  \iffieldundef{labelyear}%
}{%
  \ifboolexpr{
    togl {bbx:nonodate}
    and
    not test {\iflabeldateisdate}}%
}{}{}
%    \end{macrocode}
%
% We activate the label-is-title toggle if the |labeltitle| is printed.
%
%    \begin{macrocode}
\providetoggle{bbx:labelistitle}
\xpatchbibmacro{cite:label}{%
  \printtext[bibhyperref]{\printfield[citetitle]{labeltitle}}%
}{%
  \printtext[bibhyperref]{\printfield[citetitle]{labeltitle}}%
  \toggletrue{bbx:labelistitle}%
}{}{}
%    \end{macrocode}
% \iffalse
%</cbx>
% \fi

